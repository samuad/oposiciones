\chapter[Contrastes de hip\'otesis.]{Contrastes de hip\'otesis. \\
\normalsize Errores y potencia de un contraste. Hip\'otesis simples. Lema de Neyman-Pearson.}

\sectioncol{Introducci\'on.}

Dentro del contexto general de la inferencia estad\'istica, veremos el contraste o test de hip\'otesis estad\'isticas. Consiste esta t \'ecnica en formular una hip\'otesis acerca de una poblaci\'on y, bas\'andonos en las observaciones contenidas en una muestra, decidir si podemos rechazarla.

B\'asicamente, se formula una hip\'otesis nula, $H_0$, que es la que queremos validar, frente a una hip\'otesis alternativa $H_1$, que agrupa todos los casos en los que la hip\'otesis nula no es cierta. Si se cumple la hip\'otesis nula, la poblaci\'on presentar\'a una distribuci\'on de probabilidad que cumplir\'a una serie de condiciones, mientras que si no se cumple, la distribuci\'on no cumplir\'a esas condiciones. Teniendo en cuanta estas condiciones, hemos de calcular la probabilidad de que se presente la muestra aleatoria que hemos obtenido si se cumple la hip\'otesis nula, y si esta probabilidad es menor que un umbral que nosotros determinamos, decidimos que no se cumple la hip\'otesis nula y la rechazamos.

Hay que tener varias cosas en cuenta:
\begin{itemize}
\item El contraste de hip\'otesis no decide entre la hip\'otesis nula y la alternativa, solo nos indica si tenemos suficiente evidencia experimental para rechazar la hip\'otesis nula.
\item Por tanto, Los contrastes siempre tendr\'an un sesgo evidente hacia la aceptaci\'on de la hip\'otesis nula.
\item Aquellas hip\'otesis bajo las cuales queda totalmente definida la distribuci\'on de probabilidad de la poblaci\'on se llaman hip\'otesis simples. Aquellas hip\'otesis bajo las cuales la distribuci\'on de probabilidad de la poblaci\'on no queda totalmente determinada se llaman hip\'otesis compuestas.
\item Si la hip\'otesis se refiere al valor de un par\'ametro desconocido de la poblaci\'on hablamos de un contraste param\'etrico. Si no se refiere a ning\'un par\'ametro, si no a la distribuci\'on poblacional globalmente, hablamos de contrastes no param\'etricos.
\end{itemize}

En este tema nos centraremos en los constrastes param\'etricos simples.

\sectioncol{Errores y potencia de un contraste.}
\subsectioncol{Planteamiento general de los contrastes de hip\'otesis.}

Los elementos principales de un contraste de hip\'otesis