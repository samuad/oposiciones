\chapter[\'Indices de desigualdad y medidas de concentraci\'on.]{\'Indices de desigualdad y medidas de concentraci\'on.}

\sectioncol{Introducci\'on.}

El estudio de la desigualdad o de la concentraci\'on, referido casi siempre dentro del \'ambito de la distribuci\'on de rentas dentro de una poblaci\'on, se ve lastrado por la dificultad de definir una medici\'on para la mismas. Al hablar de desigualdad nos encontramos con un concepto intuitivo muy f\'acil de comprender, pero con una dif\'icil medici\'on objetiva.

Esto plantea un problema, ya que si no somos capaces de asignar una medida de desigualdad a nuestros objetos de estudio no seremos capaces de estudiar c\'omo afecta dicha desigualdad a otros factores.

\sectioncol{Medidas de desigualdad.}

Para abordar las medidas de desigualdad establecemos un marco en el que tenemos una poblaci\'on formada por $N$ individuos. Para cada uno de estos individuos se mide un factor (que normalmente suele ser las rentas, los salarios o la riqueza) y se le asigna el valor del mismo. Con estos $N$ individuos se forma un vector, orden\'andolos seg\'un el valor de la magnitud medida. Las medidas de desigualdad se basan en comparar varias realizaciones posibles del vector, que se conocen como distribuciones del factor en estudio.

\subsectioncol{Propiedades deseables.}

Hay una serie de propiedades que nos sugiere la l\'ogica que debe presentar una medida de desigualdad (o indicador de desigualdad) para ser realmente \'util para un estudio de la misma. Las principales son:
\begin{itemize}
\item \textbf{Indepencia de escala:} El indicador no debe varian ante transformaciones proporcionales de la distribuci\'on (por ejemplo, ante cambios en las unidades de medida).
\item \textbf{Independencia del tama\~no de la poblaci\'on:} El indicador debe mantenerse si se agrega un n\'umero proporcional de individuos a todos los niveles de la magnitud.
\item \textbf{Independencia ante cambios de posici\'on:} El indicador no debe variar si dos elementos cualesquiera intercambian su posici\'on en la distribuci\'on, sin que var\'ien los respectivos valores.
\item \textbf{Principio ``d\'ebil'' de transferencias:} La desigualdad debe disminuir si se transfiere parte del valor de un elemento o otro con un valor de la magnitud menor.
\item \textbf{Principio ``fuerte'' de transferencias:} Ante una transmisi\'on de valor de un individuo a otro con un valor menor, la desigualdad debe disminuir m\'as cuanta mayor sea la diferencia entre los valores de los elementos.

\end{itemize}


