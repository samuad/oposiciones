\chapter[An\'alisis de conglomerados.]{An\'alisis de conglomerados. \\
\normalsize  Medidas de disimilaridad. M\'etodos jer\'arquicos aglomerativos: el dendrograma. M\'etodos jer\'arquicos divisivos. M\'etodos no jer\'arquicos de clasificaci\'on.}

\sectioncol{Introducci\'on.}

El estudio de cualquier fen\'omeno o poblaci\'on del mundo real en general presenta la necesidad de estudiar una multitud de individuos que vienen descritos por diversas variables distintas. De esta tarea se encarga el an\'alisis multivariante de datos. Tiene por objeto el estudio de m\'ultiples variables medidas en los elementos de una poblaci\'on. Los objetivos que pretende obtener son:
\begin{enumerate}
\item Resumir el conjunto de variables en unas pocas variables construidas a partir de las originales, con una p\'erdida m\'inima de informaci\'on. Esto nos permite eliminar informaci\'on redundante y facilita el estudio de la informaci\'on, a la vez que puede facilitar su interpretaci\'on.
\item Encontrar agrupaciones en los datos, si existen. De esta forma, podemos agrupar observaciones similares.
\item Clasificar nuevas observaciones en grupos definidos. Es decir, a partir de nuestras variables asignar la probabilidad de que una nueva observaci\'on pertenezca a un grupo o a otro.
\item Relacionar entre s\'i dos conjuntos de variables. De esta forma podemos conocer la relaci\'on entre dos grupos de variables referidas a \'ambitos distintos, o entre las mismas variables en momentos distintos del tiempo.
\end{enumerate}

En este tema nos centraremos en el an\'alisis de conglomerados. Este an\'alisis tiene por objeto investigar la estructura de grupos que pueda haber en los datos. Intuitivamente, si representamos nuestros elementos en un espacio con tantas dimensiones como variables estemos estudiando, pertenecer\'an a un mismo grupo aquellos elementos que est\'en cercanos entre s\'i, y a la vez alejados del resto de elementos no pertenecientes al grupo. El an\'alisis de conglomerados estudia tres tipos de problemas:
\begin{enumerate}
\item \textit{Partici\'on de los datos:} Disponemos de un conjunto de datos que se sospecha son heterog\'eneos y se desea dividirlos en un n\'umero predeterminado de grupos de manera que:
\begin{itemize}
\item Cada elemento pertenezca a uno y solo uno de los grupos.
\item Todo elemento pertenezca a un grupo.
\item Los elementos que pertenezcan a cada grupo sean homog\'eneos.
\end{itemize}
\item \textit{Construcci\'on de jerarqu\'ias:} Deseamos estructurar los elementos de forma jer\'arquica por su similitud. Una clasificaci\'on jer\'arquica implica que los elementos se ordenan por niveles, de manera que los niveles superiores contienen a los inferiores. Estrictamente estas ordenaciones no definen grupos, sino la estructura de asociaci\'on en cadena que pueda existir entre los elementos. Sin embargo, una jerarqu\'ia permite obtener una partici\'on de los elementos en grupos.
\item \textit{Clasificaci\'on de variables:} En problemas con muchas variables puede ser interesante dividirlas en grupos por similaridad. Mediante esta divisi\'on podemos orientarlnos a la hora de acometer procesos de reducci\'on de la dimensi\'on.
\end{enumerate}

Los m\'etodos de partici\'on utilizan la matriz de datos, y los m\'etodos jer\'arquicos utilizan la matriz de distancias o similitudes entre elementos. Para clasificar variables se parte de la matriz de relaci\'on entre variables: para variables cont\'inuas se suele utilizar la matriz de correlaci\'on, y para variables discretas se construye a partir de la distancia ji-cuadrado.

\sectioncol{Medidas de disimilaridad.}

\sectioncol{M\'etodos jer\'arquicos aglomerativos: el dendrograma.}

\sectioncol{M\'etodos jer\'arquicos divisivos.}

\sectioncol{M\'etodos no jer\'arquicos de clasificaci\'on.}

