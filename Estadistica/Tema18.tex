\chapter[Distribuciones en el muestreo asociadas con poblaciones normales.]{Distribuciones en el muestreo asociadas con poblaciones normales. \\
\normalsize Distribuciones de la media, varianza y diferencia de medias. Estad\'isticos ordenados. Distribuci\'on del mayor y menor valor. Distribuci\'on del recorrido.}

\section{Introducci\'on.}

La distribuci\'on normal est\'a presente en muchos \'ambitos de la ciencia, la econom\'ia y la ingenier\'ia. Adem\'as, por el Teorema Central del L\'imite, sabemos que la suma de variables aleatorias independientes ind\'enticamente distribuidas con media y varianza finitas converge en distribuci\'on a una normal. Por tanto, todos aquellos fen\'omenos que resulten de la adici\'on de un gran n\'umero de efectos aleatorios que cumplan esta condici\'on podr\'an describirse, al menos en primera aproximaci\'on, mediante una distribuci\'on normal.

\section{Distribuciones en el muestreo asociadas con poblaciones normales.}

Sabemos que la funci\'on caracter\'istica de una suma de variables aleatorias independientes coincide con el producto de sus funciones caracter\'isticas. Tambi\'en sabemos que $\varphi_{c\varepsilon}(t)=\varphi_{\varepsilon}(ct)$. Calculemos la funci\'on caracter\'stica de una combinaci\'on lineal de variables normales, $(X_1,\ldots,X_n)$:

\begin{align*}
  \varphi_{a_1X_1+a_2X_2+\cdots+a_nX_n}(t) & = \varphi_{X_1}(a_1t)\varphi_{X_2}(a_2t)\cdots\varphi_{X_n}(a_nt)  \\
  \varphi_{a_1X_1+a_2X_2+\cdots+a_nX_n}(t) & = e^{ia_1t\mu-\frac{1}{2}\sigma^2a_1^2t^2}e^{ia_2t\mu-\frac{1}{2}\sigma^2a_2^2t^2} \cdots e^{ia_nt\mu-\frac{1}{2}\sigma^2a_n^2t^2} \\
  \varphi_{a_1X_1+a_2X_2+\cdots+a_nX_n}(t) & = e^{it(a_1+a_2+\cdots+a_n)\mu-\frac{1}{2}(a_1^2+a_2^2+\cdots+a_n^2)\sigma^2t^2} 
\end{align*}

Y por tanto una combinaci\'on lineal de variables aleatorias normales es otra variable aleatoria normal, as\'i cunado tengamos estad\'isticos de poblaciones normales que sean combinaci\'on lineal de las observaciones, conoceremos su funci\'on de distribuci\'on muestral.

\section{Distribuciones de la media, varianza y diferencia de medias.}

\subsection{Distribuci\'on de la media muestral de una poblaci\'on $N(\mu,\sigma)$.}

\subsubsection{Varianza Poblacional conocida.}

Sabemos que $E(\bar{X})=\mu$ y $Var(\bar{X})=\dfrac{\sigma^2}{n}$ para cualquier poblaci\'on, sea o no normal.

Como la poblaci\'on es normal, $\bar{X}\sim N(\mu,\dfrac{\sigma}{\sqrt{n}})$ y por tanto:
\begin{equation*}
\dfrac{\bar{X}-\mu}{\dfrac{\sigma}{\sqrt{n}}}\sim N(0,1)
\end{equation*}
