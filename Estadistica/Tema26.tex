\chapter[Contrastes de bondad de ajuste.]{Contrastes de bondad de ajuste. \\
\normalsize Contraste $\chi^2$ de Pearson. Contraste de Kolmogorov-Smirnov. Contrastes de normalidad. Contrastes de independencia. Contraste de homogeneidad.}

\sectioncol{Introducci\'on.}
Hasta ahora los contrastes que hemos visto se han basado en una poblaci\'on para la que se conoce su distribuci\'on de probabilidad y se desconocen uno o varios par\'ametros que definen dicha distribuci\'on. Sin embargo, esto no es lo habitual, y el asignar una distribuci\'on de probabilidad subyacente err\'onea hace que en muchos casos el contraste deje de tener validez, o, de tenerla, disminuya mucho su eficiencia.

Para obviar este problema se han desarrollado las t\'ecnicas no param\'etricas, en las que el conjunto de hip\'otesis de partida se reducen, e incluso desaparecen, con lo que disminuye el riesgo de errores dentro del proceso inferencial debido a una mala formulaci\'on de dichas hip\'otesis.

Dentro de los contrastes no param\'etricos est\'an los de bondad del ajuste. El objetivo de los contrastes de bondad del ajuste es verificar si una muestra proviene de una poblaci\'on con una determinada distribuci\'on de probabilidad. En general la hip\'otesis nula asume que la distribuci\'on propuesta es la correcta. 


\sectioncol{Contraste $\chi^2$ de Pearson.}

Es el m\'as antiguo. Fue introducido por Pearson en 1900. La idea b\'asica consiste en comparar las frecuencias observadas en un histograma con las esperadas para el modelo te\'orico que se contrasta. Es v\'alido para todo tipo de distribuciones.