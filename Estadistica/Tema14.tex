
\chapter[Convergencias de sucesiones de variables aleatorias.]{Convergencias de sucesiones de variables aleatorias.\\
\normalsize Convergencia casi segura, convergencia en probabilidad, convergencia en media cuadr\'atica, convergencia en ley. Relaciones entre ellas. Convergencia de sumas de variables aleatorias. Leyes d\'ebiles y fuertes de los grandes n\'umeros. Aplicaciones a la inferencia estad\'istica y al muestreo. Teorema Central del L\'imite.}


\sectioncol{Introducci\'on.}

Hemos visto que una variable aleatoria es una funci\'on medible definida de un espacio de probabilidad en $\mathbb{R}$.

Definimos como una sucesi\'on de variables aleatorias a un conjunto $\left\{X_n\right\}_{n\in\mathbb{N}}$, de variables aleatorias definidas sobre el mismo espacio probabil\'istico. Es de inter\'es desde el punto de vista de la probabilidad el comportamiento de dicha sucesi\'on cuando el valor de $n$ tiende a infinito. De esta forma, definiremos los diversos tipos de convergencia de variables aleatorias.

\sectioncol{Convergencia de variables aleatorias.}
\subsectioncol{Convergencia en probabilidad.}

\begin{definicion}

Sea $ \left\{X_n\right\}_{n\in\mathbb{N}} $ una sucesi\'on de variables aleatorias definidas sobre el espacio probabil\'istico $\left(\Omega,\mathbb{A}, \mathbb{P}\right) $, se dice que converge en probabilidad a otra variable aleatoria $X$ definida sobre el mismo espacio y se denota por $X_{n} \overset{p}{\to} X $ si se cumple que:

\begin{equation}
P\left\{ \omega\in\Omega:\left|X_{n}(\omega)-X(\omega)\right|>\varepsilon\right\} <\alpha
\end{equation}

\end{definicion}
o, lo que es lo mismo, que para todo $\varepsilon > 0$:

\begin{equation}
\lim_{n\rightarrow\infty}P\left\{ \omega\in\Omega:\left|X_{n}(\omega)-X(\omega)\right|>\varepsilon\right\} =0
\end{equation}

Es decir, que para un $ n $ suficientemente grande, la diferencia entre $ X_n $ y la variable $ X $ es muy peque\~na con probabilidad muy alta.

Como caso particular, diremos que $ \left\{X_n\right\}_{n\in\mathbb{N}} $ converge a una constante $ a $ si:
\begin{equation}
\lim_{n\rightarrow\infty}P\left\{ \omega\in\Omega:\left|X_{n}(\omega)-a\right|>\varepsilon\right\} =0, \;\forall\varepsilon>0
\end{equation}

\subsubsectioncol{Propiedades.}

\begin{enumerate}
\item $X_{n} \overset{p}{\to} X \Leftrightarrow X_{n}-X \overset{p}{\to} 0 $.
\item Si $X_{n} \overset{p}{\to} X $ y $X_{n} \overset{p}{\to} Y $, entonces $P\left(X=Y\right)=1$.
\item Si $X_{n} \overset{p}{\to} X $, entonces Si $X_{n}-X_{m} \overset{p}{\to} 0 $ cuando $n,m\to\infty$.
\item Si $X_{n} \overset{p}{\to} X $ y $Y_{n} \overset{p}{\to} Y $ entonces $X_{n}\pm Y_{n} \overset{p}{\to} X\pm Y $.
\item Si $X_{n} \overset{p}{\to} X $ y $k$ es una constante, entonces $kX_{n} \overset{p}{\to} kX $.
\item Si $X_{n} \overset{p}{\to} a $ y $a$ es una constante, entonces $X_{n}^2 \overset{p}{\to} a^2 $.
\item Si $X_{n} \overset{p}{\to} a $ y $Y_{n} \overset{p}{\to} b $  $a,b$ constantes, entonces $X_{n}Y_{n} \overset{p}{\to} ab $.
\item Si $X_{n} \overset{p}{\to} 1 $ y $X_{n}(\omega)\neq 0, \forall\omega\in\Omega$, entonces $X_{n}^{-1} \overset{p}{\to} 1 $.
\item Si $X_{n} \overset{p}{\to} a $ y $Y_{n} \overset{p}{\to} b $  $a,b$ constantes, $b\neq0$, entonces $X_{n}Y_{n}^{-1} \overset{p}{\to} ab^{-1} $.
\item Si $X_{n} \overset{p}{\to} X $ y $ Y $, es una variable aleatoria, entonces $YX_{n} \overset{p}{\to} YX $.
\item Si $X_{n} \overset{p}{\to} X $ y $Y_{n} \overset{p}{\to} Y $ entonces $X_{n}Y_{n} \overset{p}{\to} XY $.
\end{enumerate}

\begin{teorema}
Si $X_{n} \overset{p}{\to} X $, y $g$ es una funci\'on cont\'inua definida sobre $\mathbb{R}$, entonces $g(X_{n}) \overset{p}{\to} g(X) $
\end{teorema}

\begin{corolario}
Si $X_{n} \overset{p}{\to} c $, donde $c$ es constante, y $g$ es una funci\'on cont\'inua definida sobre $\mathbb{R}$, entonces $g(X_{n}) \overset{p}{\to} g(c) $
\end{corolario}

\subsectioncol{Convergencia casi segura.}
\begin{definicion}
Sea $ \left\{X_n\right\}_{n\in\mathbb{N}} $ una sucesi\'on de variables aleatorias definidas sobre el espacio probabil\'istico $\left(\Omega,\mathbb{A}, \mathbb{P}\right) $, se dice que converge casi seguramente a otra variable aleatoria $X$ definida sobre el mismo espacio y se denota por $X_{n} \overset{c.s.}{\to} X $ si se cumple que:

\begin{equation}
P\left\{ \omega\in\Omega:\lim_{n\to\infty}X_{n}(\omega)=X(\omega)\right\}=1
\end{equation}

\end{definicion}

Es decir, la probabilidad del conjunto del espacio probabil\'istico donde la sucesi\'on num\'erica de variables converge puntualmente a la variable es igual a uno.

\subsubsectioncol{Propiedades.}
\begin{enumerate}
\item Si $X_{n} \overset{c.s.}{\to} X $, entonces $X_{n} \overset{p}{\to} X $. En efecto, por la convergencia casi segura, podemos elegir un $n_0$ tal que ... La rec\'iproca no es cierta.
\item Si  $ \left\{X_n\right\}_{n\in\mathbb{N}} $ es una sucesi\'on estrictamente decreciente de variables aleatorias positivas, entonces $X_{n} \overset{p}{\to} 0$ implica que $X_{n} \overset{c.s.}{\to} 0 $.
\end{enumerate}

\subsectioncol{Convergencia en ley.}
\begin{definicion}
Sea $ \left\{X_n\right\}_{n\in\mathbb{N}}$ una sucesi\'on de variables aleatorias definidas sobre el espacio probabil\'istico $\left(\Omega,\mathbb{A}, \mathbb{P}\right) $, se dice que converge en ley o en distribuci\'on a otra variable aleatoria $X$ definida sobre el mismo espacio y se denota por $X_{n} \overset{\ell}{\to} X $ si y solo si la correspondiente sucesi\'on de funciones de distribuci\'on de las variables aleatorias $ \left\{X_n\right\}_{n\in\mathbb{N}}$, denotada por $ \left\{F_n\right\}_{n\in\mathbb{N}} $, converge a la funci\'on de distribuci\'on de la variable aleatoria $X$ en todo punto de continuidad de esta funci\'on, es  decir, si:

\begin{equation}
\lim_{n\to\infty}F_n(x)=F(x)\;\forall x/F(x+0)=F(x-0)
\end{equation}

\end{definicion}

\subsubsectioncol{Propiedades.}

\begin{enumerate}
\item 
\begin{teorema}
    La convergencia en probabilidad implica la convergencia en ley. Es decir, 
     \begin{equation}
X_{n} \overset{p}{\to} X \Rightarrow X_{n} \overset{\ell}{\to} X
\end{equation}
\end{teorema}
\item 
\begin{teorema}
    Sea una sucesi\'on de variables aleatorias, con funci\'on de masa $p_n(k)=P\left\{X_n=k\right\}$ y sea la variable aleatoria $X$ con funci\'on de masa $p(k)=P\left\{X=k\right\}$, entonces:
     \begin{equation}
p_{n}(k)\to p(k)\;\forall k \Leftrightarrow X_{n} \overset{\ell}{\to} X
\end{equation}
\end{teorema}
\item 
\begin{teorema}
    Sea una sucesi\'on de variables aleatorias cont\'inuas, con funci\'on de densidad $f_n(x)$ y sea la variable aleatoria cont\'inua $X$ con funci\'on de densidad $f(x)$, y se cumple que $\lim_{n\to\infty}f_n(x)=f(x)$ para casi todo $x$, entonces $X_{n} \overset{\ell}{\to} X$.
\end{teorema}
\item  Si $X_{n} \overset{\ell}{\to} X $ y $c$ es una constante, entonces $X_{n}+c \overset{\ell}{\to} X+c $ y $cX_{n} \overset{\ell}{\to} cX $ si $c\neq0$.
\item  Si $k$ es una constante y $X_{n} \overset{\ell}{\to} k $, entonces $X_{n} \overset{p}{\to} k $.
\end{enumerate}

\subsectioncol{Convergencia en media cuadr\'atica.}
\begin{definicion}
Sea $ \left\{X_n\right\}_{n\in\mathbb{N}}$ una sucesi\'on de variables aleatorias definidas sobre el espacio probabil\'istico $\left(\Omega,\mathbb{A}, \mathbb{P}\right) $, y supongamos que se cumple $E[|X_n|^2]<\infty$, $\forall n\in\mathbb{N}$. Se dice que $ \left\{X_n\right\}_{n\in\mathbb{N}}$ converge en media cuadr\'atica hacia la variable aleatoria $X$ definida sobre el mismo espacio y se denota por $X_{n} \overset{m.c.}{\to} X $ si y solo si:

\begin{equation}
\lim_{n\to\infty}E[(X_n-X)^2]=0
\end{equation}

\end{definicion}

\subsubsectioncol{Propiedades.}
\begin{enumerate}
\item Si $X_{n} \overset{m.c.}{\to} X $, entonces $X_{n} \overset{p}{\to} X $.
\item Si $X_{n} \overset{m.c.}{\to} X $, entonces $E[X_n]\underset{n\to\infty}{\to}E[X]$ y $E[X_n^2]\underset{n\to\infty}{\to}E[X^2]$.
\item Si $X_{n} \overset{m.c.}{\to} X $, entonces $V[X_n]\underset{n\to\infty}{\to}V[X]$.
\item Sean $ \left\{X_n\right\}_{n\in\mathbb{N}}$, $ \left\{Y_m\right\}_{m\in\mathbb{N}}$ dos sucesiones de variables aleatorias tales que $X_{n} \overset{m.c.}{\to} X $ y $Y_{m} \overset{m.c.}{\to} Y $, entonces $E[X_nY_n]\underset{m,n\to\infty}{\to}E[XY]$.
\item Sean $ \left\{X_n\right\}_{n\in\mathbb{N}}$, $ \left\{Y_m\right\}_{m\in\mathbb{N}}$ dos sucesiones de variables aleatorias tales que $X_{n} \overset{m.c.}{\to} X $ y $Y_{m} \overset{m.c.}{\to} Y $, entonces $Cov[X_n,Y_n]\underset{m,n\to\infty}{\to}Cov[X,Y]$

\end{enumerate}
\subsectioncol{Convergencia de las funciones caracter\'isticas.}

La funci\'on caracter\'istica es una forma de caracterizar la distribuci\'on de probabilidad de una variable aleatoria. Parece l\'ogico, por tanto, que tenga alguna relaci\'on con la convergencia en ley de las mismas.

En concreto, se puede demostrar que, sea $ \left\{X_n\right\}_{n\in\mathbb{N}}$ una sucesi\'on de variables aleatorias definidas sobre el espacio probabil\'istico $\left(\Omega,\mathbb{A}, \mathbb{P}\right) $, con funciones caracter\'isticas $\varphi_n(t)$, y sea $X$ una variable aleatoria definida sobre el mismo espacio y con funci\'on caracter\'istica $\varphi(t)$, se cumple:
\begin{enumerate}
\item Si $X_{n} \overset{\ell}{\to} X $, $\Rightarrow \lim_{n\to\infty}\varphi_n(t)=\varphi(t)$, $\forall t\in\mathbb{R}$.
\item Si $\lim_{n\to\infty}\varphi_n(t)=\varphi(t)$, $\forall t\in\mathbb{R}$ y $\varphi(t)$ es cont\'inua en $t=0$, entonces $X_{n} \overset{\ell}{\to} X $.
\end{enumerate}
El primer apartado se demuestra mediante el teorema de Levy:
\begin{teorema}
Sea $ \left\{X_n\right\}_{n\in\mathbb{N}}$ una sucesi\'on de variables aleatorias tales que $X_{n} \overset{\ell}{\to} X $, siendo $F_n(x)$ y $F(x)$ las funciones de distribuci\'on de las $X_n$ y $X$, respectivamente, y sea 
\begin{equation}
\varphi_n(t)=\int_{\mathbb{R}}e^{itx}dF_n(x)
\end{equation}
entonces,
\begin{equation}
\lim_{n\to\infty}\varphi_n(t)=\int_{\mathbb{R}}e^{itx}dF(x)
\end{equation}
y ese l\'imite se alcanza uniformemente en todo intervalo finito de $t$.

\end{teorema}
El segundo apartado es debido a Cramer. 

\sectioncol{Leyes de los grandes n\'umeros.}

La interpretaci\'on frecuentista de la probabilidad establece que, dado un experimento aleatorio, si consideramos un suceso $A$, la frecuencia relativa de aparici\'on de dicho suceso tiende a estabilizarse hacia un valor definido a medida que aumentamos las repeticiones del experimento. A este valor se le llamar\'ia probabilidad del suceso $A$.

Esta interpretaci\'on frecuentista de la probabilidad tiene una caracterrizaci\'on matem\'atica dentro de la concepci\'on axiom\'atica de la probabilidad a trav\'es de las conocidas como \textbf{leyes de los grandes n\'umeros}.

\subsectioncol{Ley d\'ebil de los grandes n\'umeros.}

\begin{definicion}
Sea $ \left\{X_n\right\}_{n\in\mathbb{N}}$ una sucesi\'on de variables aleatorias definidas sobre el espacio probabil\'istico $\left(\Omega,\mathcal{A}, P\right) $ tales que existe $E[X_i]=\alpha_i<\infty,\forall i \in\mathbb{N}$; sea
\begin{equation*}
\bar{X}_n=\dfrac{1}{n}\sum_{i=1}^{n}X_i.
\end{equation*}

Diremos que la sucesi\'on $ \left\{X_n\right\}_{n\in\mathbb{N}}$ obedece a la ley d\'ebil de los grandes n\'umeros, y lo denotaremos por $ \left\{X_n\right\}_{n\in\mathbb{N}}\in\mathscr{D}$ si y solo si las sucesi\'on $\{\bar{X}_n-a_n\}_{n\in\mathbb{N}}$ converge en probabilidad hacia cero, siendo 
\begin{equation*}
a_n=\dfrac{1}{n}\sum_{i=1}^{n}E[X_i]=E[\bar{X}_n].
\end{equation*}
Es decir,
\begin{equation*}
\left\{X_n\right\}_{n\in\mathbb{N}}\in\mathscr{D}\Leftrightarrow \bar{X}_n-a_n\overset{p}{\to} 0.
\end{equation*}
\end{definicion}

En t\'erminos m\'as generales, se dice que una sucesi\'on $ \left\{X_n\right\}_{n\in\mathbb{N}}$ obedece a la ley d\'ebil de los grandes n\'umeros respecto de las constantes de normalizaci\'on $B_n>0$ si existe una sucesi\'on de constantes $ \left\{A_n\right\}_{n\in\mathbb{N}}$, llamadas de centralizaci\'on tales que:
\begin{equation*}
\dfrac{S_n-A_n}{B_n}\overset{p}{\to}0
\end{equation*}

con $S_n=n\bar{X}_n=\sum_{i=1}^{n}X_i$. Llegamos a la definici\'on inicial si $B_n=n$ y $A_n=\sum_{i=1}^{n}E[X_i]$.

Veremos ahora una serie de teoremas que establecen condiciones bajo las cuales una sucesi\'on verifica la ley d\'ebil de los grandes n\'umeros.

\subsubsectioncol{Teoremas.}
\begin{teorema}
Sea $ \left\{X_n\right\}_{n\in\mathbb{N}}$ una sucesi\'on de variables aleatorias independientes definidas sobre el espacio probabil\'istico $\left(\Omega,\mathcal{A}, P\right) $ y tales que 
\begin{itemize}
\item Las variables aleatorias de la sucesi\'on est\'an id\'enticamente distribuidas.
\item Las $X_n$ tienen media y varianza finitas, $\forall n\in\mathbb{N}$.
\end{itemize}
Entonces $ \left\{X_n\right\}_{n\in\mathbb{N}}\in\mathscr{D}$.
\end{teorema}

Como las variables tienen la misma distribuci\'on, tendr\'an la misma media y varianza, por tanto, $\alpha_n=E[X_n]=\mu$, $V(X_n)=\sigma^2<\infty$, $\forall n\in\mathbb{N}$. Por tanto, $a_n=E[\bar{X}_n]=\mu$, $V(\bar{X}_n)=\dfrac{\sigma^2}{n}$ por ser las variables independientes. As\'i pues,
\begin{equation*}
\lim_{n\to\infty}P\left\{\omega\in\Omega/\left|\bar{X}_n-a_n\right|>\varepsilon\right\}=\lim_{n\to\infty}P\left\{\omega\in\Omega/\left|\bar{X}_n-\mu\right|>\varepsilon\right\}\le \dfrac{E[(\bar{X}_n-\mu)^2]}{\varepsilon^2}=\lim_{n\to\infty}\dfrac{\sigma^2}{n\varepsilon^2}=0
\end{equation*}
Donde hemos utilizado la acotaci\'on de Tchebychev, por tanto, $\bar{X}_n-a_n\overset{p}{\to} 0$ y $\left\{X_n\right\}_{n\in\mathbb{N}}\in\mathscr{D}$.

\begin{teorema}[Tchebychev]
Sea $ \left\{X_n\right\}_{n\in\mathbb{N}}$ una sucesi\'on de variables aleatorias independientes definidas sobre el espacio probabil\'istico $\left(\Omega,\mathcal{A}, P\right) $ y tales que toda variable $X_n$ tiene varianza acotada, es decir, $V(X_n)=\sigma_n^2<c$, $\forall n\in\mathbb{N}$ para alguna cota $c$, entonces $ \left\{X_n\right\}_{n\in\mathbb{N}}\in\mathscr{D}$.
\end{teorema}
Como $a_n=E[\bar{X}_n]=\dfrac{1}{n}\sum_{i=1}^nX_i$ y $V(\bar{X}_n)=\dfrac{\sum_{i=1}^nV(X_i)}{n^2}\leq\dfrac{c}{n}$, para todo $\varepsilon>0$ tenemos que
\begin{equation*}
\lim_{n\to\infty}P\left\{\omega\in\Omega/\left|\bar{X}_n-a_n\right|>\varepsilon\right\}\le \dfrac{E[(\bar{X}_n-a_n)^2]}{\varepsilon^2}=\dfrac{V(\bar{X}_n)}{\varepsilon^2}\leq\lim_{n\to\infty}\dfrac{c}{n\varepsilon^2}=0
\end{equation*}
Donde hemos utilizado la acotaci\'on de Tchebychev, por tanto, $\bar{X}_n-a_n\overset{p}{\to} 0$ y $\left\{X_n\right\}_{n\in\mathbb{N}}\in\mathscr{D}$.

\begin{teorema}[Markov]
Sea $ \left\{X_n\right\}_{n\in\mathbb{N}}$ una sucesi\'on de variables aleatorias definidas sobre el espacio probabil\'istico $\left(\Omega,\mathcal{A}, P\right) $ y tales que $\lim_{n\to\infty}V(\bar{X}_n)=0$, entonces $ \left\{X_n\right\}_{n\in\mathbb{N}}\in\mathscr{D}$.
\end{teorema}

La demostraci\'on es inmediata utilizando la desigualdad de Tchebychev.

\begin{teorema}[Khintchine]
Sea $ \left\{X_n\right\}_{n\in\mathbb{N}}$ una sucesi\'on de variables aleatorias independientes definidas sobre el espacio probabil\'istico $\left(\Omega,\mathcal{A}, P\right) $ y tales que 
\begin{itemize}
\item Las variables aleatorias de la sucesi\'on est\'an id\'enticamente distribuidas, con funci\'on de distribuci\'on $F$.
\item Las $X_n$ tienen media finita, $\forall n\in\mathbb{N}$.
\end{itemize}
Entonces $ \left\{X_n\right\}_{n\in\mathbb{N}}\in\mathscr{D}$.
\end{teorema}

\begin{teorema}[Teorema de Bernouilli]
Sea $ \left\{X_n\right\}_{n\in\mathbb{N}}$ una sucesi\'on de variables aleatorias independientes definidas sobre el espacio probabil\'istico $\left(\Omega,\mathcal{A}, P\right) $ y tales que $X_n\sim B(1,p)$, entonces $ \left\{X_n\right\}_{n\in\mathbb{N}}\in\mathscr{D}$.
\end{teorema}

Dado que las variables aleatorias son independientes e id\'enticamente distribuidas, con $E[X_n]=p$, $V8X_N)=p(1-p)$, ambas finitas, si aplicamos el primer teorema, vemos que la sucesi\'on cumple la ley d\'ebil de los grandes n\'umeros.

Adem\'as, $a_n=\dfrac{1}{n}\sum_{i=1}^{n}E[X]_i=1\cdot p + 0\cdot(1-p)=p$, y como $ \left\{X_n\right\}_{n\in\mathbb{N}}\in\mathscr{D}$, $ \bar{X}_n-a_n\overset{p}{\to} 0$, o sea, la frecuencia relativa de los \'exitos en el experimento de Bernouilli converge en probabilidad a la probabilidad de \'exito, que es equivalente a la interpretaci\'on frecuentista de la probabilidad.

\subsectioncol{Ley fuerte de los grandes n\'umeros.}

\begin{definicion}
Sea $ \left\{X_n\right\}_{n\in\mathbb{N}}$ una sucesi\'on de variables aleatorias definidas sobre el espacio probabil\'istico $\left(\Omega,\mathcal{A}, P\right) $ tales que existe $E[X_i]=\alpha_i<\infty,\forall i \in\mathbb{N}$; sea
\begin{equation*}
\bar{X}_n=\dfrac{1}{n}\sum_{i=1}^{n}X_i.
\end{equation*}

Diremos que la sucesi\'on $ \left\{X_n\right\}_{n\in\mathbb{N}}$ obedece a la ley fuerte de los grandes n\'umeros, y lo denotaremos por $ \left\{X_n\right\}_{n\in\mathbb{N}}\in\mathscr{F}$ si y solo si las sucesi\'on $\{\bar{X}_n-a_n\}_{n\in\mathbb{N}}$ converge hacia cero casis seguro, siendo 
\begin{equation*}
a_n=\dfrac{1}{n}\sum_{i=1}^{n}E[X_i]=E[\bar{X}_n].
\end{equation*}
Es decir,
\begin{equation*}
\left\{X_n\right\}_{n\in\mathbb{N}}\in\mathscr{D}\Leftrightarrow \bar{X}_n-a_n\overset{p}{\to} 0.
\end{equation*}
\end{definicion}

En t\'erminos m\'as generales, se dice que una sucesi\'on $ \left\{X_n\right\}_{n\in\mathbb{N}}$ obedece a la ley fuerte de los grandes n\'umeros respecto de las constantes de normalizaci\'on $B_n>0$ si existe una sucesi\'on de constantes $ \left\{A_n\right\}_{n\in\mathbb{N}}$, llamadas de centralizaci\'on tales que:
\begin{equation*}
\dfrac{S_n-A_n}{B_n}\overset{c.s.}{\to}0
\end{equation*}

con $S_n=n\bar{X}_n=\sum_{i=1}^{n}X_i$. Llegamos a la definici\'on inicial si $B_n=n$ y $A_n=\sum_{i=1}^{n}E[X_i]$.

Veremos ahora una serie de teoremas que establecen condiciones bajo las cuales una sucesi\'on verifica la ley fuerte de los grandes n\'umeros.

\subsubsectioncol{Teoremas.}

\textbf{Desigualdad de Kolmogorov:} Es una generalizaci\'on de la desigualdad de Tchebychev.

Sea $ \left\{X_n\right\}_{n\in\mathbb{N}}$ una sucesi\'on de variables aleatorias definidas sobre el espacio probabil\'istico $\left(\Omega,\mathcal{A}, P\right) $ tales que existe $E[X_n]=\mu_n<\infty, V(X_n)=\sigma_n^2<\infty,\forall i \in\mathbb{N}$. Entonces para todo $H$ positivo se verifica:
\begin{equation*}
P\left(\left\{\bigcup_{k=1}^n\{\omega\in\Omega/\left|S_n-E[S_n]\right|\geq HV_n\}\right\}\right)\leq\dfrac{1}{H^2}
\end{equation*}

donde
\begin{equation*}
S_n=\sum_{i=1}^nX_i\;\;V_n^2=V(S_n)=\sum_{i=1}^n\sigma_i^2
\end{equation*}

\begin{teorema}[Kolmogorov]
Sea $ \left\{X_n\right\}_{n\in\mathbb{N}}$ una sucesi\'on de variables aleatorias definidas sobre el espacio probabil\'istico $\left(\Omega,\mathcal{A}, P\right) $ tales que existe $E[X_n]=\mu_n, V(X_n)=\sigma_n^2,\forall i \in\mathbb{N}$ y se cumple que $\sum_{i=1}^{\infty}\sigma^2_i<\infty$, entonces $ \left\{X_n\right\}_{n\in\mathbb{N}}\in\mathscr{F}$.
\end{teorema}

Como corolario, una sucesi\'on con media y varianza acotada cumple la ley fuerte de los grandes n\'umeros.
\begin{teorema}[Borel-Cantelli]
La frecuencia relativa de un suceso dicot\'omico obedece a la ley fuerte de los grandes n\'umeros.
\end{teorema}

\begin{teorema}[Khintchine]
Sea $ \left\{X_n\right\}_{n\in\mathbb{N}}$ una sucesi\'on de variables aleatorias independientes e id\'enticamente distribuidas con media finita $\mu$. Entonces $\left\{X_n\right\}_{n\in\mathbb{N}}\in\mathscr{F}$.
\end{teorema}

\sectioncol{Teorema central del l\'imite.}

\subsectioncol{Introducci\'on.}


\subsectioncol{Definici\'on.}
\begin{definicion}
Sea $ \left\{X_n\right\}_{n\in\mathbb{N}}$ una sucesi\'on de variables aleatorias definidas sobre el espacio probabil\'istico $\left(\Omega,\mathcal{A}, P\right) $, con medias y varianza finitas. Diremos que la sucesi\'on obedece al teorema central del l\'imite, y lo denotaremos por $\left\{X_n\right\}_{n\in\mathbb{N}}\in LN$ si y solo si la sucesi\'on $\left\{S_n\right\}_{n\in\mathbb{N}}$ definida por $S_n=\sum_{i=1}^nX_i$ converge en ley hacia una distribuci\'on normal. Es decir:
\begin{equation*}
\left\{X_n\right\}_{n\in\mathbb{N}}\in LN\Leftrightarrow \dfrac{S_n-E[S_n]}{D[S_n]}\overset{\ell}{\to}N(0,1)
\end{equation*}
\end{definicion}

\begin{teorema}[De Moivre]
Sea $ \left\{X_n\right\}_{n\in\mathbb{N}}$ una sucesi\'on de variables aleatorias independientes e id\'enticamente distribuidas seg\'un una distribuci\'on $B(1,p)$, entonces $\left\{X_n\right\}_{n\in\mathbb{N}}\in LN $.
\end{teorema}
\begin{teorema}[Levy-Lindeberg]
Sea $ \left\{X_n\right\}_{n\in\mathbb{N}}$ una sucesi\'on de variables aleatorias independientes e id\'enticamente distribuidas con media y varianza finitas, entonces $\left\{X_n\right\}_{n\in\mathbb{N}}\in LN $.
\end{teorema}
