\selectlanguage{english}%

\chapter{Muestreo con probabilidades iguales. Estimadores lineales. Varianzas
de los estimadores y sus estimaciones. Comparaci\'on entre el muestreo
con y sin reposici\'on. Consideraciones sobre el tama\~no de la muestra.}


\section{Muestreo con probabilidades iguales.}

Cuando la probabilidad que tiene cualquier unidad de la poblaci\'on
de ser elegida para la muestra es la misma para todas la unidades
decimos que estamos ante un m\'etodo de muestreo con probabilidades
iguales. Habr\'a que distinguir entre muestreo sin reposici\'on y con
reposici\'on, y entre los casos en que el orden de colocaci\'on de los
elementos intervenga o no sea as\'i.


\subsection{Muestreo sin reposici\'on.}


\subsubsection{No interviene el orden.}

En este caso cada unidad de muestreo puede aparecer como m\'aximo una
vez en la muestra. Por tanto, en la primera extracci\'on cada unidad
tendr\'a una probabilidad $P\left(u_{i}\right)=\dfrac{1}{N}$, en la
segunda la probabilidad de elegir una de las restantes ser\'a $P\left(u_{j}\right)=\dfrac{1}{N-1}$,
y as\'i sucesivamente.

En este caso, el n\'umero de muestras de tama\~no $n$ de un espacio muestral
con $N$ unidades ser\'a el de las combinaciones sin repetici\'on de $N$
elementos tomados de $n$ en $n$, es decir $C_{N,n}=\dbinom{N}{n}=\dfrac{N!}{n!\left(N-n\right)!}$,
y la probabilidad de una muestra cualquiera ser\'a $P\left(u_{1},u_{2},\ldots,u_{n}\right)=\dfrac{1}{\dbinom{N}{n}}$.
Calculando a partir de las probabilidades de elegir una unidad, tenemos:
\[
\begin{array}{c}
P\left(u_{1},u_{2},\ldots,u_{n}\right)=n!P\left(\left\{ u_{1},u_{2},\ldots,u_{n}\right\} \right)=n!P\left(u_{1}\right)P\left(u_{2}/u_{1}\right)P\left(u_{3}/u_{1}u_{2}\right)\cdots P\left(u_{n}/u_{1}u_{2}\ldots u_{n-1}\right)=\\
=n!\dfrac{1}{N}\dfrac{1}{N-1}\dfrac{1}{N-2}\cdots\dfrac{1}{N-\left(n-1\right)}=\dfrac{1}{\dfrac{N!}{n!\left(N-n\right)!}}
\end{array}
\]


Calculemos ahora la probabilidad que tiene una unidad $u_{i}$ de
pertenecer a la muestra, ya que es necesario para desarrollar toda
la teor\'ia del muestreo. La probabilidad de una unidad de pertenecer
a la muestra ser\'a la suma de la probabilidad que tiene de salir en
la priemer extracci\'on, m\'as la probabilidad de salir en la segunda
extracci\'on no habiendo salido en la primera, m\'as la probabilidad de
salir en la tercera extracci\'on no habiendo salido en la primera ni
en la segunda... y s\'i hasta la extracci\'on $n$-\'esima. As\'i podemos
escribir:

\[
P\left(u_{i}=u_{1}\right)=\dfrac{1}{N}
\]


\[
P\left(u_{i}=u_{2}\cap u_{i}\neq u_{1}\right)=P\left(u_{i}=u_{2}/u_{i}\neq u_{1}\right)P\left(u_{i}\neq u_{1}\right)=\dfrac{1}{N-1}\dfrac{N-1}{N}=\dfrac{1}{N}
\]


\[
P\left(u_{i}=u_{3}\cap u_{i}\neq u_{2}\cap u_{i}\neq u_{1}\right)=P\left(u_{i}=u_{3}/u\neq u_{2}\cap u_{i}\neq u_{1}\right)P\left(u_{i}\neq u_{2}/u_{i}\neq u_{1}\right)P\left(u_{i}\neq u_{1}\right)=\dfrac{1}{N-2}\dfrac{N-2}{N-1}\dfrac{N-1}{N}=\dfrac{1}{N}
\]


\[
\vdots
\]


\[
P\left(u_{i}=u_{n}\cap u_{i}\neq u_{n-1}\cap\cdots\cap u_{i}\neq u_{1}\right)=P\left(u_{i}=u_{n}/u_{i}\neq u_{n-1}\cap\cdots\cap u_{i}\neq u_{1}\right)\cdots P\left(u_{i}\neq u_{2}/u_{i}\neq u_{1}\right)P\left(u_{i}\neq u_{1}\right)=
\]


\[
=\dfrac{1}{N-\left(n-1\right)}\dfrac{N-\left(n-1\right)}{N-\left(n-2\right)}\cdots\dfrac{N-2}{N-1}\dfrac{N-1}{N}=\dfrac{1}{N}
\]


y por tanto, $\pi_{i}=\dfrac{n}{N}$.

Tambi\'en ser\'a necesario calcular la probabilidad de dos unidades distintas
de pertenecer simult\'aneamente a la muestra. El n\'umero de muestras
posibles ser\'a $\dbinom{N}{n}$, y el n\'umero de muestras en las que
est\'en presentes las dos unidades que nos interesan ser\'a $\dbinom{N-2}{n-2}$.
APlicando la visi\'on frecuentista del c\'alculo de probabilidades, la
probabilidad ser\'a el n\'umero de casos favorables entre el n\'umero de
casos posibles: 
\[
\pi_{ij}=\dfrac{\dbinom{N-2}{n-2}}{\dbinom{N}{n}}=\dfrac{\dfrac{\left(N-2\right)!}{\left(n-2\right)!\left(N-n\right)!}}{\dfrac{N!}{n!\left(N-n\right)!}}=\dfrac{\left(N-2\right)!n!}{N!\left(n-2\right)!}=\dfrac{n\left(n-1\right)}{N\left(N-1\right)}
\]


Si definimos la variable aleatoria $e_{i}$ como el n\'umero de veces
que la unidad $u_{i}$ puede aparecer en la muestra, esta variable
puede tomar valores entre $0$ y $1$ para cada unidad, y sigue una
distribuci\'on de Bernouilli de par\'ametro $p=\dfrac{n}{N}$. Por tanto,
$E\left[e_{i}\right]=\dfrac{n}{N}$, $Var\left(e_{i}\right)=\dfrac{n\left(N-n\right)}{N^{2}}$,
$Cov\left(e_{i},e_{j}\right)=-\dfrac{n\left(N-n\right)}{N^{2}\left(N-1\right)}$.


\subsubsection{Interviene el orden.}

En este caso, el n\'umero de muestras de tama\~no $n$ de un espacio muestral
con $N$ unidades ser\'a el de las variaciones sin repetici\'on de $N$
elementos tomados de $n$ en $n$, es decir $V_{N,n}=\dbinom{N}{n}n!=\dfrac{N!}{\left(N-n\right)!}$,
y la probabilidad de una muestra cualquiera ser\'a $P\left(u_{1},u_{2},\ldots,u_{n}\right)=\dfrac{1}{\dbinom{N}{n}n!}$.
Las probabilidades de una unidad de pertenecer a la muestra son las
mismas.


\subsection{Muestreo con reposici\'on.}


\subsubsection{No interviene el orden.}

En este caso, el n\'umero de muestras de tama\~no $n$ de un espacio muestral
con $N$ unidades ser\'a el de las combinaciones con repetici\'on de $N$
elementos tomados de $n$ en $n$, es decir $CR_{N,n}=\dbinom{N+n-1}{n}=\dfrac{\left(N+n-1\right)!}{n!\left(N-1\right)!}$,
y la probabilidad de todas las muestras no ser\'a la misma, ya que este
m\'etodo de selecci\'on no produce muestras equiprobables.


\subsubsection{Interviene el orden.}

En este caso, el n\'umero de muestras de tama\~no $n$ de un espacio muestral
con $N$ unidades ser\'a el de las variaciones con repetici\'on de $N$
elementos tomados de $n$ en $n$, es decir $VR_{N,n}=N^{n}$, y la
probabilidad de una muestra cualquiera ser\'a $P\left(u_{1},u_{2},\ldots,u_{n}\right)=\dfrac{1}{N^{n}}$.

Si definimos la variable aleatoria $e_{i}$ como el n\'umero de veces
que la unidad $u_{i}$ puede aparecer en la muestra, esta variable
puede tomar valores entre $0$ y $n$ para cada unidad, y sigue una
distribuci\'on binomial de par\'ametros $n$ y $p=\dfrac{1}{N}$. Por
tanto, $E\left[e_{i}\right]=\dfrac{n}{N}$, $Var\left(e_{i}\right)=\dfrac{n\left(N-1\right)}{N^{2}}$,
$Cov\left(e_{i},e_{j}\right)=-nP_{i}P_{j}=-\dfrac{n}{N^{2}}$.


\section{Estimadores lineales insesgados.}

Supongamos que tenemos una poblaci\'on de tama\~no $N$, para la que hemos
definido una caracter\'istica $X_{i}$ que toma el valor $X_{i}$ en
cada unidad $u_{i}$. Supongamos que tenemos un par\'ametro poblacional,
funci\'on de los $N$ valores de las $X_{i}$, que es el que queremos
estimar. En general este par\'ametro se puede expresar como una suma
de elementos $Y_{i}$, que son funci\'on de los valores que la caracter\'istica
$X_{i}$ presenta, $\theta=\sum_{i=1}^{N}Y_{i}$ donde $Y_{i}=Y\left(X_{i}\right)$.
Por ejemplo:
\begin{description}
\item [{Total~poblacional:}] $X=\theta\left(X_{1},\ldots,X_{N}\right)=\sum_{i=1}^{N}X_{i}$,
donde $Y_{i}=X_{i}$.
\item [{Media~poblacional:}] $X=\theta\left(X_{1},\ldots,X_{N}\right)=\dfrac{1}{N}\sum_{i=1}^{N}X_{i}$,
donde $Y_{i}=\dfrac{X_{i}}{N}$.
\item [{Total~de~clase:}] $A=\theta\left(A_{1},\ldots,A_{N}\right)=\sum_{i=1}^{N}A_{i}$,
donde $Y_{i}=A_{i}$. Donde $A_{i}$ se usa para caracter\'isticas cualitativas
dicot\'omicas: $A_{i}=1$ si $u_{1}$ presenta la caracter\'istica, y
$A_{i}=0$ si $u_{1}$ no presenta la caracter\'istica.
\item [{Proporci\'on~de~clase:}] $A=\theta\left(A_{1},\ldots,A_{N}\right)=\dfrac{1}{N}\sum_{i=1}^{N}A_{i}$,
donde $Y_{i}=\dfrac{A_{i}}{N}$.
\end{description}
En general las mejores propiedades suelen presentarlas los estimadores
lineales, de la forma $\hat{\theta}=\sum_{i=1}^{n}w_{i}Y_{i}$.
\begin{itemize}
\item Todas las mediciones de la variable de estudio que aparecen en la
muestra intervienen en el estimador.
\item La importancia de la aportaci\'on al estimador de cada unidad muestral
puede controlarse mediante su coeficiente.
\item Cuando $w_{i}=1$ todas las unidades muestrales intervienen con la
misma importancia en el estimador.
\item Cuando las unidades de la muestra son compuestas, el valor de $w_{i}$
puede regular la importancia de cada unidad compuesta asoci\'andola
con su tama\~no o con el n\'umero de unidades elementales que contiene.
\item Los coeficientes pueden depender del tama\~no de las unidades muestrales,
de su orden en la muestra o de las probabilidades que tienen de pertenecer
a la muestra.
\item Las funciones lineales son las m\'as sencillas de manejar matem\'aticamente.
\end{itemize}

\subsection{Muestreo sin reposici\'on.}

Queremos que el estimador sea insesgado, es decir, que su esperanza
sea igual al par\'ametro que queremos estimar. Veamos cual ha de ser
el valor de los coeficientes para que esto ocurra.

\[
E\left(\hat{\theta}\right)=E\left(\sum_{i=1}^{n}w_{i}Y_{i}\right)=E\left(\sum_{i=1}^{N}w_{i}Y_{i}e_{i}\right)=\sum_{i=1}^{N}w_{i}Y_{i}E\left(e_{i}\right)
\]


Ya que $e_{i}=1$ si $u_{i}$ pertenece a la muestra y$e_{i}=0$ si
$u_{i}$ no pertenece a la muestra. Como $\theta=\sum_{i=1}^{N}Y_{i}$,
entonces se tiene que cumplir que $w_{i}E\left(e_{i}\right)=1$ y
por tanto, $w_{i}=\dfrac{1}{E\left(e_{i}\right)}$. Para muestreo
sin reposici\'on y probabilidades iguales $E\left(e_{i}\right)=\dfrac{n}{N}$
y por tanto$w_{i}=\dfrac{N}{n}$. Con esto podemos construir los estimadores:
\begin{description}
\item [{Total~poblacional:}] Como $Y_{i}=X_{i}$, $\hat{X}=\sum_{i=1}^{n}w_{i}Y_{i}=\sum_{i=1}^{n}\dfrac{N}{n}X_{i}=\dfrac{N}{n}\sum_{i=1}^{n}X_{i}=N\bar{x}$.
\item [{Media~poblacional:}] Como $Y_{i}=\dfrac{X_{i}}{N}$, $\hat{\bar{X}}=\sum_{i=1}^{n}w_{i}Y_{i}=\sum_{i=1}^{n}\dfrac{N}{n}\dfrac{X_{i}}{N}=\dfrac{1}{n}\sum_{i=1}^{n}X_{i}=\bar{x}$.
\item [{Total~de~clase:}] Como $Y_{i}=A_{i}$, $\hat{A}=\sum_{i=1}^{n}w_{i}Y_{i}=\sum_{i=1}^{n}\dfrac{N}{n}A_{i}=\dfrac{N}{n}\sum_{i=1}^{n}A_{i}$.
\item [{Proporci\'on~de~clase:}] Como $Y_{i}=\dfrac{A_{i}}{N}$, $\hat{P}=\sum_{i=1}^{n}w_{i}Y_{i}=\sum_{i=1}^{n}\dfrac{N}{n}\dfrac{A_{i}}{N}=\dfrac{1}{n}\sum_{i=1}^{n}A_{i}$.
\end{description}

\subsection{Muestreo con reposici\'on.}

Queremos que el estimador sea insesgado, es decir, que su esperanza
sea igual al par\'ametro que queremos estimar. Veamos cual ha de ser
el valor de los coeficientes para que esto ocurra.

\[
E\left(\hat{\theta}\right)=E\left(\sum_{i=1}^{n}w_{i}Y_{i}\right)=E\left(\sum_{i=1}^{N}w_{i}Y_{i}e_{i}\right)=\sum_{i=1}^{N}w_{i}Y_{i}E\left(e_{i}\right)
\]


Ya que $e_{i}$ es igual al n\'umero de veces que $u_{i}$ aparece en
la muestra, por tanto $Y_{i}e_{i}$ es lo mismo que sumar $Y_{i}$
tantas veces como la unidad$u_{i}$ aparece en la muestra. Como $\theta=\sum_{i=1}^{N}Y_{i}$,
entonces se tiene que cumplir que $w_{i}E\left(e_{i}\right)=1$ y
por tanto, $w_{i}=\dfrac{1}{E\left(e_{i}\right)}$. Para muestreo
con reposici\'on y probabilidades iguales $E\left(e_{i}\right)=\dfrac{n}{N}$
y por tanto$w_{i}=\dfrac{N}{n}$. Con esto podemos construir los estimadores:
\begin{description}
\item [{Total~poblacional:}] Como $Y_{i}=X_{i}$, $\hat{X}=\sum_{i=1}^{n}w_{i}Y_{i}=\sum_{i=1}^{n}\dfrac{N}{n}X_{i}=\dfrac{N}{n}\sum_{i=1}^{n}X_{i}=N\bar{x}$.
\item [{Media~poblacional:}] Como $Y_{i}=\dfrac{X_{i}}{N}$, $\hat{\bar{X}}=\sum_{i=1}^{n}w_{i}Y_{i}=\sum_{i=1}^{n}\dfrac{N}{n}\dfrac{X_{i}}{N}=\dfrac{1}{n}\sum_{i=1}^{n}X_{i}=\bar{x}$.
\item [{Total~de~clase:}] Como $Y_{i}=A_{i}$, $\hat{A}=\sum_{i=1}^{n}w_{i}Y_{i}=\sum_{i=1}^{n}\dfrac{N}{n}A_{i}=\dfrac{N}{n}\sum_{i=1}^{n}A_{i}$.
\item [{Proporci\'on~de~clase:}] Como $Y_{i}=\dfrac{A_{i}}{N}$, $\hat{P}=\sum_{i=1}^{n}w_{i}Y_{i}=\sum_{i=1}^{n}\dfrac{N}{n}\dfrac{A_{i}}{N}=\dfrac{1}{n}\sum_{i=1}^{n}A_{i}$.
\end{description}
Podemos observar que coinciden con los estimadores de estoa par\'ametros
si el muestreo se realiza sin reposici\'on.


\section{Varianzas de los estimadores y sus estimaciones.}

Es importante conocer la expresi\'on de la varianza de estos estimadores,
as\'i como una forma de estimar la misma, para poder evaluar la calidad
de los mismos.


\subsection{Varianza del estimador en muestreo sin reposici\'on.}

\[
\begin{array}{c}
Var\left(\hat{\theta}\right)=Var\left(\dfrac{N}{n}\sum_{i=1}^{n}Y_{i}\right)=Var\left(\dfrac{N}{n}\sum_{i=1}^{N}Y_{i}e_{i}\right)=\dfrac{N^{2}}{n^{2}}\left[\sum_{i=1}^{N}Y_{i}^{2}V\left(e_{i}\right)+\sum_{i=1}^{N}\sum_{j\neq i}^{N}Y_{i}Y_{j}Cov\left(e_{i},e_{j}\right)\right]=\\
=\dfrac{N^{2}}{n^{2}}\left[\dfrac{n\left(N-n\right)}{N^{2}}\sum_{i=1}^{N}Y_{i}^{2}-\dfrac{n\left(N-n\right)}{N^{2}\left(N-1\right)}\sum_{i=1}^{N}\sum_{j\neq i}^{N}Y_{i}Y_{j}\right]=\dfrac{N-n}{n}\left[\sum_{i=1}^{N}Y_{i}^{2}-\dfrac{1}{\left(N-1\right)}\sum_{i=1}^{N}\sum_{j\neq i}^{N}Y_{i}Y_{j}\right]=\\
=\dfrac{N-n}{n}\left[\sum_{i=1}^{N}Y_{i}^{2}-\dfrac{1}{\left(N-1\right)}\left(\sum_{i=1}^{N}\sum_{j=1}^{N}Y_{i}Y_{j}-\sum_{i=1}^{N}Y_{i}^{2}\right)\right]=\dfrac{N-n}{n\left(N-1\right)}\left[\left(N-1\right)\sum_{i=1}^{N}Y_{i}^{2}+\left(\sum_{i=1}^{N}Y_{i}^{2}-Y^{2}\right)\right]=\\
=\dfrac{N-n}{n\left(N-1\right)}\left[N\sum_{i=1}^{N}Y_{i}^{2}-Y^{2}\right]=\dfrac{N^{2}\left(N-n\right)}{n\left(N-1\right)}\left[\sum_{i=1}^{N}\dfrac{Y_{i}^{2}}{N}-\bar{Y}^{2}\right]=\dfrac{N^{2}\left(N-n\right)}{n\left(N-1\right)}\sigma_{Y}^{2}=\dfrac{N^{2}\left(N-n\right)}{N}\dfrac{S_{Y}^{2}}{n}\\
\\
\end{array}
\]


Y se obtienen las siguientes varianzas:
\begin{description}
\item [{Total~poblacional:}] Como $Y_{i}=X_{i}$, $Var\left(\hat{X}\right)=N^{2}\left(1-f\right)\dfrac{S_{X}^{2}}{n}$,
con $f=\dfrac{n}{N}$ y $S_{X}^{2}=\dfrac{1}{N-1}\sum_{i=1}^{N}\left(X_{i}-\bar{X}\right)^{2}$,
cuasivarianza poblacional.
\item [{Media~poblacional:}] Como $Y_{i}=\dfrac{X_{i}}{N}$, $Var\left(\hat{\bar{X}}\right)=\left(1-f\right)\dfrac{S_{X}^{2}}{n}$,
con $f=\dfrac{n}{N}$ y $S_{X}^{2}=\dfrac{1}{N-1}\sum_{i=1}^{N}\left(X_{i}-\bar{X}\right)^{2}$,
cuasivarianza poblacional.
\item [{Total~de~clase:}] Como $Y_{i}=A_{i}$, $Var\left(\hat{A}\right)=\dfrac{N^{3}}{N-1}\left(1-f\right)\dfrac{P\left(1-P\right)}{n}$.
\item [{Proporci\'on~de~clase:}] Como $Y_{i}=\dfrac{A_{i}}{N}$, $Var\left(\hat{P}\right)=\dfrac{N}{N-1}\left(1-f\right)\dfrac{P\left(1-P\right)}{n}$.
\end{description}

\subsection{Estimaci\'on de las varianzas.}

Los estimadores insesgados para estas varainzas son los siguientes:
\begin{description}
\item [{Total~poblacional:}] Como $Y_{i}=X_{i}$, $\hat{Var}\left(\hat{X}\right)=N^{2}\left(1-f\right)\dfrac{s_{X}^{2}}{n}$,
con $f=\dfrac{n}{N}$ y $s_{X}^{2}=\dfrac{1}{n-1}\sum_{i=1}^{n}\left(X_{i}-\bar{x}\right)^{2}$,
cuasivarianza muestral.
\item [{Media~poblacional:}] Como $Y_{i}=\dfrac{X_{i}}{N}$, $\hat{Var}\left(\hat{\bar{X}}\right)=\left(1-f\right)\dfrac{s_{X}^{2}}{n}$,
con $f=\dfrac{n}{N}$ y $s_{X}^{2}=\dfrac{1}{n-1}\sum_{i=1}^{n}\left(X_{i}-\bar{x}\right)^{2}$,
cuasivarianza muestral.
\item [{Total~de~clase:}] Como $Y_{i}=A_{i}$, $\hat{Var}\left(\hat{A}\right)=\dfrac{N^{3}}{N-1}\left(1-f\right)\dfrac{p\left(1-p\right)}{n}$,
con $f=\dfrac{n}{N}$ y $p=\dfrac{1}{n}\sum_{i=1}^{n}A_{i}$, proporci\'on
muestral.
\item [{Proporci\'on~de~clase:}] Como $Y_{i}=\dfrac{A_{i}}{N}$, $\hat{Var}\left(\hat{P}\right)=\dfrac{N}{N-1}\left(1-f\right)\dfrac{p\left(1-p\right)}{n}$,
con $f=\dfrac{n}{N}$ y $p=\dfrac{1}{n}\sum_{i=1}^{n}A_{i}$, proporci\'on
muestral.
\end{description}

\subsection{Varianza del estimador en muestreo con reposici\'on.}

\[
\begin{array}{c}
Var\left(\hat{\theta}\right)=Var\left(\dfrac{N}{n}\sum_{i=1}^{n}Y_{i}\right)=Var\left(\dfrac{N}{n}\sum_{i=1}^{N}Y_{i}e_{i}\right)=\dfrac{N^{2}}{n^{2}}\left[\sum_{i=1}^{N}Y_{i}^{2}V\left(e_{i}\right)+2\sum_{i=1}^{N}\sum_{j>i}^{N}Y_{i}Y_{j}Cov\left(e_{i},e_{j}\right)\right]=\\
=\dfrac{N^{2}}{n^{2}}\left[\dfrac{n\left(N-1\right)}{N^{2}}\sum_{i=1}^{N}Y_{i}^{2}-2\dfrac{n}{N^{2}}\sum_{i=1}^{N}\sum_{j>i}^{N}Y_{i}Y_{j}\right]=\dfrac{1}{n}\left[\left(N-1\right)\sum_{i=1}^{N}Y_{i}^{2}-2\sum_{i=1}^{N}\sum_{j>i}^{N}Y_{i}Y_{j}\right]\cdots\\
\left(\sum_{i=1}^{N}Y_{i}\right)^{2}=\sum_{i=1}^{N}Y_{i}^{2}+2\sum_{i=1}^{N}\sum_{j>i}^{N}Y_{i}Y_{j}\\
-2\sum_{i=1}^{N}\sum_{j>i}^{N}Y_{i}Y_{j}=\sum_{i=1}^{N}Y_{i}^{2}-\left(\sum_{i=1}^{N}Y_{i}\right)^{2}=\sum_{i=1}^{N}Y_{i}^{2}-\theta^{2}\\
\dfrac{1}{n}\left[\left(N-1\right)\sum_{i=1}^{N}Y_{i}^{2}-2\sum_{i=1}^{N}\sum_{j>i}^{N}Y_{i}Y_{j}\right]=\dfrac{1}{n}\left[\left(N-1\right)\sum_{i=1}^{N}Y_{i}^{2}+\sum_{i=1}^{N}Y_{i}^{2}-\theta^{2}\right]=\dfrac{1}{n}\left[N\sum_{i=1}^{N}Y_{i}^{2}-\theta^{2}\right]\\
Var\left(\hat{\theta}\right)=\dfrac{1}{n}\left[N\sum_{i=1}^{N}Y_{i}^{2}-\theta^{2}\right]
\end{array}
\]


Y se obtienen las siguientes varianzas:
\begin{description}
\item [{Total~poblacional:}] Como $Y_{i}=X_{i}$, $Var\left(\hat{X}\right)=\dfrac{1}{n}\left[N\sum_{i=1}^{N}X_{i}^{2}-\left(\sum_{i=1}^{N}X_{i}\right)^{2}\right]=\dfrac{N^{2}}{n}\left[\bar{X^{2}}-\bar{X}^{2}\right]=\dfrac{N^{2}}{n}\sigma_{X}^{2}$,
con $\sigma_{X}^{2}=\dfrac{1}{N}\sum_{i=1}^{N}\left(X_{i}-\bar{X}\right)^{2}$,
varianza poblacional.
\item [{Media~poblacional:}] Como $Y_{i}=\dfrac{X_{i}}{N}$, $Var\left(\hat{\bar{X}}\right)=\dfrac{1}{n}\left[N\sum_{i=1}^{N}\dfrac{X_{i}^{2}}{N^{2}}-\left(\sum_{i=1}^{N}\dfrac{X_{i}}{N}\right)^{2}\right]=\dfrac{1}{n}\left[\bar{X^{2}}-\bar{X}^{2}\right]=\dfrac{1}{n}\sigma_{X}^{2}$,
con $\sigma_{X}^{2}=\dfrac{1}{N}\sum_{i=1}^{N}\left(X_{i}-\bar{X}\right)^{2}$,
varianza poblacional.
\item [{Total~de~clase:}] Como $Y_{i}=A_{i}$, $Var\left(\hat{A}\right)=\dfrac{1}{n}\left[N\sum_{i=1}^{N}A_{i}^{2}-\left(\sum_{i=1}^{N}A_{i}\right)^{2}\right]=\dfrac{N^{2}}{n}\left[P-P^{2}\right]=\dfrac{N^{2}}{n}P(1-P)$.
\item [{Proporci\'on~de~clase:}] Como $Y_{i}=\dfrac{A_{i}}{N}$, $Var\left(\hat{P}\right)=\dfrac{1}{n}\left[N\sum_{i=1}^{N}\dfrac{A_{i}^{2}}{N^{2}}-\left(\sum_{i=1}^{N}\dfrac{A_{i}}{N}\right)^{2}\right]=\dfrac{1}{n}\left[P-P^{2}\right]=\dfrac{1}{n}P(1-P)$.
\end{description}

\subsection{Estimaci\'on de las varianzas.}

Los estimadores insesgados para estas varainzas son los siguientes:
\begin{description}
\item [{Total~poblacional:}] Como $Y_{i}=X_{i}$, $\hat{Var}\left(\hat{X}\right)=N^{2}\dfrac{s_{X}^{2}}{n}$,
con $s_{X}^{2}=\dfrac{1}{n-1}\sum_{i=1}^{n}\left(X_{i}-\bar{x}\right)^{2}$,
cuasivarianza muestral.
\item [{Media~poblacional:}] Como $Y_{i}=\dfrac{X_{i}}{N}$, $\hat{Var}\left(\hat{\bar{X}}\right)=\dfrac{s_{X}^{2}}{n}$,
con $s_{X}^{2}=\dfrac{1}{n-1}\sum_{i=1}^{n}\left(X_{i}-\bar{x}\right)^{2}$,
cuasivarianza muestral.
\item [{Total~de~clase:}] Como $Y_{i}=A_{i}$, $\hat{Var}\left(\hat{A}\right)=N^{2}\dfrac{p\left(1-p\right)}{n}$,
con $p=\dfrac{1}{n}\sum_{i=1}^{n}A_{i}$, proporci\'on muestral.
\item [{Proporci\'on~de~clase:}] Como $Y_{i}=\dfrac{A_{i}}{N}$, $\hat{Var}\left(\hat{P}\right)=\dfrac{p\left(1-p\right)}{n}$,
con $p=\dfrac{1}{n}\sum_{i=1}^{n}A_{i}$, proporci\'on muestral.
\end{description}

\section{Comparaci\'on entre el muestreo con y sin reposici\'on.}

Es interesante comparar la eficiencia de los estimadores si se realiza
el muestreosin reposici\'on respecto a si se realiza con reposici\'on.
La forma de comprobar cual es m\'as eficiente es comparar sus varianzas.
El estimador que a igual tama\~no de la muestra tenga menor varianza
ser\'a el m\'as eficiente.

Antes de empezar, es conveniente tener en cuenta que $S_{X}^{2}=\dfrac{N}{N-1}\sigma_{X}^{2}$.
Veamos la varianza del estimador del total:

\[
\left.\begin{array}{l}
Var\left(\hat{X}_{SR}\right)=N^{2}\left(1-f\right)\dfrac{S_{X}^{2}}{n}=N^{2}\left(\dfrac{N-n}{N}\right)\dfrac{N}{N-1}\dfrac{\sigma_{X}^{2}}{n}=N^{2}\dfrac{N-n}{N-1}\dfrac{\sigma_{X}^{2}}{n}\\
Var\left(\hat{X}_{CR}\right)=N^{2}\dfrac{\sigma_{X}^{2}}{n}
\end{array}\right\} \Rightarrow\dfrac{Var\left(\hat{X}_{SR}\right)}{Var\left(\hat{X}_{CR}\right)}=\dfrac{N-n}{N-1}<1\Rightarrow Var\left(\hat{X}_{SR}\right)<Var\left(\hat{X}_{CR}\right)
\]
\[
\left.\begin{array}{l}
Var\left(\hat{\bar{X}}_{SR}\right)=\left(1-f\right)\dfrac{S_{X}^{2}}{n}=\left(\dfrac{N-n}{N}\right)\dfrac{N}{N-1}\dfrac{\sigma_{X}^{2}}{n}=\dfrac{N-n}{N-1}\dfrac{\sigma_{X}^{2}}{n}\\
Var\left(\hat{\bar{X}}_{CR}\right)=\dfrac{\sigma_{X}^{2}}{n}
\end{array}\right\} \Rightarrow\dfrac{Var\left(\hat{\bar{X}}_{SR}\right)}{Var\left(\hat{\bar{X}}_{CR}\right)}=\dfrac{N-n}{N-1}<1\Rightarrow Var\left(\hat{\bar{X}}_{SR}\right)<Var\left(\hat{\bar{X}}_{CR}\right)
\]
\[
\left.\begin{array}{l}
Var\left(\hat{A}_{SR}\right)=\dfrac{N^{3}}{N-1}\left(1-f\right)\dfrac{P\left(1-P\right)}{n}=\dfrac{N^{2}}{N-1}\left(N-n\right)\dfrac{P\left(1-P\right)}{n}\\
Var\left(\hat{A}_{CR}\right)=\dfrac{N^{2}}{n}P(1-P)
\end{array}\right\} \Rightarrow\dfrac{Var\left(\hat{A}_{SR}\right)}{Var\left(\hat{A}_{CR}\right)}=\dfrac{N-n}{N-1}<1\Rightarrow Var\left(\hat{A}_{SR}\right)<Var\left(\hat{A}_{CR}\right)
\]
\[
\left.\begin{array}{l}
Var\left(\hat{P}_{SR}\right)=\dfrac{N}{N-1}\left(1-f\right)\dfrac{P\left(1-P\right)}{n}=\dfrac{1}{N-1}\left(N-n\right)\dfrac{P\left(1-P\right)}{n}\\
Var\left(\hat{P}_{CR}\right)=\dfrac{1}{n}P(1-P)
\end{array}\right\} \Rightarrow\dfrac{Var\left(\hat{P}_{SR}\right)}{Var\left(\hat{P}_{CR}\right)}=\dfrac{N-n}{N-1}<1\Rightarrow Var\left(\hat{P}_{SR}\right)<Var\left(\hat{P}_{CR}\right)
\]


Por tanto, el muestreo sin reposici\'on es m\'as eficiente que con reposici\'on.
El grado de eficiencia que se consigue depende de la relaci\'on entre
$N$ y $n$. Cuanto menor sea $n$ respecto al tama\~no de la poblaci\'on
menor ser\'a la ganancia en precisi\'on del muestreo sin reemplazamiento.
Si $N-1\approx N$, entonces $V_{SR}=\left(1-f\right)V_{CR}$, es
decir, cuanto menor sea la fracci\'on de muestreo mayor ser\'a la ganancia
en precisi\'on. A $\left(1-f\right)$ se le llama factor de correcci\'on
para poblaciones finitas.


\section{Consideraciones sobre el tama\~no de la muestra.}

La importancia del tama\~no muestral, radica en la precisi\'on del estimador
y la representatividad de muestra. es decir, cuanto mayor sea el tama\~no
muestral m\'as preciso ser\'a el estimador.

Sin embargo, este razonamiento nos llevar\'ia a proponer muestras de
tama\~no $N$, es decir, que contengan a toda la poblaci\'on. Normalmente
esto no es posible por problemas de coste, disponibilidad... Es por
esto que es necesario definir un criterio respecto a lo que se espera
del estimador, y a partir de ah\'i tomar la muestra m\'as peque\~na que
nos permita cumplir con ese criterio. Ese criterio normalmente se
expresa en funci\'on del error de muestreo deseado.

Normalmente el tama\~no de la muestra depender\'a de par\'ametros poblacionales
desconocidos que habr\'a que estimar.


\subsection{Tama\~no de la muestra para un error de muestreo dado.}

Sea $e=\sigma_{\hat{\theta}}$ el error de muestreo m\'aximo que estamos
dispuestos a admitir. Veamos cual es el tama\~no de la muestra a seleccionar
para cometer ese error:


\subsubsection{Muestreo sin reemplazamiento.}
\begin{description}
\item [{Total~poblacional:}] $e^{2}=N^{2}\left(1-\dfrac{n}{N}\right)\dfrac{S_{X}^{2}}{n}=\dfrac{N^{2}S_{X}^{2}}{n}-\dfrac{N^{2}S_{X}^{2}}{N}$
y por tanto, $n=\dfrac{N^{2}S_{X}^{2}}{e^{2}+NS_{X}^{2}}$.
\item [{Media~poblacional:}] $e^{2}=\left(1-\dfrac{n}{N}\right)\dfrac{S_{X}^{2}}{n}=\dfrac{S_{X}^{2}}{n}-\dfrac{S_{X}^{2}}{N}$
y por tanto, $n=\dfrac{NS_{X}^{2}}{Ne^{2}+S_{X}^{2}}$.
\item [{Total~de~clase:}] $e^{2}=\dfrac{N^{3}}{N-1}\left(1-\dfrac{n}{N}\right)\dfrac{P\left(1-P\right)}{n}=N^{2}\left(\dfrac{N}{N-1}\dfrac{P\left(1-P\right)}{n}-\dfrac{N}{N-1}\dfrac{P\left(1-P\right)}{N}\right)$
y por tanto, $n=\dfrac{\dfrac{N}{N-1}P\left(P-1\right)}{\dfrac{e^{2}}{N^{2}}+\dfrac{N}{N-1}\dfrac{P\left(1-P\right)}{N}}=\dfrac{N^{3}P\left(P-1\right)}{e^{2}\left(N-1\right)+N^{2}P\left(P-1\right)}$
\item [{Proporci\'on~de~clase:}] $e^{2}=\dfrac{N}{N-1}\left(1-\dfrac{n}{N}\right)\dfrac{P\left(1-P\right)}{n}=\dfrac{N}{N-1}\dfrac{P\left(1-P\right)}{n}-\dfrac{N}{N-1}\dfrac{P\left(1-P\right)}{N}$,
$n=\dfrac{NP(1-P)}{e^{2}\left(N-1\right)+P(1-P)}$.
\end{description}

\subsubsection{Muestreo con reemplazamiento.}
\begin{description}
\item [{Total~poblacional:}] $e^{2}=N^{2}\dfrac{\sigma_{X}^{2}}{n}$ y
por tanto, $n=\dfrac{N^{2}\sigma_{X}^{2}}{e^{2}}$.
\item [{Media~poblacional:}] $e^{2}=\dfrac{\sigma_{X}^{2}}{n}$ y por
tanto, $n=\dfrac{\sigma_{X}^{2}}{e^{2}}$.
\item [{Total~de~clase:}] $e^{2}=\dfrac{N^{2}}{n}P(1-P)$ y por tanto,
$n=\dfrac{N^{2}P\left(P-1\right)}{e^{2}}$
\item [{Proporci\'on~de~clase:}] $e^{2}=\dfrac{1}{n}P(1-P)$, $n=\dfrac{P(1-P)}{e^{2}}$.
\end{description}

\subsection{Tama\~no de la muestra para un error de muestreo y un coeficiente de
confianza dados.}

Muchas veces no queremos fijar tanto la varianza de nuestro estimador
cmo establecer la m\'axima aplitud del intervalo de confianza que a
un nivel de confianza dado que generaremos con el mismo, es decir
queremos fijar un $e_{\alpha}$ y un $\alpha$ tales que: 
\[
P\left(-e_{\alpha}\leq\hat{\theta}-\theta\leq e_{\alpha}\right)=1-\alpha
\]


Por tanto:
\[
P\left(\dfrac{-e_{\alpha}}{\sigma\left(\hat{\theta}\right)}\leq\dfrac{\hat{\theta}-\theta}{\sigma\left(\hat{\theta}\right)}\leq\dfrac{e_{\alpha}}{\sigma\left(\hat{\theta}\right)}\right)=1-\alpha
\]


Y suponiendo una distribuci\'on normal, $\lambda_{\alpha}=\dfrac{e_{\alpha}}{\sigma\left(\hat{\theta}\right)}$,
tal que $P\left(x\leq\lambda_{\alpha}\right)=1-\dfrac{\alpha}{2}$,
de las tablas de la distribuci\'on normal est\'andar. Por tanto, necesitaremos
que $\sigma\left(\hat{\theta}\right)=\dfrac{e_{\alpha}}{\lambda_{\alpha}}$,
y con este valor podemos aplicar las f\'ormulas del caso anterior.\selectlanguage{spanish}%

