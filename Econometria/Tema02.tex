\chapter[Inferencia en el modelo lineal.]{Inferencia en el modelo lineal. \\ \normalsize Distribuci\'on muestral de los estimadores MCO. Contraste de hip\'otesis. Contraste acerca de un coeficiente del modelo. Contraste de un subconjunto param\'etrico. Inferencia para grandes muestras: consistencia, eficiencia y normalidad asint\'otica. Contrastes para grandes muestras basados en el Multiplicador de Lagrange. Predicci\'on en el modelo lineal.}


\section{Inferencia en el modelo lineal.}

Una vez hemos estimado el modelo puede ser interesante o necesario
el contraste de distintas hip\'otesis sobre los coeficientes del modelo,
para comprobar si cumplen las hip\'otesis te\'oricas subyacentes al mismo.
Esto contrastes pueden ser de significaci\'on (contrastar si una o varias
variables explicativas influyen en el valor de la variable dependiente),
de valor del mismo, o puede ser interesante la definici\'on de intervalos
o regiones de confianza para el valor de los par\'ametros.


\section{Contraste de hip\'otesis.}

A partir de ahora, mantendremos el supuesto de que el t\'ermino de error
del modelo sigue para cada observaci\'on una distribuci\'on normal, de
media cero y varianza constante para todas las observaciones $\sigma_{u}^{2}$,
es decir el vector $\boldsymbol{u}$ se distribuye seg\'un una normal
multivariante $N\left(\boldsymbol{0}_{N};\sigma_{u}^{2}\boldsymbol{I}_{N}\right)$.
Como el estimador MCO es una transformaci\'on lineal del vector $\boldsymbol{u}$,
se tiene que $\hat{\boldsymbol{\beta}}_{MCO}\sim N_{k}\left(\boldsymbol{\beta};\sigma_{u}^{2}\left(\boldsymbol{X}^{\prime}\boldsymbol{X}\right)^{-1}\right)$.

Adem\'as, se puede demostrar que siendo $\boldsymbol{x}\sim N_{k}\left(\boldsymbol{0}_{k};\sigma^{2}\boldsymbol{I}_{k}\right)$
y $\boldsymbol{A}$ una matriz sim\'etrica e idempotente de rango $r$,
entonces $\dfrac{1}{\sigma^{2}}\boldsymbol{x}^{\prime}\boldsymbol{A}\boldsymbol{x}\sim\chi^{2}(r)$.
Por tanto, como $\hat{\boldsymbol{u}}^{\prime}\hat{\boldsymbol{u}}=\boldsymbol{u}^{\prime}\boldsymbol{M}\boldsymbol{u}$
se deduce que $\dfrac{1}{\sigma_{u}^{2}}\hat{\boldsymbol{u}}^{\prime}\hat{\boldsymbol{u}}=\left(N-k\right)\dfrac{\hat{\sigma}_{u}^{2}}{\sigma_{u}^{2}}\sim\chi^{2}(N-k)$.

Como $\hat{\boldsymbol{\beta}}_{MCO}\sim N_{k}\left(\boldsymbol{\beta};\sigma_{u}^{2}\left(\boldsymbol{X}^{\prime}\boldsymbol{X}\right)^{-1}\right)$,
se deduce que $\boldsymbol{X}\left(\hat{\boldsymbol{\beta}}-\boldsymbol{\beta}\right)\sim N_{k}\left(\boldsymbol{0}_{k};\sigma_{u}^{2}\boldsymbol{I}_{k}\right)$,
as\'i que $\dfrac{1}{\sigma_{u}^{2}}\left(\hat{\boldsymbol{\beta}}-\boldsymbol{\beta}\right)^{\prime}\boldsymbol{X}^{\prime}\boldsymbol{I}_{k}\boldsymbol{X}\left(\hat{\boldsymbol{\beta}}-\boldsymbol{\beta}\right)=\dfrac{1}{\sigma_{u}^{2}}\left(\hat{\boldsymbol{\beta}}-\boldsymbol{\beta}\right)^{\prime}\boldsymbol{X}^{\prime}\boldsymbol{X}\left(\hat{\boldsymbol{\beta}}-\boldsymbol{\beta}\right)\sim\chi^{2}(k)$

Por otro lado, sabemos que $\hat{\boldsymbol{\beta}}-\boldsymbol{\beta}=\left(\boldsymbol{X}^{\prime}\boldsymbol{X}\right)^{-1}\boldsymbol{X}^{\prime}\boldsymbol{u}$
y que $\hat{\boldsymbol{u}}^{\prime}\hat{\boldsymbol{u}}=\boldsymbol{u}^{\prime}\boldsymbol{M}\boldsymbol{u}$.
Como una forma lineal $\boldsymbol{L}\boldsymbol{x}$ y una forma
cuadr\'atica $\boldsymbol{x}^{\prime}\boldsymbol{A}\boldsymbol{x}$ se distribuyen
de forma independiente si $\boldsymbol{LA}=\boldsymbol{0}$, y $\left(\boldsymbol{X}^{\prime}\boldsymbol{X}\right)^{-1}\boldsymbol{X}^{\prime}\boldsymbol{M}=\left(\boldsymbol{X}^{\prime}\boldsymbol{X}\right)^{-1}\boldsymbol{X}^{\prime}\left[\boldsymbol{I}_{N}-\boldsymbol{X}\left(\boldsymbol{X}^{\prime}\boldsymbol{X}\right)^{-1}\boldsymbol{X}^{\prime}\right]=\boldsymbol{0}_{N}$,
podemos decir que $\hat{\boldsymbol{\beta}}-\boldsymbol{\beta}$ y
$\hat{\boldsymbol{u}}^{\prime}\hat{\boldsymbol{u}}=\left(N-k\right)\hat{\sigma}_{u}^{2}$
se distribuyen de forma independiente.

Combinando todos estos resultados, podemos llegar a la conclusi\'on
de que el estad\'istico:
\[
\dfrac{\dfrac{1}{\sigma_{u}^{2}}\left(\hat{\boldsymbol{\beta}}-\boldsymbol{\beta}\right)^{\prime}\boldsymbol{X}^{\prime}\boldsymbol{X}\left(\hat{\boldsymbol{\beta}}-\boldsymbol{\beta}\right)/k}{\dfrac{1}{\sigma_{u}^{2}}\hat{\boldsymbol{u}}^{\prime}\hat{\boldsymbol{u}}/\left(N-k\right)}=\left(\hat{\boldsymbol{\beta}}-\boldsymbol{\beta}\right)^{\prime}\left[\hat{\sigma}_{u}^{2}\left(\boldsymbol{X}^{\prime}\boldsymbol{X}\right)^{-1}\right]^{-1}\left(\hat{\boldsymbol{\beta}}-\boldsymbol{\beta}\right)/k\sim F_{k,N-k}
\]


Con este estad\'istico podemos contrastar la hip\'otesis nula $H_{0}:\boldsymbol{\beta}=\boldsymbol{\beta}^{0}$,
sustituyendo en el estad\'istico el valor de $\boldsymbol{\beta}$ por
$\boldsymbol{\beta}^{0}$ y los estimadores por los valores obtenidos
en la estimaci\'on. SI el valor del estad\'istico es menor que el valor
de $F_{k,N-k}$ para el nivel de significaci\'on elegido, no podremos
rechazar la hip\'otesis nula. En otro caso, la rechazaremos.


\section{Tratamiento general del contraste de hip\'otesis.}

El apartado anterior nos permite contrastar una hip\'otesis sobre todos
los coeficientes del modelo, pero nos puede interesar contrastar hip\'otesis
sobre el valor de uno o varios coeficientes del modelo, o sobre la
significaci\'on de uno o varios de los coeficientes (es decir, sobre
si su valor es cero). Para ello vamos a desarrollar un m\'etodo m\'as
general de contrastaci\'on de hip\'otesis.

Definimos una hip\'otesis general, $H_{0}:\boldsymbol{R}\boldsymbol{\beta}=\boldsymbol{r}$,
siendo $\boldsymbol{R}$ una matriz $q\times k$, siendo $q$ el n\'umero
de restricciones, con los coeficientes de los par\'ametros $\boldsymbol{\beta}$
en cada una de las restricciones y $\boldsymbol{r}$ un vector con
$q$ filas con los valores de las restricciones. De esta forma podemos
definir cualquier conjunto de hip\'otesis lineales sobre los coeficientes
del modelo. Por ejemplo, para la hip\'otesis $H_{0}:\beta_{1}=\beta_{1}^{0}$,
$\boldsymbol{R}$ ser\'ia una matriz $1\times k$ con el primer t\'ermino
igual a 1 y el resto cero, y $\boldsymbol{r}$ ser\'ia un escalar con
valor $\beta^{0}$.

Como $\boldsymbol{R}$ es una matriz constante, $\boldsymbol{R}\left(\hat{\boldsymbol{\beta}}-\boldsymbol{\beta}\right)\sim N_{k}\left(\boldsymbol{0};\sigma_{u}^{2}\boldsymbol{R}\left(\boldsymbol{X}^{\prime}\boldsymbol{X}\right)^{-1}\boldsymbol{R}^{\prime}\right)$.
Si la hip\'otesis nula es cierta, $\boldsymbol{R}\boldsymbol{\beta}=\boldsymbol{r}$
y por tanto $\boldsymbol{R}\hat{\boldsymbol{\beta}}-\boldsymbol{r}\sim N_{q}\left(\boldsymbol{0}_{q};\sigma_{u}^{2}\boldsymbol{R}\left(\boldsymbol{X}^{\prime}\boldsymbol{X}\right)^{-1}\boldsymbol{R}^{\prime}\right)$.
Finalmente, $\left(\boldsymbol{R}\hat{\boldsymbol{\beta}}-\boldsymbol{r}\right)^{\prime}\left[\sigma_{u}^{2}\boldsymbol{R}\left(\boldsymbol{X}^{\prime}\boldsymbol{X}\right)^{-1}\boldsymbol{R}^{\prime}\right]^{-1}\left(\boldsymbol{R}\hat{\boldsymbol{\beta}}-\boldsymbol{r}\right)\sim\chi^{2}(q)$,
y por tanto,
\[
\dfrac{\left(\boldsymbol{R}\hat{\boldsymbol{\beta}}-\boldsymbol{r}\right)^{\prime}\left[\boldsymbol{R}\left(\boldsymbol{X}^{\prime}\boldsymbol{X}\right)^{-1}\boldsymbol{R}^{\prime}\right]^{-1}\left(\boldsymbol{R}\hat{\boldsymbol{\beta}}-\boldsymbol{r}\right)/q}{\dfrac{\hat{\boldsymbol{u}}^{\prime}\hat{\boldsymbol{u}}}{N-k}}\sim F_{q,N-k}
\]


O, lo que es lo mismo, $\left(\boldsymbol{R}\hat{\boldsymbol{\beta}}-\boldsymbol{r}\right)^{\prime}\left[\hat{\sigma}_{u}^{2}\boldsymbol{R}\left(\boldsymbol{X}^{\prime}\boldsymbol{X}\right)^{-1}\boldsymbol{R}^{\prime}\right]^{-1}\left(\boldsymbol{R}\hat{\boldsymbol{\beta}}-\boldsymbol{r}\right)/q\sim F_{q,N-k}$.
Y por tanto, si el valor del estad\'istico es menor que el valor de
$F_{k,N-k}$ para el nivel de significaci\'on elegido, no podremos rechazar
la hip\'otesis nula. En otro caso, la rechazaremos.


\section{Contraste acerca de un coeficiente del modelo.}

En este caso, $H_{0}:\beta_{i}=\beta_{i}^{0}$. Entonces, $\boldsymbol{R}=\left[0,0,\ldots,1,0,\ldots,0\right]$
ocupando el $1$ la posici\'on $i$-\'esima, y $\boldsymbol{r}=\beta_{i}^{0}$,
y por tanto $\boldsymbol{R}\hat{\boldsymbol{\beta}}-\boldsymbol{r}=\hat{\beta}_{i}-\beta_{i}^{0}$,
escalar. El producto $\boldsymbol{R}\left(\boldsymbol{X}^{\prime}\boldsymbol{X}\right)^{-1}\boldsymbol{R}^{\prime}=a_{ii}$,
siendo $a_{ii}$ el elemento $i$-\'esimo de la diagonal de la matriz
$\left(\boldsymbol{X}^{\prime}\boldsymbol{X}\right)^{-1}$, y el estad\'istico
se convierte en $\dfrac{\left(\hat{\beta}_{i}-\beta_{i}^{0}\right)^{2}}{\hat{\sigma}_{u}^{2}a_{ii}}\sim F_{1,N-k}$.
Como $Var(\hat{\boldsymbol{\beta}})=\sigma_{u}^{2}\left(\boldsymbol{X}^{\prime}\boldsymbol{X}\right)^{-1}$,
$\hat{Var}(\hat{\beta}_{i})=\hat{\sigma}_{u}^{2}a_{ii}$. Adem\'as,
aplicando ra\'ices cuadradas, $\dfrac{\hat{\beta}_{i}-\beta_{i}^{0}}{\hat{\sigma}_{u}\sqrt{a_{ii}}}\sim t_{N-k}$.
Para contrastar la significaci\'on de la variable explicativa $x_{i}$
en el modelo, contrastamos que el valor de su coeficiente sea igual
a cero, es decir, utilizamos el estad\'istico$\dfrac{\hat{\beta}_{i}}{\hat{\sigma}_{u}\sqrt{a_{ii}}}\sim t_{N-k}$,
que se suele conocer como el estad\'istico $t$ del cociente estimado
$\hat{\beta}_{i}$.


\section{Contraste de un subconjunto param\'etrico.}

Ahora contrastamos la significaci\'on de un subconjunto de variables
explicativas. Sin p\'erdida de generalidad supondremos que son las \'ultimas
del modelo, por tanto la matriz ser\'a: $\boldsymbol{R}=\left[\boldsymbol{0}_{s,k-s};\boldsymbol{I}_{s}\right]$
y el vector $\boldsymbol{r}=\boldsymbol{0}_{s}$. Particionaremos
la matriz $\boldsymbol{X}=\left[\boldsymbol{X}_{1};\boldsymbol{X}_{2}\right]$,
con $\boldsymbol{X}_{1}$ de dimensi\'on $N\times k-s$ y $\boldsymbol{X}_{2}$
de dimensiones $N\times s$, y el vector de par\'ametros en $\boldsymbol{\beta}=\left(\mathbf{\boldsymbol{\beta}_{1};\boldsymbol{\beta}_{2}}\right)$.
El modelo econom\'etrico puede escribirse como $\boldsymbol{y}=\left(\boldsymbol{X}_{1};\boldsymbol{X}_{2}\right)\left(\begin{array}{c}
\hat{\boldsymbol{\boldsymbol{\beta}}}_{1}\\
\boldsymbol{\hat{\boldsymbol{\beta}}}_{2}
\end{array}\right)+\hat{\boldsymbol{u}}=\boldsymbol{X}_{1}\hat{\boldsymbol{\boldsymbol{\beta}}}_{1}+\boldsymbol{X}_{2}\boldsymbol{\hat{\boldsymbol{\beta}}}_{2}+\hat{\boldsymbol{u}}$. Por tanto, $\boldsymbol{R}\hat{\boldsymbol{\beta}}-\boldsymbol{r}=\boldsymbol{\hat{\boldsymbol{\beta}}}_{2}$,
y el producto $\boldsymbol{R}\left(\boldsymbol{X}^{\prime}\boldsymbol{X}\right)^{-1}\boldsymbol{R}^{\prime}$
tiene como resultado la submatriz $s\times s$ inferior derecha de
$\left(\boldsymbol{X}^{\prime}\boldsymbol{X}\right)^{-1}$. Esta submatriz
es igual a $\left(\boldsymbol{X}^{\prime}_{2}\boldsymbol{M}_{1}\boldsymbol{X}_{2}\right)^{-1}$,
con $\boldsymbol{M}_{1}=\boldsymbol{I}_{N}-\boldsymbol{X_{1}}\left(\boldsymbol{X_{1}}^{\prime}\boldsymbol{X_{1}}\right)^{-1}\boldsymbol{X_{1}}^{\prime}$,
y por tanto, 
\[
F=\dfrac{\hat{\boldsymbol{\beta}}^{\prime}_{2}\left(\boldsymbol{X}^{\prime}_{2}\boldsymbol{M}_{1}\boldsymbol{X}_{2}\right)\hat{\boldsymbol{\beta}}_{2}/s}{\dfrac{\hat{\boldsymbol{u}}^{\prime}\hat{\boldsymbol{u}}}{N-k}}\sim F_{s,N-k}
\]


Si en lugar de contrastar la significaci\'on queremos contrastar los
valores de los par\'ametro, el estad\'{s}itico ser\'ia:
\[
F=\dfrac{\left(\hat{\boldsymbol{\beta}}_{2}-\boldsymbol{\beta}_{2}^{0}\right)^{\prime}\left(\boldsymbol{X}^{\prime}_{2}\boldsymbol{M}_{1}\boldsymbol{X}_{2}\right)\left(\hat{\boldsymbol{\beta}}_{2}-\boldsymbol{\beta}_{2}^{0}\right)/s}{\dfrac{\hat{\boldsymbol{u}}^{\prime}\hat{\boldsymbol{u}}}{N-k}}\sim F_{s,N-k}
\]



\section{Contraste de significaci\'on global del modelo.}

Si queremos contrastar la significaci\'on de todas las variables del
modelo la matriz ser\'a: $\boldsymbol{R}=\left[\boldsymbol{0}_{k-1,1};\boldsymbol{I}_{k-1}\right]$
y el vector $\boldsymbol{r}=\boldsymbol{0}_{k-1}$ (el t\'ermino independiente
no se contrasta). Particionaremos la matriz $\boldsymbol{X}=\left[\boldsymbol{1}_{N};\boldsymbol{X}_{2}\right]$,
con $\boldsymbol{X}_{1}$ de dimensi\'on $N\times1$ y $\boldsymbol{X}_{2}$
de dimensiones $N\times k-1$, y el vector de par\'ametros en $\boldsymbol{\beta}=\left(\beta_{1};\boldsymbol{\beta}_{2}\right)$.
\[
\boldsymbol{X}^{\prime}\boldsymbol{X}=\left(\begin{array}{cc}
N & \boldsymbol{1}^{\prime}_{N}\boldsymbol{X}_{2}\\
\boldsymbol{X}^{\prime}_{2}\boldsymbol{1}_{N} & \boldsymbol{X}^{\prime}_{2}\boldsymbol{X}_{2}
\end{array}\right)
\]


\[
F=\dfrac{\hat{\boldsymbol{\beta}}^{\prime}_{2}\left(\boldsymbol{X}^{\prime}_{2}\boldsymbol{Q}\boldsymbol{X}_{2}\right)\hat{\boldsymbol{\beta}}_{2}/\left(k-1\right)}{\dfrac{\hat{\boldsymbol{u}}^{\prime}\hat{\boldsymbol{u}}}{N-k}}\sim F_{k-1,N-k}
\]


Con $\boldsymbol{Q}=\boldsymbol{I}_{N}-\dfrac{1}{N}\boldsymbol{1}_{N}\boldsymbol{1}^{\prime}_{N}$.
Este estad\'istico admite una expresi\'on alternativa: $F=\dfrac{\dfrac{SCE}{k-1}}{\dfrac{SCR}{N-k}}=\dfrac{R^{2}/\left(k-1\right)}{\left(1-R^{2}\right)/\left(N-k\right)}$,
s\'olo si el modelo contiene un t\'ermino independiente. Aunque ninguna
de las variables sea significativa, el t\'ermino independiente ser\'ia
aproximadamente igual a la media de la variable end\'ogena, y por tanto
deber\'ia ser significativo.


\section{Intervalos y regiones de confianza.}

Obtener una estimaci\'on puntual a partir de una muestra dada no es
muy \'util a no ser que se pueda proporcionar un intervalo de confianza
de dicha estimaci\'on. Para obtener estos intervalos utilizaremos los
resultados anteriores.


\subsection{Intervalo de confianza para un solo coeficiente.}

Hemos visto que el estad\'istico $\dfrac{\hat{\beta}_{i}-\beta_{i}}{\hat{\sigma}_{u}\sqrt{a_{ii}}}\sim t_{N-k}$.
Si definimos $t_{N-k;\alpha/2}$ como el valor para el que se cumple
$P\left(t\geq t_{N-k;\alpha/2}\right)=\alpha/2$ en una distribuci\'on
$t$ de Student, un intervalo de confianza del $\left(1-\alpha\right)\%$
ser\'a $\left[\hat{\beta}_{i}-t_{N-k;\alpha/2}\hat{\sigma}_{u}\sqrt{a_{ii}},\hat{\beta}_{i}+t_{N-k;\alpha/2}\hat{\sigma}_{u}\sqrt{a_{ii}}\right]$.
El tama�o del intervalo depende directamente de la varianza del estimador
de $\beta_{i}$. Una vez construido el intervalo, contrastar una hip\'otesis
$H_{0}:\beta_{i}=\beta_{i}^{0}$ equivale a rechazar la hip\'otesis
nula si $\beta_{i}^{0}$ est\'a fuera del intervalo de confianza y no
rechazarla si $\beta_{i}^{0}$ est\'a dentro del intervalo de confianza.


\subsection{Regiones de confianza para varios coeficientes.}

Cuando se busca un rango de valores para varios coeficientes, no se
tiene un intervalo, se tiene una regi\'on de confianza. An\'alogamente
a la anterior secci\'on, utilizaremos el estad\'istico $F$. La matriz
$\boldsymbol{R}$ ser\'a la que seleccione los coeficientes para los
que queremos hallar la regi\'on de confianza, y el estad\'istico resultante
ser\'a:
\[
\dfrac{\left(\hat{\boldsymbol{\beta}}_{2}-\boldsymbol{\beta}_{2}\right)^{\prime}\left[\left(\boldsymbol{X}^{\prime}\boldsymbol{X}\right)_{2}^{-1}\right]^{-1}\left(\hat{\boldsymbol{\beta}}_{2}-\boldsymbol{\beta}_{2}\right)/s}{\hat{\sigma}_{u}^{2}}\sim F_{s,N-k}
\]


donde $\boldsymbol{\beta}_{2}$ denota el subvector cuya regi\'on de
confianza se quiere construir y $\left(\boldsymbol{X}^{\prime}\boldsymbol{X}\right)_{2}^{-1}$
denota la submatriz correspondiente extra\'ida de la matriz $\left(\boldsymbol{X}^{\prime}\boldsymbol{X}\right)^{-1}$.
SI colocamos los coeficientes que nos interesan al final del modelo,
ya hemos visto que $\left(\boldsymbol{X}^{\prime}\boldsymbol{X}\right)_{2}^{-1}=\left(\boldsymbol{X}^{\prime}_{2}\boldsymbol{M}_{1}\boldsymbol{X}_{2}\right)^{-1}$,
con $\boldsymbol{M}_{1}=\boldsymbol{I}_{N}-\boldsymbol{X_{1}}\left(\boldsymbol{X_{1}^{\prime}}\boldsymbol{X_{1}}\right)^{-1}\boldsymbol{X_{1}}^{\prime}$,
y la regi\'on de confianza se obtiene mediante:

\[
1-\alpha=P\left[\left(\hat{\boldsymbol{\beta}}_{2}-\boldsymbol{\beta}_{2}\right)^{\prime}\left(\boldsymbol{X}^{\prime}_{2}\boldsymbol{M}_{1}\boldsymbol{X}_{2}\right)\left(\hat{\boldsymbol{\beta}}_{2}-\boldsymbol{\beta}_{2}\right)\leq\lambda_{\alpha}s\hat{\sigma}_{u}^{2}\right]
\]


Con $\lambda_{\alpha}$ un valor tal que $P\left(F\leq\lambda_{\alpha}\right)=1-\alpha$
en una distribuci\'on $F_{s,N-k}$. Es conveniente recordar que $\left(\boldsymbol{X}^{\prime}_{2}\boldsymbol{M}_{1}\boldsymbol{X}_{2}\right)$
es la submatriz de $\left(\boldsymbol{X}^{\prime}\boldsymbol{X}\right)$ correspondiente
a los coeficientes considerados.

La dificultas surge de que el expresi\'on a la izquierda de la desigualdad
es un polinomio de grado igual al n\'umero de coeficientes considerados,
por lo que estas regiones s\'olo son \'utiles para el caso de dos coeficientes.
La regi\'on de confianza consiste en una elipse, cuyas dimensiones dependen
de las varianzas con las que se hayan estimado los coeficientes y
su inclinaci\'on de la covarianza entre las estimaciones de los coeficientes.
Se puede contrastar la hip\'otesis conjunta examinando si el punto compuesto
por los valores de la hip\'otesis est\'a dentro de la regi\'on de confianza.


\section{Estimaci\'on bajo restricciones.}

En la mayor\'ia de las ocasiones en econometr\'ia se contrasta una hip\'otesis
de cuya validez se est\'a razonablemente seguro, por lo que si el contraste
no la rechaza se consideran como v\'alidas en el modelo que se est\'e
considerando. Dado que el concepto de potencia de un contraste hace
que una hip\'otesis nula no se descarte aun siendo err\'onea, conviene
contrastar \'unicamente hip\'otesis nulas aceptables desde el punto de
vista conceptual.

Parece l\'ogico entonces que si contrastamos una hip\'otesis nula y \'esta
no se rechaza, el modelo ganar\'ia en precisi\'on incorporando esta hip\'otesis
al proceso de estimaci\'on. Veamos c\'omo.

Buscaremos un estimador que minimice la suma de cuadrados, pero esta
vez imponiendo un conjunto de restricciones lineales, $\boldsymbol{R}\boldsymbol{\beta}=\boldsymbol{r}$.
El lagrangiano del problema ser\'a:
\[
L=\left(\boldsymbol{y}-\boldsymbol{X}\boldsymbol{\beta}\right)^{\prime}\left(\boldsymbol{y}-\boldsymbol{X}\boldsymbol{\beta}\right)-2\lambda^{\prime}\left(\boldsymbol{R}\boldsymbol{\beta}-\boldsymbol{r}\right)
\]
 donde $\lambda$ es un vector de dimensiones $q\times1$, siendo
$q$ el n\'umero de restricciones a considerar, de multiplicadores de
Lagrange. Minimizamos el lagrangiano tomando derivadas:
\[
\begin{array}{cc}
\dfrac{\partial L}{\partial\boldsymbol{\beta}}=-2\boldsymbol{X}^{\prime}\boldsymbol{y}+2\boldsymbol{X}^{\prime}\boldsymbol{X}\boldsymbol{\beta}-2\boldsymbol{R}^{\prime}\lambda=\boldsymbol{0}_{k} & \left(k\;{\textstyle derivadas}\right)\\
\dfrac{\partial L}{\partial\boldsymbol{\beta}}=-2\left(\boldsymbol{R}\boldsymbol{\beta}-\boldsymbol{r}\right)=\boldsymbol{0}_{q} & \left(q\;{\textstyle derivadas}\right)
\end{array}
\]
 las soluciones a este sistema de ecuaciones son el estimador de \textbf{m\'inimos
cuadrados restringidos}, $MCR$, y el vector de precios sombra (multiplicadores
de Lagrange) de las restricciones.

Premultiplicando la primera ecuaci\'on por $\boldsymbol{R}\left(\boldsymbol{X}^{\prime}\boldsymbol{X}\right)^{-1}$:

\[
\boldsymbol{R}\hat{\boldsymbol{\beta}}_{R}-\boldsymbol{R}\left(\boldsymbol{X}^{\prime}\boldsymbol{X}\right)^{-1}\boldsymbol{X}^{\prime}\boldsymbol{y}-\boldsymbol{R}\left(\boldsymbol{X}^{\prime}\boldsymbol{X}\right)^{-1}\boldsymbol{R}^{\prime}\lambda=\boldsymbol{0}_{k}
\]


Y como $\hat{\boldsymbol{\beta}}_{MCO}=\left(\boldsymbol{X}^{\prime}\boldsymbol{X}\right)^{-1}\boldsymbol{X}^{\prime}\boldsymbol{y}$,
y seg\'un la segunda ecuaci\'on $\boldsymbol{R}\hat{\boldsymbol{\beta}}_{R}=\boldsymbol{r}$,
tenemos que:
\[
\lambda=\left[\boldsymbol{R}\left(\boldsymbol{X}^{\prime}\boldsymbol{X}\right)^{-1}\boldsymbol{R}^{\prime}\right]^{-1}\left(\boldsymbol{r}-\boldsymbol{R}\hat{\boldsymbol{\beta}}_{MCO}\right)
\]


Y, sustituyendo $\lambda$ en la primera ecuaci\'on:
\[
\begin{array}{c}
\left(\boldsymbol{X}^{\prime}\boldsymbol{X}\right)\hat{\boldsymbol{\beta}}_{R}-\boldsymbol{X}^{\prime}\boldsymbol{y}-\boldsymbol{R}^{\prime}\left[\boldsymbol{R}\left(\boldsymbol{X}^{\prime}\boldsymbol{X}\right)^{-1}\boldsymbol{R}^{\prime}\right]^{-1}\left(\boldsymbol{r}-\boldsymbol{R}\hat{\boldsymbol{\beta}}_{MCO}\right)=\boldsymbol{0}_{k}\\
\hat{\boldsymbol{\beta}}_{R}=\hat{\boldsymbol{\beta}}_{MCO}+\left(\boldsymbol{X}^{\prime}\boldsymbol{X}\right)^{-1}\boldsymbol{R}^{\prime}\left[\boldsymbol{R}\left(\boldsymbol{X}^{\prime}\boldsymbol{X}\right)^{-1}\boldsymbol{R}^{\prime}\right]^{-1}\left(\boldsymbol{r}-\boldsymbol{R}\hat{\boldsymbol{\beta}}_{MCO}\right)
\end{array}
\]
 donde $\hat{\boldsymbol{\beta}}_{R}$ es el estimador de m\'inimos
cuadrados restringidos del modelo. La interpretaci\'on que podemos inferir
es que el estimador restringido es una correcci\'on del estimador sin
restringir. Esta correcci\'on ser\'a mayor cuanto m\'as lejos est\'e el estimador
$MCO$ de satisfacer las restricciones. Hay una serie de caracter\'isticas
del estimador restringido:
\begin{enumerate}
\item El estimador $MCR$ es insesgado s\'olo si las restricciones bajo las
que se ha obtenido son ciertas.
\item El estimador $MCR$ satisface las restricciones.
\item El estimador $MCR$ difiere del estimador $MCO$ s\'olo si \'este no satisface
las restricciones en $H_{0}$, si el estimador $MCO$ satisface las
restricciones coincide con el estimador $MCR$.
\item La matriz de covarianzas del estimador $MCR$ es siempre inferior
a la matriz de covarianzas del estimador $MCO$, incluso si las restricciones
no son ciertas. Esto ocurre porque al imponer las restricciones limitamos
la regi\'on del espacio param\'etrico en que buscamos el estimador.
\end{enumerate}
La matriz de covarianzas del estimador ser\'a:

\[
\begin{array}{c}
Var(\hat{\boldsymbol{\beta}}_{R})=\sigma_{u}^{2}\left[\left(\boldsymbol{X}^{\prime}\boldsymbol{X}\right)^{-1}-\left(\boldsymbol{X}^{\prime}\boldsymbol{X}\right)^{-1}\boldsymbol{R}^{\prime}\left[\boldsymbol{R}\left(\boldsymbol{X}^{\prime}\boldsymbol{X}\right)^{-1}\boldsymbol{R}^{\prime}\right]^{-1}\boldsymbol{R}\left(\boldsymbol{X}^{\prime}\boldsymbol{X}\right)^{-1}\right]\\
Var(\hat{\boldsymbol{\beta}}_{MCO})-Var(\hat{\boldsymbol{\beta}}_{R})=\sigma_{u}^{2}\left(\boldsymbol{X}^{\prime}\boldsymbol{X}\right)^{-1}\boldsymbol{R}^{\prime}\left[\boldsymbol{R}\left(\boldsymbol{X}^{\prime}\boldsymbol{X}\right)^{-1}\boldsymbol{R}^{\prime}\right]^{-1}\boldsymbol{R}\left(\boldsymbol{X}^{\prime}\boldsymbol{X}\right)^{-1}
\end{array}
\]
 la diferencia es una matriz semidefinida positiva, es decir, los
elementos de la diagonal principal de $Var(\hat{\boldsymbol{\beta}}_{R})$
ser\'an menores que los de $Var(\hat{\boldsymbol{\beta}}_{MCO})$.


\section{Contraste de cambio estructural: Test de Chow.}

Un contraste bastante importante es el que contrasta la hip\'otesis
nula de que dos submuestras han sido generadas por la misma estructura
, es decir, el mismo modelo. Se utiliza cuando se tiene informaci\'on
acerca de un cambio que pueda afectar al modelo en un momento dado
y se quiere contrastar si esa variaci\'on afecta a los coeficientes
del modelo. Se considera el modelo restringido, $MR$,: 
\[
\begin{array}{cc}
y_{i}=\boldsymbol{x}^{\prime}_{i}\boldsymbol{\beta}+u_{i} & i=1,2,\ldots,N_{1},N_{1}+1,\ldots,N\end{array}
\]
 y el modelo sin restringir, $MSR$:

\[
\begin{array}{cc}
y_{i}=\boldsymbol{x}^{\prime}_{i}\boldsymbol{\beta}_{1}+u_{i} & i=1,2,\ldots,N_{1}\\
y_{i}=\boldsymbol{x}^{\prime}_{i}\boldsymbol{\beta}_{2}+u_{i} & i=N+1,\ldots,N
\end{array}
\]
 con una regresi\'on diferente para cada submuestra. La hip\'otesis nula
es: $H_{0}:\boldsymbol{\beta}_{1}=\boldsymbol{\beta}_{2}$. A este
contraste se le llama Test de Chow. Llamamos suma residual restringida,
$SRR$, a la suma de los residuos del modelo restringido, y suma residual
sin restringir, $SRS$, al agregado de las sumas residuales de los
dos modelos sin restringir, que denotamos por $SR_{1}$ y $SR_{2}$.
El estad\'istico para el contraste de la hip\'otesis de ausencia de cambio
estructura ser\'a:
\[
F=\dfrac{\dfrac{SRR-\left(SR_{1}+SR_{2}\right)}{k}}{\dfrac{SR_{1}+SR_{2}}{N-2k}}\sim F_{k,N-2k}
\]


Si el estad\'istico $F$ es mayor que el valor de $F_{k,N-2k}$ para
el nivel de significaci\'on contemplado, rechazaremos la hip\'otesis nula
de ausencia de cambio estructural.

Muchas veces se utiliza este contraste para comprobar si las \'ultimas
observaciones recibidas suponen un cambio respecto al resto de la
muestra. En estos casos puede que no se pueda estimar el segundo modelo
sin restricciones por falta de grados de libertad.

Si estamos en el caso l\'imite en el que $N_{2}=k$, los residuos son
cero, y por tanto $SR_{2}=0$, por lo que el estad\'istico se reduce
a:
\[
F=\dfrac{\left(SRR-SR_{1}\right)/T_{2}}{SR_{1}/\left(T_{1}-k\right)}\sim F_{T_{2},T_{1}-k}
\]


Si tenemos que $T_{2}<k$ se puede demostrar que el estad\'istico anterior
sigue siendo v\'alido.


\section{Predicci\'on en el modelo lineal.}

El objeto final de los modelos lineales es, una vez estimados los
mismos, utilizarlos para hacer predicciones sobre la variable end\'ogena
conocidos los valores de las variables explicativas. Esto tiene sentido
ya que el modelo representa la relaci\'on entre las variables, y es
v\'alido a menos que la relaci\'on sea muy inestable.


\subsection{C\'alculo de las predicciones.}

El estimador $MCO$ refleja la mejor relaci\'on lineal entre las variables
explicativas y la variable end\'ogena para la muestra de la que disponemos.
Suponemos que esa relaci\'on estimada es tambi\'en la mejor relaci\'on entre
las variables fuera de la muestra. Bajo este supuesto, denotamos por
$E_{T}$ el valor esperado en base a la informaci\'on disponible hasta
el momento $T$. La funci\'on a utilizar para predecir $y_{T+1}$ ser\'a:
\[
E_{T}y_{T+1}=E_{T}\left(\boldsymbol{x}^{\prime}_{T+1}\boldsymbol{\beta}+u_{T+1}\right)=E_{T}\left(\boldsymbol{x}^{\prime}_{T+1}\boldsymbol{\beta}\right)+E_{T}u_{T+1}=E_{T}\left(\boldsymbol{x}^{\prime}_{T+1}\right)\hat{\boldsymbol{\beta}}_{T}+E_{T}u_{T+1}
\]
Por tanto, para predecir $\boldsymbol{x}^{\prime}_{T+1}\boldsymbol{\beta}$
multiplicamos los valores previstos para las $\boldsymbol{x}^{\prime}_{T+1}$
por el estimador $MCO$ de los coeficientes. Este estimador lo representamos
como $\hat{\boldsymbol{\beta}}_{T}$ para explicitar que se ha obtenido
con datos hasta $T$. Hemos supuesto que es lo suficientemente estable
como para poder usarlo para predecir $y_{T+1}$. As\'i pues, necesitaremos
predecir el vector de variables explicativas y el t\'ermino de error.

Las variables explicativas pueden ser conocidas de antemano (ventas
en funci\'on de precios, demanda de inversi\'on en funci\'on de saldos monetarios)
o puede ser necesario estimarlas.

En cuanto al t\'ermino de error, dado que es una sucesi\'on de variables
aleatorias independientes entre s\'i y la muestra no nos proporciona
ninguna informaci\'on acerca de \'el, lo predecimos con su esperanza matem\'atica,
que ya hemos visto que es cero.

Por tanto, para obtener buenas predicciones necesitaremos:
\begin{itemize}
\item Que la relaci\'on lineal entre las variables se mantenga fuera de la
muestra.
\item Que los coeficientes sean lo suficientemente estables como para que
sus estimaciones obtenidas con la muestra sean una buena aproximaci\'on
de los valores que se obtendr\'ian incorporando las observaciones que
queremos predecir.
\item Que se conozcan los valores de las variables $\boldsymbol{x}$ para
los casos que queremos predecir, o que se puedan estimar de forma
suficientemente fiable.
\item Que el modelo est\'e bien especificado.
\item Que el horizonte de predicci\'on no est\'e muy lejano.
\end{itemize}
Por tanto, en caso de que se cumplan estos requisitos, la predicci\'on
m\'inimo-cuadr\'atica de $y_{T+1}$ ser\'ia: $E_{T}y_{T+1}=\boldsymbol{x}^{\prime}_{T+1}\hat{\boldsymbol{\beta}}_{T}$.

De todas formas, una predicci\'on no es muy \'util si no podemos dar un
intervalo de confianza de la misma. Vamos a calcularlo.


\subsection{Error de predicci\'on.}

El error de predicci\'on se define como la diferencia entre el valor
de la variable a predecir y la predicci\'on obtenida:
\[
e_{T}(1)=y_{T+1}-E_{T}y_{T+1}=\boldsymbol{x}^{\prime}_{T+1}\boldsymbol{\beta}-\boldsymbol{x}^{\prime}_{T+1}\hat{\boldsymbol{\beta}}_{T}+u_{T+1}=\boldsymbol{x}^{\prime}_{T+1}\left(\boldsymbol{\beta}-\hat{\boldsymbol{\beta}}_{T}\right)+u_{T+1}
\]


Que es una variable aleatoria con un valor desconocido, puesto que
su realizaci\'on ocurrir\'a en el instante $T+1$. Las fuentes de este
error son:
\begin{itemize}
\item El error en la predicci\'on de $\boldsymbol{x}^{\prime}_{T+1}$.
\item El error en la estimaci\'on del vector $\boldsymbol{\beta}$.
\item El error estoc\'astico inherente al modelo, $u_{T+1}$.
\end{itemize}
Como el estimador es insesgado, el error de predicci\'on tiene esperanza
cero. As\'i, cuando las variables ex\'ogenas son conocidas de antemano
la predicci\'on es insesgada.

La varianza del error de predicci\'on ser\'a: 
\[
\begin{array}{c}
\sigma_{e}^{2}=E\left\{ \boldsymbol{x}^{\prime}_{T+1}\left(\boldsymbol{\beta}-\hat{\boldsymbol{\beta}}_{T}\right)\left(\boldsymbol{\beta}-\hat{\boldsymbol{\beta}}_{T}\right)^{\prime}\boldsymbol{x}{}_{T+1}+2\boldsymbol{x}^{\prime}_{T+1}\left(\boldsymbol{\beta}-\hat{\boldsymbol{\beta}}_{T}\right)u_{T+1}+u_{T+1}^{2}\right\} =\\
=\boldsymbol{x}^{\prime}_{T+1}E\left[\left(\boldsymbol{\beta}-\hat{\boldsymbol{\beta}}_{T}\right)\left(\boldsymbol{\beta}-\hat{\boldsymbol{\beta}}_{T}\right)^{\prime}\right]\boldsymbol{x}{}_{T+1}+E\left(u_{T+1}^{2}\right)=\sigma_{u}^{2}\boldsymbol{x}^{\prime}_{T+1}\left(\boldsymbol{X}_{T}^{\prime}\boldsymbol{X}_{T}\right)^{-1}\boldsymbol{x}{}_{T+1}+\sigma_{u}^{2}
\end{array}
\]


Donde se ha usado que $E\left(\hat{\boldsymbol{\beta}}_{T}u_{T+1}\right)=0$,
ya que $u_{T+1}$ es independiente de los errores anteriores. De esta
f\'ormula el \'unico par\'ametro desconocido es $\sigma_{u}^{2}$, que sustituiremos
por su estimador.


\subsection{Intervalos de confianza para la predicci\'on.}

Bajo el supuesto de normalidad del t\'ermino de error, el error de predicci\'on
es combinaci\'on lineal de dos variables con distribuci\'on normal:

\[
e_{T}(1)=-\boldsymbol{x}^{\prime}_{T+1}\left(\hat{\boldsymbol{\beta}}_{T}-\boldsymbol{\beta}\right)+u_{T+1}=-\boldsymbol{x}^{\prime}_{T+1}\left(\boldsymbol{X}_{T}^{\prime}\boldsymbol{X}_{T}\right)^{-1}\boldsymbol{X}_{T}^{\prime}u_{T}+u_{T+1}
\]
 y por tanto, $e_{t}(1)\sim N(0,\sigma_{e}^{2})$, donde $\sigma_{e}^{2}$
lo hemos calculado antes. Por un razonamiento an\'alogo al realizado
con las inferencias sobre coeficientes, tenemos que 
\[
\dfrac{e_{T}(1)}{\hat{\sigma}_{e}^{2}}=\dfrac{y_{T+1}-E_{T}y_{T+1}}{\hat{\sigma}_{e}^{2}}\sim t_{T-k}
\]
 y podemos utilizar esta expresi\'on para calcular un intervalo de confianza
para el valor futuro $y_{T+1}$.

Estos resultados s\'olo son v\'alidos si se cumplen los supuestos que
hemos asumido para obtenerlos. En particular, si las variables $\boldsymbol{x}{}_{T+1}$
no son conocidos con certeza, las expresiones son cotas inferiores
para la varianza del error de predicci\'on. En tales situaciones se
podr\'ia emplear la desigualdad de Tchebichev, expresada como $P\left[\left|E_{T}y_{T+1}-y_{t+1}\right|\geq\lambda\sigma_{e}\right]\leq\dfrac{1}{\lambda^{2}}$,
asignando a $\lambda$ un valor apropiado (por ejemplo, tal que $\dfrac{1}{\lambda^{2}}=0,05$,
y suponiendo que sustituir $\sigma_{e}$ por su estimador no supondr\'a
un gran error. Por tanto 
\[
P\left[E_{T}y_{T+1}-\lambda\hat{\sigma}_{e}\leq y_{T+1}\leq E_{T}y_{T+1}+\lambda\hat{\sigma}_{e}\right]\geq1-\dfrac{1}{\lambda^{2}}
\]

