
\chapter[Modelo lineal: especificaci\'on.]{Modelo lineal: especificaci\'on. \\
\normalsize  Formas funcionales lineales y no lineales en el modelo de regresi\'on m\'ultiple. Regresi\'on m\'ultiple con variables explicativas con informaci\'on cualitativa. Regresi\'on m\'ultiple con variables binarias que interact\'uan. Uso de variables proxy para variables explicativas no observables. Errores de especificaci\'on. Contraste RESET. Contraste contra alternativas no anidadas.}


\sectioncol{Introducci\'on.}

\sectioncol{Formas funcionales lineales y no lineales en el modelo de regresi\'on m\'ultiple.}



\subsectioncol{Formas funcionales lineales.}

Veamos el significado de los coeficientes del modelo lineal. Si tenemos el modelo:
\[y=\beta_0+\beta_1x_1+\beta_2x_2+\cdots+\beta_kx_k+u\]

Supongamos que la variable $x_2$ var\'ia en una unidad manteniendo el resto de coeficientes constantes. Entonces:
\[\hat{y}_1=\hat{\beta}_0+\hat{\beta}_1x_1+\hat{\beta}_2x_2+\cdots+\hat{\beta}_kx_k\]
\[\hat{y}_2=\hat{\beta}_0+\hat{\beta}_1x_1+\hat{\beta}_2(x_2+1)+\cdots+\hat{\beta}_kx_k\]
\[\hat{y}_2-\hat{y}_1=\Delta\hat{y}=\hat{\beta}_2\]

Por tanto, el coeficiente de cada variable nos dice lo que var\'ia la estimaci\'on de la variable dependiente por cada unidad de dicha variable, manteniendose el resto de variables independientes constantes.

EL problema con esta interpretaci\'on es que los coeficientes dependen de las unidades de medida de cada variable, y por tanto no podemos saber la importancia relativa de las variables independientes si sus unidades de medida son dif\'iciles de interpretar (por ejemplo, las puntuaciones de un examen.

EN ese caso, podemos tipificar anto las variable independientes como la variable dependiente. Para ello les restamos su media muestral y las dividimos por su desviaci\'on t\'ipica muestral. As\'i, si el modelo estimado es:
\[y_i=\hat{\beta}_0+\hat{\beta}_1x_{i1}+\hat{\beta}_2x_{i2}+\cdots+\hat{\beta}_kx_{ik}+\hat{u}_i\]

Si promediamos la ecuaci\'on para todas las unidades de la muestra, teniendo en cuanta que los residuos tienen de media cero, tenemos que:

\[\bar{y}=\hat{\beta}_0+\hat{\beta}_1\bar{x}_{1}+\hat{\beta}_2\bar{x}_{2}+\cdots+\hat{\beta}_k\bar{x}_{k}\]

Y restando esta ecuaci\'on a las $n$ ecuaciones del modelo:
\[y_i-\bar{y}=\hat{\beta}_1(x_{i1}-\bar{x}_{1})+\hat{\beta}_2(x_{i2}-\bar{x}_{2})+\cdots+\hat{\beta}_k(x_{ik}-\bar{x}_{k})+\hat{u}_i\]

Si llamamos $\hat{\sigma}_y$ a la desviaci\'on t\'ipica muestral de la variable dependiente y $\hat{\sigma}_i$ a la desviaci\'on t\'ipica muestral de la variable independiente $x_i$, tras unos c\'alculos sencillos llegamos a la siguiente ecuaci\'on:
\[(y_i-\bar{y})/\hat{\sigma}_y=(\hat{\sigma}_1/\hat{\sigma}_y)\hat{\beta}_1[(x_{i1}-\bar{x}_{1})/\hat{\sigma}_k]+(\hat{\sigma}_2/\hat{\sigma}_y)\hat{\beta}_2[(x_{i2}-\bar{x}_{2})/\hat{\sigma}_2]+\cdots+(\hat{\sigma}_k/\hat{\sigma}_y)\hat{\beta}_k[(x_{ik}-\bar{x}_{k})/\hat{\sigma}_k]+(\hat{u}_i/\hat{\sigma}_y)\]


Y ajustando este modelo llegamos a los coeficientes tipificados o \textit{coeficientes beta}, $\hat{b}_i$, cuya f\'ormula es: $\hat{b}_i=(\hat{\sigma}_i/\hat{\sigma}_y)\hat{\beta}_i$.

EL significado de estos coeficientes \textit{beta} es el siguiente: si $x_i$ aumenta en una desviaci\'on t\'ipica, entonces la variable dependiente, $y$, aumentar\'a en $\hat{b}_i$ desviaciones t\'ipicas. As\'i, podemos medir la influencia de cada variable independiente en la variable dependiente sin tener en cuenta las unidades de medida, lo que puede ser interesante.


El modelo lineal s\'olo tiene que ser lineal en los coeficientes para que podamos estimarlo. As\'i, en muchos casos es interesante transfromar las variables por distintas razones.

\subsectioncol{Formas funcionales logar\'itmicas.}

Veamos una serie de casos en las que las variables se transforman tomando logaritmos. Por ejemplo, el modelo:

\[\log{y}=\beta_0+\beta_1\log{x_1}+\beta_2x_2+\cdots+\beta_kx_k+u\]

Veamos la interpretaci\'on que tiene el coeficiente $\beta_1$: Si $x_1$ var\'ia manteni\'endose el resto de variables constantes, tendremos:

\[\hat{\log{y_1}}=\hat{\beta}_0+\hat{\beta}_1\log{x_1}+\hat{\beta}_2x_2+\cdots+\hat{\beta}_kx_k\]
\[\hat{\log{y_2}}=\hat{\beta}_0+\hat{\beta}_1\log{(x_1+\Delta x_1)}+\hat{\beta}_2x_2+\cdots+\hat{\beta}_kx_k\]
\[\hat{\log{y_2}}-\hat{\log{y_1}}=\hat{\beta}_1(\log{(x_1+\Delta x_1)}-\log{x_1})\]

Y como $\log{x}-\log{y}=\log{x/y}$, y $\log{(1+x)}=\sum_{n=1}^{\infty}\dfrac{(-1)^{n-1}}{n}x^n\approx x$ si $|x|$ es suficientemente peque\~no, podemos decir:
\[\hat{\log{y_2}}-\hat{\log{y_1}}=\hat{\log{y_2/y_1}}\approx\dfrac{y_2-y_1}{y_1}\]
\[\hat{\log{x_1+\Delta x_1}}-\hat{\log{x_1}}=\hat{\log{x_1+\Delta x_1/x_1}}\approx\dfrac{\Delta x_1}{x_1}\]

Y por tanto, 
\[\beta_1=\dfrac{\Delta y/y}{\Delta x_1/x_1}\]
Es decir, el coeficiente es igual a la elasticidad de $y$ respecto a $x$. Por tanto, por cada incremento de $x_1$ en un $1\%$, $y$ se incrementa un $\beta_1\%$. Esta aproximaci\'on solo es v\'alida si tanto $\Delta y/y$ como $\Delta x_1/x_1$ son suficientemente peque\~nos como para considerar $\sum_{n=2}^{\infty}\dfrac{(-1)^{n-1}}{n}x^n$ despreciable.

Si realizamos el mismo ejercicio para el coeficiente $\beta_2$, tenemos que:

\[\hat{\log{y+\Delta y}}-\hat{\log{y}}=\hat{\beta}_2\Delta x_2\]

Aplicando la misma aproximaci\'on tenemos que 
\[\%\hat{\Delta y}\approx100\hat{\beta}_2\Delta x_2 \]

Es decir, que por cada incremento en una unidad de $x_2$, $y$ se incrementa un $100\hat{\beta}_2\%$. Sin embargo, esta aproximaci\'on no es v\'alida para cambios grandes en el logaritmo. EN estos casos, el valor exacto ser\'a:
\[\%\hat{\Delta y}=100[e^{\hat{\beta}_2\Delta x_2}-1] \]
y si $x_2$ var\'ia en una unidad, $y$ var\'ia un $100[e^{\hat{\beta}_2}-1]\%$. Hay que tener en cuenta que este estimador no es insesgado, aunque s\'i es consistente. Esto se debe a que la esperanza no es invariante ante funciones no lineales, aunque la convergencia en probabilidad es invariante ante funciones cont\'inuas.

Si tuvi\'esemos el modelo:
\[y=\beta_0+\beta_1\log{x_1}+\beta_2x_2+\cdots+\beta_kx_k+u\]

Siguiendo los razonamientos anteriores, podremos decir que el coeficiente $\beta_1$ nos dice que si $x_1$ var\'ia un $1\%$, $y$ var\'ia $100\hat{\beta}_1$ unidades, siempre que la variaci\'on de $x_1$ no sea muy grande.

Aparte de las interpretaciones de los coeficientes, hay que tener en cuenta que para los coeficientes de las variables logar\'itmicas son invariantes ante cambios de escala, con lo cual tambi\'en son independientes de la sunidades en que est\'en expresadas las mismas.

Si $y>0$, $\log(y)$ puede hacer que la variable cumpla mejor las condiciones del modelo, reduciendo o incluso eliminando la heterocedasticidad y asimetr\'ias en la variable. Tambi\'en ocurre que tomar logaritmos reduce el rango de variaci\'on de la variable, por lo que las estimaciones ser\'an menos sensibles a valores extremos tanto de las variables independientes como de la dependiente. En general se suelen tomar logaritmos cuando las variables toman vaores positivos muy grandes. Si la variable empieza en cero, se puede transformar mediante $\log(1+y)$, sin que cambie la interpretaci\'on de los coeficientes , salvo en el caso de que $y=0$.

Un inconveniente que tiene el tomar logaritmos de la variable $y$ es que esto nos permite estimar el valor de $\log y$, no el de $y$. Adem\'as, dos modelos, uno en $\log y$ y el otro en $y$ no se pueden comparar mediante sus $R^2$.

\subsectioncol{Formas funcionales cuadr\'aticas.}

Las funciones cuadr\'aticas son muy \'utilse para reflejar efectos marginales decrecientes de $x$ sobre $y$. Por ejemplo, en el modelo $y=\beta_0+\beta_1x+\beta_2x^2+u$, tenemos que $\partial y/\partial x=\beta_1+2\beta_2x$, y si $\beta_2<0$, el efecto de $x$ sobre $y$ decrecer\'a a medida que aumente $x$.

Por tanto, para incrementos peque\~nos tenemos que 
\[\Delta\hat{y}\approx(\hat{\beta}_1+2\hat{\beta}_2x)\Delta x\]

EL problema de emplear una forma funcional cuadr\'atica es que llega un momento en que la relaci\'on entre $x$ e $y$ se invierte, y esto no siempre es l\'ogico desde el punto de vista del modelo. Es cecir, si $x=-\dfrac{\hat{\beta}_1}{2\hat{\beta}_2}$, el efecto de la $x$ sobre la $y$ es nulo, y a partir de ese punto pasa a ser de signo contrario al que ten\'ia antes. Si esto no tiene l\'ogica en la situaci\'on de nuestro modelo se puede deber a varios factores:
\begin{itemize}
\item El punto en el que el efecto se hace cero est\'a fuera del rango de valores posibles de la $x$.
\item Nuestra muestra tiene muy pocos valores en ese intervalo.
\item Hay alg\'un tipo de sesgo en nuestros datos.
\item El modelo est\'a infraespecificado: hay alguna otra variable que explica el comportamiento.
\end{itemize}


Para el caso de modelos del tipo:
\[\log{y}=\beta_0+\beta_1x+\beta_2x^2+u\]

Se interpreta como que el efecto del cambio en la $x$ afecta al incremento porcentual de $y$ de forma creciente/decreciente, seg\'un el valor de los coeficientes.

Si el modelo es del tipo:
\[\log{y}=\beta_0+\beta_1\log{x}+\beta_2[\log{x}]^2+u\]

Se interpreta como que la elasticidad de $y$ con respecto a $x$ es variable, de la forma $\hat{\beta}_1+\hat{\beta}_2\log{x}$.

Tambi\'en se pueden estimar modelos con formas polin\'omicas de grado superior. La interpretaci\'on de los coeficientes en estos casos es m\'as compleja.

\subsectioncol{T\'erminos de interacci\'on.}

Veremos ahora modelos en los que las variables independientes interaccionan entre ellas. Por ejemplo:
\[y=\beta_0+\beta_1x_1+\beta_2x_2+\beta_3x_1x_2+u\]

En este caso, el efecto parcial de $x_2$ sobre $y$, manteniendo constantes el resto de variables ser\'a:
\[\dfrac{\Delta{\hat{y}}}{\Delta{x_2}}=\hat{\beta}_2+\hat{\beta}_3x_1\]

Si $\beta_3>0$, el efecto de $x_2$ sobre $y$ ser\'a mayor cuanto mayor sea $x_1$, es decir, las variables interaccionan. Para resumir esta interacci\'on podemos calcular el efecto de $x_2$ sobre $y$ para valores interesantes de $x_1$ como su media, su mediana, su moda o alg\'un cuantil que consideremos interesante.

Si reparametrizamos el modelo de la siguiente forma:
\[y=\alpha_0+\delta_1x_1+\delta_2x_2+\beta_3(x_1-\mu_1)(x_2-\mu_2)+u\]

Siendo $\mu_1$ y $\mu_2$ las medias de $x_1$ y $x_2$, los coeficientes $\delta_1$  es el efecto parcial de $x_1$ en el valor medio de $x_2$, y $\delta_2$ es el efecto parcial de $x_2$ en el valor medio de $x_1$. Esto lo podemos reproducir para cualquier valor interesante de $x_1$ y $x_2$ que queramos considerar.

\sectioncol{Regresi\'on m\'ultiple con variables explicativas con informaci\'on cualitativa.}

En muchos casos nuestros modelos tienen variables explicativas cualitativas, es decir, no num\'ericas, como por ejemplo el sexo de un individuo, la provincia en la que est\'a situada una vivienda, el partido pol\'itico al que se ha votado... Vamos a ver como incluir estas variables en nuestros modelos.

\subsectioncol{Descripci\'on de las variables cualitativas.}

Si tenemos una variable cualitativa que puede tomar $r$ valores o categor\'ias, debemos definir $r-1$ variables ficticias.

Es decir, sea la caracter\'istica $z$ que puede tomar los valores $a_1, a_2,\ldots,a_r$, definimos las variables ficticias $z_1, z_2,\ldots,z_{r-1}$, tales que 
\[z_i=\left\{\begin{array}{ll}
1 & \text{si } z=a_i \\
0 & \text{si } z\neq a_i 
\end{array}\right.\]

No definimos la variable $z_k$, porque dado que el individuo siempre ha de pertenecer a una categor\'ia, tendr\'iamos que $z_k=1-\sum_{i=1}^{r-1}z_i$, y esto introducir\'ia colinealidad en el modelo.

Los valores $1$ y $0$ son arbitrarios, y se podr\'ian haber elegido otros. Se eligen estos porque facilitan la interpretaci\'on de los coeficientes del modelo.

As\'i, el modelo quedar\'ia:
\[y=\beta_0+\beta_1x_1+\cdots\beta_kx_k+\delta_1z_1+\cdots\delta_{r-1}z_{r-1}+u\]

Y podr\'iamos realizar la estimaci\'on del modelo como habitualmente.

\subsectioncol{Interpretaci\'on de los coeficientes.}

Si se cumple el supuesto de esperanza condicionada nula para el t\'ermino de error, cada coeficiente $\delta_i$ se puede expresar como:

\[\delta_i=E(y|x_1,\ldots,x_k,z_1,\ldots,z_i=1,\ldots,z_{r-1})-E(y|x_1,\ldots,x_k,z_1,\ldots,z_i=0,\ldots,z_{r-1})\]

Es decir, es la diferencia entre la esperanza de la variable dependiente si el individuo presenta el valor $a_i$ en la caracter\'istica $z$ o si presenta el valor $a_r$ (ya que el resto de los $z_j$, permanecen constantes y por lo tanto valdr\'an cero). As\'i, vemos que la categor\'ia que excluimos de las variables act\'ua como ``categor\'ia base'' con la que se comparan todas las dem\'as. Es conveniente tener esto en cuenta a la hora de elegir dicha categor\'ia base.

En el caso de que la variable dependiente se presente en forma logar\'itmica, los coeficientes $\delta$ multiplicados por 100 se interpretan como la diferencia en porcentaje entre la categor\'ia base y la categor\'ia del coeficiente.

En el caso de que los valores que puede tomar nuestra variable sean demasiados como para desglosarlos en variables ficticias, podemos agrupar los mismos en categor\'ias, especialmente si estamos hablando de variables ordinales.


\sectioncol{Regresi\'on m\'ultiple con variables binarias que interact\'uan.}

\subsectioncol{Interacci\'on entre variables binarias.}

Puede haber modelos en los que aparezcan dos caracter\'isticas binarias que pueden interactuar. Para reflejar estos casos definimos modelos del tipo:

\[y=\beta_0+\delta_1z_1+\delta_2z_2+\delta_3z_1z_2+u\]

En este caso, si $z_1=1$, la diferencia entre ambas categor\'ias de la caracter\'istica que refleja $z_2$ es de $\delta_2+\delta_3$, y si $z_1=0$ la diferencia es de $\delta_2$, y de forma an\'aloga la influencia de $z_1$ depende de $z_2$. Contrastar la significatividad de $\delta_3$ implica contrastar que las caracter\'isticas no interact\'uan.

\subsectioncol{Interacci\'on entre variables binarias y ordinarias.}

Supongamos que tenes un modelo con la caracter\'istica $z$, binaria, que modelamos mediante la variable $z_1$, y una caracter\'istica num\'erica que modelamos mediante la variable $x_1$  modelamos seg\'un elmodelo:
\[y=\beta_0+\delta_0z_1+\beta_1x_1+\delta_1z_1x_1+u\]

En este caso, el coeficiente $\delta_0$ refleja la diferencia entre los t\'erminos constantes de ambas categor\'ias de la caracter\'istica $z$, y el coeficiente $\delta_1$ refleja la diferencia de pendiente entre las categor\'ias. Es decir, la pertenencia a una u otra categor\'ia tambi\'en influye en el impacto de la variable $x_1$ sobre la variable dependiente.

Para contrastar que la caracter\'istica $z$ no influye en el impacto de la variable $x_1$, tendremos que hacer un contraste de significaci\'on sobre el coeficiente $\delta_1$. Para contrastar que la caracter\'istica $z$ no influye en la variable $y$, tenemos que contrastar la significatividad conjnta de $\delta_0$ y $\delta_1$.

Este mismo sistema se puede utilizar si queremos contrastar si hay diferencia en el modelo de regresi\'on para dos grupos distintos de una poblaci\'on que se distinguen en una caracter\'aitica. Para ello, definimos la variable $z_1$ tal que $z_1=1$, $z_1=2$, refleja la pertenencia  acada grupo. Estimamos el modelo:

\[y=\beta_0 + \delta_0z_1+\sum_{i=1}^k\beta_ix_i+\sum_{i=1}^k\delta_ix_iz_1+u\]

Y para contrastar la hip\'otesis nula de que ambos modelos son iguales, contrastamos la significaci\'on conjunta de las variables $\delta_i$, es decir, $H_0:\delta_0=\delta_1=\ldots=\delta_k=0$, frente a $h_1:\exists i/\delta_i\neq 0$.

Esto tambi\'en se puede contrastar usando el contraste de ambio estructural de Chow, que consiste en estimar el modelo restringido, considerando ambos grupos como una sola poblaci\'on, y por otro lado el modelo sin restringir, en el que consideramos ambos grupos como dso poblaciones y estimamos un modelo para cada uno de ellos. De esta forma, la suma de cuadrados de los residuos del modelo sin restringir saer\'a simplemente la suma de cuadados de los residuos de ambos modelos, y por tanto el estad\'istico del contraste ser\'a:
\[F=\dfrac{SCR_T-(SCR_1+SCR_2)}{(SCR_1+SCR_2)}\dfrac{n-2(k+1)}{k+1}\sim F_{k+1,n-2(k+1)}\]

A este contraste se le llama contraste de Chow. Como es de tipo $F$, requiere que exista homocedasticidad. Si realizamos un an\'alisis asint\'otico no es necesaria la normalidad de los residuos.

En el contraste de Chow la hip\'otesis nula no permite ninguna diferencia entre los grupos, y muchas veces es \'util contrastar que las pendientes no dependen de la pertenencia a un grupo, aunqe el t\'ermino constante s\'i pueda depender. Para contrastar esta situaci\'on se puede realizar, bien contrastando la significatividad conjunta \'unicamente de los t\'erminos de interacci\'on. La otra es formando un estad\'istico $F$ en el que la regresi\'on del modelo restringido incluye una variable binaria ficticia en la que distinguimos entre los dos grupos.

\sectioncol{Uso de variables proxy para variables explicativas no observables.}

Hay ocasiones en las que no podemos incluir en el modelo alguna variable explicativa, debido a que \'esta es no observable (por ejemplo, la influencia de la habilidad en el salario de los trabajadores). El problema de esta exclusi\'on es que hace que los estimadores de los coeficientes sean sesgados, ya que si dividimos la matriz de variables independientes, $\boldsymbol{X}=\left(\boldsymbol{X_0};\boldsymbol{Z}\right)$, siendo $\boldsymbol{X_0}$ matriz $n\times r$ de variables observadas, y $\boldsymbol{Z}$ matriz $n\times(k-r)$ matriz de variables no observadas y omitidas, y siendo $\boldsymbol{\beta}^{\prime}=\left(\boldsymbol{\beta_0}^{\prime};\boldsymbol{\beta_z}^{\prime}\right)$, siendo $\boldsymbol{\beta_0}$ vector $r\times 1$ de coeficientes de las variables observadas, y $\boldsymbol{\beta_Z}$ matriz $(k-r)\times 1$ vector de coeficientes de las variable omitidas, el estimador de m\'inimos cuadrados ordinarios para el modelo con las variables omitidas ser\'a:

\[\hat{\boldsymbol{\beta}}_{\boldsymbol{0}}=\left(\boldsymbol{X_0}^{\prime}\boldsymbol{X_0}\right)^{-1}\boldsymbol{X_0}^{\prime}\boldsymbol{y}\] 

Y como la expresi\'on real del modelo es: 
\[\boldsymbol{y}=\left[\boldsymbol{X_0}; \boldsymbol{Z}\right]\left[\begin{matrix}
\boldsymbol{\beta_0} \\
\boldsymbol{\beta_Z}
\end{matrix}\right]+\boldsymbol{u} \]

Tenemos que:

\[\hat{\boldsymbol{\beta}}_{\boldsymbol{0}}=\left(\boldsymbol{X_0}^{\prime}\boldsymbol{X_0}\right)^{-1}\boldsymbol{X_0}^{\prime}\left[\boldsymbol{X_0}; \boldsymbol{Z}\right]\left[\begin{matrix}
\boldsymbol{\beta_0} \\
\boldsymbol{\beta_Z}
\end{matrix}\right]+\left(\boldsymbol{X_0}^{\prime}\boldsymbol{X_0}\right)^{-1}\boldsymbol{X_0}^{\prime}\boldsymbol{u}\]

Y como:
\[\left(\boldsymbol{X_0}^{\prime}\boldsymbol{X_0}\right)^{-1}\boldsymbol{X_0}^{\prime}\left[\boldsymbol{X_0}; \boldsymbol{Z}\right]=\left[\left(\boldsymbol{X_0}^{\prime}\boldsymbol{X_0}\right)^{-1}\boldsymbol{X_0}^{\prime}\boldsymbol{X_0};\left(\boldsymbol{X_0}^{\prime}\boldsymbol{X_0}\right)^{-1}\boldsymbol{X_0}^{\prime}\boldsymbol{Z}\right]=\left[\boldsymbol{I_r};\left(\boldsymbol{X_0}^{\prime}\boldsymbol{X_0}\right)^{-1}\boldsymbol{X_0}^{\prime}\boldsymbol{Z}\right]\]

Como consecuencia:

\[E(\hat{\boldsymbol{\beta}}_{\boldsymbol{0}})=\left[\boldsymbol{I_r};\left(\boldsymbol{X_0}^{\prime}\boldsymbol{X_0}\right)^{-1}\boldsymbol{X_0}^{\prime}\boldsymbol{Z}\right]\left[\begin{matrix}
\boldsymbol{\beta_0} \\
\boldsymbol{\beta_Z}
\end{matrix}\right]=\boldsymbol{\beta_0}+\left(\boldsymbol{X_0}^{\prime}\boldsymbol{X_0}\right)^{-1}\boldsymbol{X_0}^{\prime}\boldsymbol{Z}\boldsymbol{\beta_Z}\]

Con lo que si las variables omitidas son significativas el estimador ser\'a sesgado.

En general, lo que nos suele interesar es conocer el efecto de las variables observadas sobre nuestra varaible dependiente. El t\'ermino independiente no nos suele interesar, y dado que las
variables omitidas no son observables, no tiene mucho sentido conocer su efecto sobre la variable dependiente, dado que en general no sabremos interpretar ese valor.

Para eliminar o reducir el sesgo sobre las variables omitidas, intentaremos sutituir las variables no observables por variables que s\'i se puedan observar y mantengan una correlaci\'on con ellas. A estas variables las llamaremos \textbf{variables proxy}.

Supongamos que tenemos el modelo:
\[\boldsymbol{y}=\left[\boldsymbol{X_0}; \boldsymbol{X_1}\right]\left[\begin{matrix}
\boldsymbol{\beta_0} \\
\boldsymbol{\beta_1}
\end{matrix}\right]+\boldsymbol{u} \]
 visto anteriormente, con $k-r$ variables $\boldsymbol{X_1}$ no observables. SI podemos encontrar $k-r$ variables proxy que denotamos por $\boldsymbol{Z}$, que s\'i podamos observar y est\'en relacionadas con las variables no onsevables, podremos escribir para cada una de ellas:
\[x_{1i}=\delta_{0i}+\delta_{1i}z_i+v_i\]

Donde $v_i$ es la perturbaci\'on resultante de que la correlaci\'on entre ambas variables no es perfecta, y si $\delta_{1i}=0$, no tendremos una buena variable proxy. El m\'etodo consiste en sustituir las variables originales con las proxy. A este procedimiento se le conoce como \textbf{soluci\'on por sustituci\'on del problema de variables omitidas}. Como estamos reemplazando unas variables por otras que no son iguales, vamos a ver c\'omo deben ser estas variables proxy para que los estimadores sean consistentes.

Para que los estimadores de los coeficientes de las variables no omitidas sean consistentes se tiene que cumplir:
\begin{itemize}
\item Si se cumple las hip\'otesis del modelo lineal, el error $u$ estar\'a incorrelacionado con las $\boldsymbol{X_0}$ y $\boldsymbol{X_1}$. Adem\'as, $u$ tami\'en debe estar incorrelacionado con las $\boldsymbol{Z}$. Esto quiere decir que si el modelo incluye las variables $\boldsymbol{X_0}$ y $\boldsymbol{X_1}$, entonces las $\boldsymbol{Z}$ son irrelevantes para el modelo, lo cual es l\'ogico, dado que s\'olo deben influir en el modelo por su relaci\'on con las $\boldsymbol{X_1}$. Por tanto, esta hip\'otesis es bastante razonable.
\item Los errores $v_1$ est\'an incorrelacionados con $\boldsymbol{X_0}$, $\boldsymbol{X_1}$ y $\boldsymbol{Z}$. Para que esto ocurra, las $\boldsymbol{Z}$ deben ser buenas aproximaciones de las $\boldsymbol{X_1}$, y no deben depender de las $\boldsymbol{X_0}$.
\end{itemize}

Si estimamos el modelo sustituyendo las $\boldsymbol{X_1}$ por las $\boldsymbol{Z}$:
\[\hat{\boldsymbol{\beta}}=\left(\left[\begin{matrix}
\boldsymbol{X_0}^{\prime} \\
\boldsymbol{Z}^{\prime}
\end{matrix}\right]\left[\boldsymbol{X_0}; \boldsymbol{Z}\right]\right)^{-1}\left[\begin{matrix}
\boldsymbol{X_0}^{\prime} \\
\boldsymbol{Z}^{\prime}
\end{matrix}\right]\left[\boldsymbol{X_0}; \boldsymbol{X_1}\right]\left[\begin{matrix}
\boldsymbol{\beta_0} \\
\boldsymbol{\beta_1}
\end{matrix}\right]+\left(\left[\begin{matrix}
\boldsymbol{X_0}^{\prime} \\
\boldsymbol{Z}^{\prime}
\end{matrix}\right]\left[\boldsymbol{X_0}; \boldsymbol{Z}\right]\right)^{-1}\left[\begin{matrix}
\boldsymbol{X_0}^{\prime} \\
\boldsymbol{Z}^{\prime}
\end{matrix}\right]\boldsymbol{u}\]

Por tanto, si ponemos el modelo en forma de ecuaciones:

\[y=\beta_0+\sum_{i=1}^r\beta_ix_i+\sum_{i=r+1}^k\beta_ix_i+u\]

Y sustituimos las $x_i$ no observadas por su expresi\'on en fucni\'on de las $z_i$:

\[y=\beta_0+\sum_{i=1}^r\beta_ix_i+\sum_{i=r+1}^k\beta_i(\delta_{0i}+\delta_{1i}z_i+v_i)+u\]

Y operando y reordenando, obtenemos:

\[y=\beta_0+\sum_{i=r+1}^k\beta_i\delta_{0i}+\sum_{i=1}^r\beta_ix_i+\sum_{i=r+1}^k\beta_i\delta_{1i}z_i+u+\sum_{i=r+1}^k\beta_iv_i\]

COn lo cual vemos que si sustituimos la svariables no observadas por sus variables proxy, obtenemos un modelo con estas caracter\'isticas:
\begin{itemize}
\item Si se cumplen las dos hip\'otesis que hemos fijado, cumple las hip\'otesis de un modelo lineal, ya que $u+\sum_{i=r+1}^k\beta_iv_i$ est\'an incorrelacionadas con las variables ex\'ogenas. Por tanto, los estimadores ser\'an insesgados o al menos consistentes.
\item El estimador de los coeficientes correspondientes a las variables observadas no var\'ia.
\item Tanto el t\'ermino independiente como los coeficientes de las variables no observadas no se estiman, pero ya hemos visto que estos en cualquier caso no nos interesan. Obtenemos en su lugar estimaciones del t\'ermino independiente y los coeficientes para las variables proxy, que pueden o no tener inter\'es para nuestro estudio.
\end{itemize}

Es importante tener en cuenta que si las variables proxy no cumplen las hip\'otesis que les hemos exigido, los estimadores que obtengamos ser\'an sesgados.

\colsubsection{Uso de variables dependientes retardadas como variables proxy.}

Hay vces en las que tenemos claro cual es la variable independiente no observable que afecta a nuestro modelo, y podemos encontrar una variable proxy adecuada. Sin embargo, e otras ocasiones bien no tenemos claro que factor puede afectar al modelo, o no podemos encontrar una variable proxy. En estos casos, se suele utilizar un retardo de la variable dependiente como mecanismo de control de las variables ocultas.

Esto aumenta los requisitos sobre los datos si estamos trabajando sobre datos de corte transversal (necesitamos datos de dos periodos), pero nos permite tener en cuenta factores hist\'oricos que pueden tener influencia en el valor actual de la variable dependiente. De hecho, es de suponer los factores no observables que tengan influencia sobre el modelo en la actualidad tambi\'en la tendr\'ian en periodos pasados.

Este es un m\'etodo que est\'a bastante extendido en la pr\'actica, y aunque no es perfecto, ayuda a mejorar la estimaci\'on en muchos casos.


\sectioncol{Errores de especificaci\'on.}

ntendemos como error de especificaci\'on aquel que se comete cuando el modelo que se quiere estimar no coincide con el modelo real que siguen los datos. Estos errores pueden ser de tres tipos: por omisi\'on de variables relevantes, por inclusi\'on de variables irrelevantes y por mala especificaci\'on funcional del modelo.

\subsectioncol{Omisi\'on de variables relevantes.}

Supongamos que tenemos un modelo del tipo:
\[\boldsymbol{y}=\left[\boldsymbol{X_0}; \boldsymbol{X_1}\right]\left[\begin{matrix}
\boldsymbol{\beta_0} \\
\boldsymbol{\beta_1}
\end{matrix}\right]+\boldsymbol{u} \]

Pero que, debido a las razones que sean, nosostros estimamos asumiendo que el modelo es:

\[ \boldsymbol{y}=\boldsymbol{X_0}\boldsymbol{\beta_0}+\boldsymbol{u} \]

El estimador MC que obtendremos ser\'a:
\[\hat{\boldsymbol{\beta}}=\left(\boldsymbol{X_0}^{\prime}\boldsymbol{X_0}\right)^{-1}\boldsymbol{X_0}^{\prime}\boldsymbol{y}\] 

Si sustituimos las $\boldsymbol{y}$ por su expresi\'on seg\'un el modelo correcto:

\[\hat{\boldsymbol{\beta}}=\left(\boldsymbol{X_0}^{\prime}\boldsymbol{X_0}\right)^{-1}\boldsymbol{X_0}^{\prime}\left[\boldsymbol{X_0}; \boldsymbol{X_1}\right]\left[\begin{matrix}
\boldsymbol{\beta_0} \\
\boldsymbol{\beta_1}
\end{matrix}\right]+\left(\boldsymbol{X_0}^{\prime}\boldsymbol{X_0}\right)^{-1}\boldsymbol{X_0}^{\prime}\boldsymbol{u}\]

Y como:
\[\left(\boldsymbol{X_0}^{\prime}\boldsymbol{X_0}\right)^{-1}\boldsymbol{X_0}^{\prime}\left[\boldsymbol{X_0}; \boldsymbol{X_1}\right]=\left[\left(\boldsymbol{X_0}^{\prime}\boldsymbol{X_0}\right)^{-1}\boldsymbol{X_0}^{\prime}\boldsymbol{X_0};\left(\boldsymbol{X_0}^{\prime}\boldsymbol{X_0}\right)^{-1}\boldsymbol{X_0}^{\prime}\boldsymbol{X_1}\right]=\left[\boldsymbol{I_r};\left(\boldsymbol{X_0}^{\prime}\boldsymbol{X_0}\right)^{-1}\boldsymbol{X_0}^{\prime}\boldsymbol{X_1}\right]\]

Como consecuencia:

\[E(\hat{\boldsymbol{\beta}})=\left[\boldsymbol{I_r};\left(\boldsymbol{X_0}^{\prime}\boldsymbol{X_0}\right)^{-1}\boldsymbol{X_0}^{\prime}\boldsymbol{X_1}\right]\left[\begin{matrix}
\boldsymbol{\beta_0} \\
\boldsymbol{\beta_1}
\end{matrix}\right]=\boldsymbol{\beta_0}+\left(\boldsymbol{X_0}^{\prime}\boldsymbol{X_0}\right)^{-1}\boldsymbol{X_0}^{\prime}\boldsymbol{X_1}\boldsymbol{\beta_1}\]

Con lo que si las variables omitidas son significativas el estimador ser\'a sesgado. El sesgo se compone del producto de dos factores:
\begin{itemize}
\item Una matriz que tiene por columnas los estimadores por m\'inimos cuadrados de los coeficientes de las regresiones de las variables excluidas sobre las no excluidas.
\item El vector de coeficientes que tienen las variables excluidas en el modelo verdadero. 
\end{itemize}

Por tanto, si las variables excluidas son ortogonales a las no excluidas, es decir, si $\boldsymbol{X_0}^{\prime}\boldsymbol{X_1}=\boldsymbol{0}_{r\times(k-r)}$. Evidentemente, esto es altamente improbable.

Por otro lado, dado que la estimaci\'on de la varianza de la perturbaci\'on es $\hat{\sigma}^2_u=\dfrac{\hat{\boldsymbol{u}}^{\prime}\hat{\boldsymbol{u}}}{N-k}$, si definimos $\boldsymbol{M}_0=\boldsymbol{I}_N-\boldsymbol{X_0}(\boldsymbol{X_0}^{\prime}\boldsymbol{X_0})^{-1}\boldsymbol{X_0}^{\prime}$, la estimaci\'on que calcularemos ser\'a:
\[E(\dfrac{\hat{\boldsymbol{u}}^{\prime}\hat{\boldsymbol{u}}}{N-r})=\sigma^u+\dfrac{\boldsymbol{\beta}^{\prime}\boldsymbol{X}^{\prime}\boldsymbol{M}_0\boldsymbol{X}\boldsymbol{\beta}}{N-r}\]

Y la estimaci\'on de la varianza tendr\'a un sesgo positivo.

\subsectioncol{Inclusi\'on de variables no relevantes.}

En este caso, la matriz de observaciones de las variables ex\'ogenas que usaremos para la estimaci\'on ser\'a $\boldsymbol{X}_0=\left[\boldsymbol{X};\boldsymbol{X}_1\right]$, donde $\boldsymbol{X}_1$ es una matriz $N\times s$ de variables no relevantes para el modelo. Por tanto, es estimador ser\'a:

\[\hat{\boldsymbol{\beta}}_0=\left(\boldsymbol{X_0}^{\prime}\boldsymbol{X_0}\right)^{-1}\boldsymbol{X_0}^{\prime}\boldsymbol{y}\] 

Si sustituimos las $\boldsymbol{y}$ por su expresi\'on seg\'un el modelo correcto:

\[\hat{\boldsymbol{\beta}}=\left(\boldsymbol{X_0}^{\prime}\boldsymbol{X_0}\right)^{-1}\boldsymbol{X_0}^{\prime}\boldsymbol{X}\boldsymbol{\beta}+\left(\boldsymbol{X_0}^{\prime}\boldsymbol{X_0}\right)^{-1}\boldsymbol{X_0}^{\prime}\boldsymbol{u}\]

Y como:
\[\left(\boldsymbol{X_0}^{\prime}\boldsymbol{X_0}\right)^{-1}\left(\boldsymbol{X_0}^{\prime}\boldsymbol{X_0}\right)=\boldsymbol{I}_{k+s}=\left(\boldsymbol{X_0}^{\prime}\boldsymbol{X_0}\right)^{-1}\left(\boldsymbol{X_0}^{\prime}\boldsymbol{X};\boldsymbol{X_0}^{\prime}\boldsymbol{X_1}\right)\]

Por tanto,
\[\left(\boldsymbol{X_0}^{\prime}\boldsymbol{X_0}\right)^{-1}\left(\boldsymbol{X_0}^{\prime}\boldsymbol{X}\right)=\left(\begin{matrix}
\boldsymbol{I}_k \\
\boldsymbol{0}_{s\times k}
\end{matrix}\right)\]

\[E\left(\begin{matrix}
\hat{\boldsymbol{\beta}} \\
\hat{\boldsymbol{\beta}}_1
\end{matrix}\right)=\left(\begin{matrix}
\boldsymbol{I}_k \\
\boldsymbol{0}_{s\times k}
\end{matrix}\right)\boldsymbol{\beta}_0=\left(\begin{matrix}
\boldsymbol{\beta} \\
\boldsymbol{0}_{s}
\end{matrix}\right)\]

Por tanto, el estimador de los coeficientes de las variables relevantes es insesgado, y el de las variables irrelevantes tiene esperanza cero, por lo que cabe esperar que los soeficientes de estas \'ultimas resultasen ser no significativos.

Se puede demostrar que el estimador de la varianza de la perturbaci\'on tambi\'en es insesgado.

Por tanto, podr\'ia parecer que la estrategia a seguir deber\'ia ser incluir cuantas m\'as variables sea posible en el modelo, dado que esto no introduce sesgo en los estimadores. Sin embargo, el incluir m\'as variables de las necesarias aumenta la varianza del estimador por lo que no solo perdemos precisi\'on, sino que nos puede hacer creer de forma err\'onea que un coeficiente no es significativo. 

\subsectioncol{Errores de especificaci\'on funcional.}

Un modelo de regresi\'on lineal adolece de errores de mala especificaci\'on cuando la forma funcional que proponemos no coincide con la forma funciona que relaciona las variables en la realidad.

Si, por ejemplo, tenemos un fen\'omeno cuya relaci\'on entre variables es $y=\beta_0+\beta_1x_1+\beta_2x_2+\beta_3x_2^2+u$, pero nosotros estimamos el modelo $y=\beta_0+\beta_1x_1+\beta_2x_2+u$, ya hemos visto que al omitir una variable relevante, los estimadores ser\'an sesgados, con un tama\~no para el sesgo que depende del valor de $\beta_3$ y de la correlaci\'on enytre las variables. Sin embargo, las consecuencias sobre el estimador del efecto de $x_2$ en el modelo son a\'un peores, pues aunque obtuvi\'esemos un estimador insesgado para $\beta_2$ nos estar\'iamos dejando el efecto de la variable al cuadrado y por tanto el efecto total ser\'ia $\beta_2+2\beta_3x_2$. Vemos que la diferencia entre el efecto real y el de nuestro modelo ser\'a mayor cuanto mayor sea $x_2$.

Este mismo problema se introduce cuando en la relaci\'on funcional aparece un t\'ermino de interacci\'on entre dos variables: no solo los coeficientes que obtenemos son sesgados, sino que los efectos de las variables que aparecen enla interacci\'on tendr\'an una interpretaci\'on err\'onea.

Otro posible mala especificaci\'on ocurrir\'a si usamos la variable en forma lineal en lugra de en forma logar\'itmica. Entonces los estimadores de los efectos parciales no ser\'an ni insesgados ni consistentes. Veremos m\'as adelante algunos contrastes para contemplar estos casos.

En general, estas malas especificaciones funcionales se pueden intuir incluyendo t\'erminos cudr\'aticos en el modelo y relaizando un contraste $F$ de significaci\'on conjunta. SI los t\'erminos cuadr\'aticos son significativos, pueden incluirse en el modelo, aunque a veces pueden reflejar otros problemas con la especificaci\'on funcional, como el uso de la variable en nivel en vez de en logaritmos o viceversa. Puede ser dif\'icil identificar la causa por la que la forma funcional que se ha seleccionado es incorrecta. En muchos casos, para detectar estas relaciones no lineales es suficiente con usar logaritmos de ciertas variables y a\~nadir t\'erminos al cuadrado.

\sectioncol{Contraste RESET.}

Se han propuesto varios contrastes para detectar la mala especificaci\'on funcional. El contraste de RESET(regression specification error test) de Ramsey ha resultado muy \'util en este sentido.

El contraste es uy sencillo: si el modelo original,
\[y=\beta_0+\beta_1x_1+\cdots+\beta_kx_k+u\]
Cumple la hip\'otesis de que la media de la perturbaci\'on condicionada a las variable independientes es cero, entonces ninguna funci\'on no lineal que a\~nadamos al modelo deber\'ia ser significativa.

Si a\~nadi\'esemos la svariables en forma cuadr\'atica, nos encontrar\'iamos con el problema de que si tenemos muchas varaibles los grados de libertad del modelo disminuyen mucho, adem\'as de que no estamos contemplando otras posibles formas funcionales. Para evitar esto, se a\~naden polinomios de las predicciones del modelo original ajustadas por MCO.

Tenemos que decidir cu\'antas funciones de los valores ajustados vamos a incluir en el contraste. No hay una respuesta v\'alida para todos los casos, pero con los t\'erminos al cuadrado y al cubo suele ser suficiente. Por tanto, la ecuaci\'on ampliada quedar\'a:

\[y=\beta_0+\beta_1x_1+\cdots+\beta_kx_k+\delta_1\hat{y}^2+\delta_2\hat{y}^3+u\]

La hip\'otesis nula es que el modelo original es correcto. Por tanto, contrastamos $H_0:\delta_1=\delta_2=0$ en el modelo ampliado. Para eso utilizamos el estad\'istico $F$, que si es significativo nos indicar\'a un posible problema de especificaci\'on funcional. Bajo la hip\'otesis nula, $F\sim F_{2;n-k-3}$ aproximadamente para muestras grandes. Tambi\'en podemos utilizar la versi\'on LM del contraste, o utilizar los m\'etodos apropiados para que el contraste sea robusto a la heterocedasticidad.

El problema de este contraste es que no nos dice nada de la especificaci\'on funcional correcta si se rechaza la hip\'otesis nula.

A veces se utiliza RESET como un contraste mucho m\'as general, para detectar variables omitidas e incluso heterocedasticidad. Sin embargo, la potencia de RESET para detectar variables omitidas es muy baja si \'estas tienen esperanzas que son lineales en las variables incluidas en el modelo, y es incapaz de detectar la heterocedasticidad si el modelo est\'a bien especificado. Por tanto, solamente debe usarse para detectar mala especificaci\'on funcional.


\sectioncol{Contraste contra alternativas no anidadas.}

Obtener contrastes para otras clases de mala especificaci\'on funcional (por ejemplo, decidir si una variable ha de aparecer en nivel o en logaritmo) no lleva a contrastes que se salen del marco habitual del contraste de hip\'otesis. As\'i, para contrastar el modelo:

\[y=\beta_0+\beta_1x_1+\beta_2x_2+u\]

contra el modelo

\[y=\beta_0+\beta_1\log{x_1}+\beta_2\log{x_2}+u\]

Nos encontramos con el problema de que los modelos no est\'an anidados y por tanto no podemos usar el contraste est\'andar de la $F$.

Una posiblidad que se ha propuesto es proponer un modelo general que contenga a cada uno de los dos como un caso particular, del tipo:
\[y=\beta_0+\beta_1x_1+\beta_2x_2+\beta_3\log{x_1}+\beta_4\log{x_2}+u\]

y contrastar las restricciones que nos lleven a esos modelos. Por ejemplo, primero se puede contrastar $H_0:\beta_3=\beta_42=0$, que de cumplirse nos llevar\'a al primer modelo y despu\'es contrastar $H_0:\beta_1=\beta_2=0$ que nos define el segundo modelo. este modelo fue propuesto por Mizon y Richard.

Otro m\'etodo es el propuesto por Davidson y MacKinnon. Estos autores destacan que si el primer modelo es el correcto, entonces el valor ajustado del segndo no deber\'ia ser significativo e el primero. Por tanto, se estima el segundo modelo por MCO y obtenemos los valores ajustado para las $y$. A estos valores les llamamos $\hat{\hat{y}}$. El \textbf{contraste de Davidson-MacKinnon} se basa el el estad\'istico $t$ correspondiente a $\hat{\hat{y}}$ en el modelo:

\[y=\beta_0+\beta_1x_1+\beta_2x_2+\theta_1\hat{\hat{y}}+u\]
 
Un estad\'istico significativo (contra una alternativa de dos colas) supone un rechazo del primer modelo.

Del mismo modo, si llamamos $\hat{y}$ a los valores ajustados de la estimaci\'on del primer modelo, el contraste para el segundo modelo se basa en el estad\'istico $t$ de $\hat{y}$ en el modelo
\[y=\beta_0+\beta_1\log{x_1}+\beta_2\log{x_2}+\theta_1\hat{y}+u\]

Y un $t$ significativo rechaza el segundo modelo.

Ambos contrastes sirve para contrastar cualquier par de modelos no anidados con la misma variable dependiente.

Hay varios problemas a la hora de utilizar estos contrastes. Para empezar, puede que ninguno de los modelos se rechace. En este caso, podemos usar el $R^2$ ajustado para decidirnos por uno de ellos. Otra posibilidad es que se rechacen los dos, en cuyo caso habr\'a que buscar una alternativa. Sin embargo, en la p\'actica, si los efectos de las variables independientes m\'as importantes sobre la variable dependiente son similares, no tendr\'a importancia uqe modelo usar.

Otro problema es que rechazar un modelo no implica que el otro sea correcto, ya que el rechazo puede deberse a diversas formas de mala especificaci\'on funcional.

Por \'ultimo, un problema m\'as dif\'icil se produce cuando la duda en la especificaci\'on est\'a en la variable dependiente (por ejemplo, nivel frente a logaritmo). En este caso los contrastes que hemos visto no son v\'alidos.
