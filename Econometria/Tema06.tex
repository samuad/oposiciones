\chapter[Modelos din\'amicos.]{Modelos din\'amicos. \\
\normalsize  Justificaci\'on te\'orica de los modelos econom\'etricos
din\'amicos. Modelos de retardos infinitos. Estimaci\'on con retardos
de la variable end\'ogena. Contraste de exogeneidad de Asuman. Eficiencia
relativa de los estimadores de variables instrumentales. Estimaci\'on
de modelos con expectativas racionales.}


\sectioncol{Introducci\'on.}


Trataremos en este tema de aquellos modelos econom\'etricos en los que las relaciones entre la variable end\'ogena y las variables explicativas no son contempor\'aneas, sino que aparecen retardos de las mismas. Esto refleja situaciones en las que la influencia de una variable sobre otra solo se produce tras un cierto intervalo de tiempo o, a\'un siendo el impacto inmediato, este sigue influyendo durante un cierto n\'umero de per\'iodos.

Ejemplos de modelos din\'amicos ser\'ian los siguientes:

\begin{align*}
y_t=&\beta_0+\beta_1x_{t-1}+u_t\\
y_t=&\beta_0+\beta_1x_t+\beta_2x_{t-1}+\beta_3x_{t-2}+u_t\\
y_t=&\beta_0+\beta_1x_{t-2}+\beta_2x_{t-3}+u_t
\end{align*}

Adem\'as, las variables econ\'omicas tienen bastante inercia, lo que hace que una variable dependa de su propio pasado, adem\'as de otras causas. Es importante tener en cuenta que la existencia de una relaci\'on din\'amica entre variables depende del periodo de observaci\'on. Si una variable influye sobre otra contempor\'aneamente y con un retardo mensual, y tenemos valores trimestrales de las mismas, la influencia retardada no se ver\'a reflejada en nuestros datos.

\paragraph{Modelos de retardos distribuidos finitos.}

En un modelo de retardos distribu\'idos finitos, una o m\'as variables ex\'ogenas afectan a la variable dependiente con alg\'un retardo. Un ejemplo podr\'ia ser:

\[ y_t=\beta_0+\sum_{j=0}^{q}\delta_jx_{t-j}+u_t\]

Este modelo se utiliza para variables que influyen en la variable dependiente pero no de forma simult\'anea, sino con alg\'un retardo. El modelo que hemos puesto como ejempo es un modelo con retardos distribuidos finitos (RDF) de orden $q$.

Para interpretar este modelo, supongamos que $x$ es constante igual a una cantidad $c$, en  el instante  $t$ aumenta hasta $c+1$ y en el instante $t+1$ vuelve a bajar hasta $c$. En ese caso, el valor esperado de $y$ en cada instante ser\'ia:
\begin{align*}
 y_{t-1}=&\beta_0+\sum_{j=0}^{q}\delta_jc = y_{t-1} \\
 y_{t}=&\beta_0+\delta_0(c+1)+\sum_{j=1}^{q}\delta_jc = y_{t-1}+\delta_0 \\
 y_{t+1}=&\beta_0+\delta_0c+\delta_1(c+1)+\sum_{j=2}^{q}\delta_jc = y_{t-1}+\delta_1 \\
 y_{t+2}=&\beta_0+\delta_0c+\delta_1c+\delta2(c+1)+\sum_{j=3}^{q}\delta_jc = y_{t-1}+\delta_2 \\
 &\cdots \\
 y_{t+q}=&\beta_0+\sum_{j=0}^{q-1}\delta_jc+\delta_q(c+1) = y_{t-1} +\delta_q\\
  y_{t+q+1}=&\beta_0+\sum_{j=0}^{q}\delta_jc = y_{t-1}
\end{align*}

Por tanto, podemos ver que $\delta_0$ es el efecto inmediato que un cambio en $x_0$ tiene en $y$. Normalmente se denomina \textbf{propensi\'on al impacto} o \textbf{modificador de impacto}. Por otro lado, $\delta_1$ es el efecto en $y$ de un cambio en $x_0$ un per\'iodo despu\'es de que el cambio se produzca, $\delta_2$ es el efecto dos periodos despu\'es del cambio, etc . En el momento $t+q+1$ $y$ vuelve a su valor inicial, debido a que en nuestro modelo hemos supuesto $q$ retardos. Si realizamos un gr\'afico de $\delta_j$ respecto a $j$, obtenemos su distribuci\'on de retardos, que muestra el efecto que tiene sobre $y$ un cambio temporal en $x$. A esta sucesi\'on de efectos se le llama \textbf{funci\'on de respuesta al impulso}.

Si el cambio en $x$ fuese permanente, es decir, si $x$ pasa de valer $c$ a valer $c+1$ para $t, t+1, \ldots$, el efecto ser\'ia el siguiente:

\begin{align*}
 y_{t-1}=&\beta_0+\sum_{j=0}^{q}\delta_jc = y_{t-1} \\
 y_{t}=&\beta_0+\delta_0(c+1)+\sum_{j=1}^{q}\delta_jc = y_{t-1}+\delta_0 \\
 y_{t+1}=&\beta_0+\delta_0(c+1)+\delta_1(c+1)+\sum_{j=2}^{q}\delta_jc = y_{t-1}+\delta_0+\delta_1 \\
 y_{t+2}=&\beta_0+\delta_0(c+1)+\delta_1(c+1)+\delta2(c+1)+\sum_{j=3}^{q}\delta_jc = y_{t-1}+\delta_0+\delta_1+\delta_2 \\
 &\cdots \\
 y_{t+q}=&\beta_0+\sum_{j=0}^{q}\delta_j(c+1) = y_{t-1}+\sum_{j=0}^{q}\delta_j \\
 y_{t+q+1}=&\beta_0+\sum_{j=0}^{q}\delta_j(c+1) = y_{t-1}+\sum_{j=0}^{q}\delta_j 
\end{align*}

Por tanto, vemos que $\sum_{j=0}^{q}\delta_j$ es el cambio a largo plazo que experimenta $y$ tras un aumento permanente de una unidad en $x_0$ y se denomina \textbf{propensi\'on a largo plazo (PLP)} o \textbf{multiplicador a largo plazo} y es a menudo de inter\'es en estos modelos. A la sucesi\'on de valores $\delta_0, \delta_0+\delta_1,\ldots$ se le llama \textbf{funci\'on de respuesta al escal\'on}.

Como a menudo existe una correlaci\'on elevada entre los retardos de la variable independiente, estos modelos pueden presentar problemas de multicolinealidad, lo que hace que las estimaciones de los $\delta_i$ individuales sean muy imprecisas. Veremos que a\'un en este caso, a menudo podemos obtener buenos estimadores de la PLP.

Estos modelos pueden tener m\'as de una variable con retardos, variables contempor\'aneas, etc. Puede ocurrir que el objetivo al estimar el modelo sea contrastar si la variable independiente tiene efecto retardado sobre la variable dependiente.

El tratamiento de estos modelos difiere seg\'un que los valores retardados que aparecen como variables explicativas sean s\'olo d evariables ex\'ogenas, o haya tambi\'en retardos de la variable end\'ogena.

\subsectioncol{Cuando todos los retardos corresponden a variables ex\'ogenas.}

En este caso no tiene por que incumplirse las hip\'otesis del modelo lineal general, con lo que solo aparecen dos posibles dificultades:
\begin{itemize}
\item Los retardos consecutivos de una variable econ\'omica suelen estar correlacionados entre s\'i. Por tanto, pueden presentar multicolinealidad, que deber\'a ser tratada como corresponde.
\item Si el modelo es de retardos infinitos, no tenemos informaci\'on suficiente para estimarlo. En estos casos necesitamos imponer alguna restricci\'on sobre ls coeficientes que nos permita transformar el modelo para reducir el n\'umero de variables explicativas.
\end{itemize}

\subsectioncol{Cuando hay valores retardados de la variable end\'ogena.}

Veamos con un ejemplo c\'omo afectar\'a a las hip\'otesis del modelo. Si tenemos el modelo:
\[y_t=\beta y_{t-1}+u_t\]

Siendo $u_t$ un proceso de ruido blanco. El estimador MCO ser\'a:

\[\hat{\beta}=\dfrac{\sum_{i=2}^Ty_ty_{t-1}}{\sum_{i=2}^Ty_{t-1}^2}=\dfrac{\sum_{i=2}^T(\beta y_{t-1}+u_t)y_{t-1}}{\sum_{i=2}^Ty_{t-1}^2}=\beta+\dfrac{\sum_{i=2}^Tu_ty_{t-1}}{\sum_{i=2}^Ty_{t-1}^2}\]
\[E(\hat{\beta})=\beta+E\left(\dfrac{\sum_{i=2}^Tu_ty_{t-1}}{\sum_{i=2}^Ty_{t-1}^2}\right)\]

En principio $y_{t-1}$ es independiente de $u_t$, por lo que $E(y_{t-1}u_t)=0$, pero dado que desarrollando el modelo podemos ver que $y_t=\sum_{s=0}^{\infty}\beta^su_{t-s}$, y en el denominador hay valors de $y_t$ posteriores a los $u_t$ correspondientes, no podemos descomponer la esperanza, y por tanto, el estimador ser\'a sesgado, auque como veremos m\'as adelante, consistente.

Si el t\'ermino de error presenta autocorrelaci\'on, las $y_{t-1}$ estar\'an correlacionadas con $u_t$, con lo que se viola una de las hip\'otesis del modelo, y no podremos garantizar la consistencia del estimador.

\sectioncol{Justificaci\'on te\'orica de los modelos econom\'etricos din\'amicos.}

Tradicionalmente en econometr\'ia se utilizan dos explicaciones te\'oricas para justificar la existencia de este tipo de modelos.

\subsectioncol{El modelo de expectativas adaptativas.}

EN 1956 Cagan propuso un modelo que hac\'ia depender la demanda de saldos monetarios reales de la expectativa de inflaci\'on futura.
\[\dfrac{M_t}{P_t}=\beta_1+\beta_2E_t\pi_{t+1}+u_t\]

Nos encontramos con que la expectativa de inflaci\'on no es una variable que se pueda observar. Este modelo encierra distintas especificaciones seg\'un como propongamos que se forman las espectativas acerca de la tasa de inflaci\'on. Cagan utiliz\'o el mecanismo de expectativas adaptativas, propuesto por Friedman en su Teor\'ia del Consumo.

\[E_t\pi_{t+1}=E_{t-1}\pi_{t}+\lambda(\pi_t-E_{t-1}\pi_{t})\]

En el que la expectativa futura de inflaci\'on depende de la expectativa de inflaci\'on para el per\'iodo anterior modificada proporcionalmente al error de predicci\'on cometido. Este modelo tambi\'en puede escribirse
\[E_t\pi_{t+1}=\lambda\pi_t+(1-\lambda)E_{t-1}\pi_{t}\]

Si incorporamos las expectativas adaptativas al modelo de regresi\'on tenemos:

\[\dfrac{M_t}{P_t}=\lambda\beta_1+\lambda\beta_2\pi_t+(1-\lambda)\left(\dfrac{M_{t-1}}{P_{t-1}}\right)+v_t\]

Que es un modelo din\'amico. Estimando este modelo, se puede obtener $\lambda$ a partir de coeficiente de $\left(\dfrac{M_{t-1}}{P_{t-1}}\right)$, y $\beta_1$ y $\beta_2$ de forma similar.

\subsectioncol{El modelo de ajuste parcial de Nerlove.}

Supongamos que el nivel de capital deseado en la econom\'ia es funci\'on del nivel de producci\'on.
\[K_t^{*}=\beta_1+\beta_2Y_t+u_t\]

De manera an\'aloga al caso anterior, el nivel de capital deseado no es una variable observable. Para evitar esta dificultad, a\~nadimos el modelo de ajuste parcial del nivel de stock, que explica c\'omo se ajusta el nivel de capital real de una econom\'ia al nivel de capital deseado:
\[K_t-K_{t-1}=\delta(K_t^{*}-K_{t-1}),\quad0<\delta<1\]

Si introducimos el modelo en la ecuaci\'on, obtenemos:
\[K_t=\delta\beta_1+\delta_beta_2 Y_t+(1-\delta)K_{t-1}+\delta u_t\]

Que es un modelos din\'amico que una vez estimado nos permitir\'a calcular $\delta$, $\beta_1$ y $\beta_2$.



\subsectioncol{El modelo de expectativas adaptativas.}

\sectioncol{Modelos de retardos infinitos.}

En ocasiones, el modelo que queremos ajustar es del tipo:
\[y_t=\beta_1+\beta_2x_t+\beta_3x_{t-1}+\beta_4x_{t-2}+\beta_5x_{t-3}+\cdots+u_t\]

s decir, la variable end\'ogena depende de infinitos retardos de la variable independiente. Claramente este modelo no se puede estimar, ya que tenemos infinitos coeficientes. Para poder trabajar con el modelo ser\'a necesario suponer alguna estructura en los eçcoeficientes de los retardos de la variable end\'ogena que reduzca el n\'umero de par\'ametros a estimar.

\subsectioncol{El modelo de Koyck.}

Una hip\'otesis habituel es que $\beta_i=\delta\beta_{i-1}$, con $|\delta|<1$ y $i\geq3$. Aplicando esta hip\'otesis a nuestro modelo tenemos:
\[y_t=\beta_1+\beta_2x_t+\delta\beta_2x_{t-1}+\delta^2\beta_2x_{t-2}+\delta^3\beta_2x_{t-3}+\cdots+u_t=\beta_1+\beta_2\sum_{i=0}^{\infty}\delta^ix_{t-i}+u_t\]

Que depende solo de $\beta_1$, $\beta_2$ y $\delta$. La exigencia de que $|\delta|<1$ es para evitar la situaci\'on en que cuanto m\'as alejada est\'e una observaci\'on de $t$ m\'as influya en $y_t$. 

La propensi\'on al impacto es $\beta_2$, y la propendi\'on a largo plazo ser\'a $\beta_2\sum_{i=0}^{\infty}\limits\delta^i$ que como $|\delta|<1$ es $\dfrac{\beta_2}{1-\delta}$.

Si restamos de $y_t$ $\delta y_{y-1}$:
\[y_t-\delta y_{t-1}=(1-\delta)\beta_1+\beta_2x_t+u_t-\delta_u{t-1}\]

Y nuestro modelo queda:
\[y_t=(1-\delta)\beta_1+\beta_2x_t+\delta y_{t-1}+v_t\]

Que es un modelo con retardos de la variable dependiente. El problema es que, dado que $y_{t-1}$ depende de $u_{t-1}$, hay correlaci\'on entre una de las variables explicativas y el t\'ermino de error, con lo cual no se cumple una de las hip\'otesis del modelo. Adem\'as, si $u_t$ est\'a incorrelacionado, $E(v_tv_{t-1})=E[(u_t-\delta_u{t-1})(u_{t-1}-\delta_u{t-2})]=-\delta E(u_{t-1}^2)=-\delta\sigma_u^2$, y por tanto el modelo presentar\'a autocorrelaci\'on.

Si las $x_t$ presentan exogeneidad estricta, estar\'an incorreladas con $u_t$ y $u_{t-1}$ y por tanto con $v_t$, e igualmente $x_{t-1}$ estar\'a incorrelado con $v_t$. Dado que si $\beta_2\neq 0$ $y_{t-1}$ est\'a correlacionado con $x_{t-1}$, podemos usar $x_{t-1}$ como instrumento de $y_{t-1}$ y obtener un estimador de variables instrumentales. A\'un en este caso, tenemos que ajustar los errores est\'andar del estimador para tener en cuenta la autocorrelaci\'on del t\'ermino de error.
%%%
%Para estimar el modelo truncamos el polinomio de retardos en el primer contenido en la muestra ($t=1$), con lo que tendremos:
%
%\[y_t=\beta_1+\beta_2\sum_{i=0}^{t-1}\delta^ix_{t-i}+\delta^t\beta_2\sum_{i=0}^{\infty}\delta^ix_{-i}+u_t\]
%
%Si definimos:
%\begin{itemize}
%\item $ z_t=\sum_{i=0}^{t-1}\limits\delta^ix_{t-i}$, que podemos calcular para cada $t$, supuesto un valor de $\delta$.
%\item $ \gamma=\beta_2\sum_{i=0}^{\infty}\limits\delta^ix_{-i}$, que es desconocido, pero constante.
%\end{itemize}
%
%El modelo se transforma en:
%\[y_t=\beta_1+\beta_2z_t+\gamma\delta^t+u_t\]
%




\sectioncol{Estimaci\'on con retardos de la variable end\'ogena.}

En estos casos es fundamental distinguir entre si el t\'ermino de error tiene o no tiene autocorrelaci\'on.

\subsectioncol{El t\'ermino de error no tiene autocorrelaci\'on.}
En este caso tenemos un modelo del tipo:
\[y_t=\beta_1+\beta_2y_{t-1}+\beta_3x_t+u_t\]

que cumple las siguientes condiciones:

\begin{itemize}
\item $E(\boldsymbol{u})=\boldsymbol{0}_N$, $E(\boldsymbol{u}\boldsymbol{u}^{\prime})=\sigma_u^2\boldsymbol{I}_N$, es decir, no hay autocorrelaci\'on.
\item $E(x_tu_t)=0$, ya que si el modelo est\'a bien especificado, $u_t$ no refleja ninguna dependencia de $y_t$ sobre $u_t$.
\item $E(y_{t-1}u_t)=0$, por no haber autocorrelaci\'on.
\item $\plim\left(\dfrac{\boldsymbol{X}^{\prime}\boldsymbol{X}}{N}\right)=\boldsymbol{\Sigma}_{xx}$ matriz sim\'etrica definida positiva. Esta condici\'on se satisface en general siempre que $|\beta_2|<1$ si existen las varianzas y covarianzas de las variables explicativas.
\end{itemize}

Con estos supuestos el teorema de Mann-Wald asegura que $\plim\left(\dfrac{\boldsymbol{X}^{\prime}\boldsymbol{u}}{N}\right)=\boldsymbol{0}_k$, y que $\left(\dfrac{\boldsymbol{X}^{\prime}\boldsymbol{u}}{\sqrt{N}}\right)\overset{d}{\to}N\left(\boldsymbol{0}_k;\sigma_u^2\boldsymbol{\Sigma}_{xx}\right)$.

Por tanto,
\[\plim\hat{\boldsymbol{\beta}}_{MCO}=\plim\left[\boldsymbol{\beta}+\left(\dfrac{\boldsymbol{X}^{\prime}\boldsymbol{X}}{N}\right)^{-1}\left(\dfrac{\boldsymbol{X}^{\prime}\boldsymbol{u}}{N}\right)\right]=\boldsymbol{\beta}+\plim\left[\left(\dfrac{\boldsymbol{X}^{\prime}\boldsymbol{X}}{N}\right)^{-1}\right]\plim\left(\dfrac{\boldsymbol{X}^{\prime}\boldsymbol{u}}{N}\right)=\boldsymbol{\beta}+\boldsymbol{\Sigma}_{xx}^{-1}\boldsymbol{0}_k=\boldsymbol{\beta}\]

As\'i que el estimador MCO es consistente. Adem\'as, como 
\[\hat{\boldsymbol{\beta}}_{MCO}=\boldsymbol{\beta}+\left(\boldsymbol{X}^{\prime}\boldsymbol{X}\right)^{-1}\boldsymbol{X}^{\prime}\boldsymbol{u}\]
Se deduce que:
\[\sqrt{N}\left(\hat{\boldsymbol{\beta}}_{MCO}-\boldsymbol{\beta}\right)=\left(\dfrac{\boldsymbol{X}^{\prime}\boldsymbol{X}}{N}\right)^{-1}\left(\dfrac{\boldsymbol{X}^{\prime}\boldsymbol{u}}{\sqrt{N}}\right)\]
y por tanto,
\[\sqrt{N}\left(\hat{\boldsymbol{\beta}}_{MCO}-\boldsymbol{\beta}\right)\overset{d}{\to}\boldsymbol{\Sigma}_{xx}^{-1}N\left(\boldsymbol{0}_k;\sigma_u^2\boldsymbol{\Sigma}_{xx}\right)=N\left(\boldsymbol{0}_k;\sigma_u^2\boldsymbol{\Sigma}_{xx}^{-1}\right)\]

Y teniendo en cuenta la convergencia en distribuci\'on y la convergencia en probabilidad de $\left(\dfrac{\boldsymbol{X}^{\prime}\boldsymbol{X}}{N}\right)$, para muestras suficientemente grandes se puede aproximar:

\[\hat{\boldsymbol{\beta}}_{MCO}\sim N\left(\boldsymbol{\beta};\sigma_u^2\left(\boldsymbol{X}^{\prime}\boldsymbol{X}\right)\right)\]

En resumen, en ausencia de autocorrelaci\'on el estimador de m\'inimos cuadrados es consistente y converge en distribuci\'on a una normal, as\'i que para muestras suficientemente grandes est\'a justificado utilizarlo, as\'i como la matriz de covarianzas habitual y los contrastes de inferencia estad\'istica habituales. 

\subsectioncol{El t\'ermino de error tiene autocorrelaci\'on.}
En este caso tenemos un modelo del tipo:
\begin{align*}
y_t=&\beta_1+\beta_2y_{t-1}+\beta_3x_t+u_t\quad|\beta_2|<1\\
u_t=&\rho u_{t-1}+\varepsilon_t\quad|\rho|<1
\end{align*}

donde $\varepsilon_t$ es ruido blanco.

Como existe autocorrelaci\'on, ya no se cumple que $E(y_{t-1}u_t)=0$. Por tanto, el estimador MCO es sesgado y no es consistente, con lo que su sesgo no disminuir\'a al aumetar el tama\~no muestral. Esta inconsistencia aparecer\'a siempre que tengamos retardos de la variable end\'ogena y autocorrelaci\'on en el t\'ermino de error.

Para obtener estimaciones consistentes de un modelo de este tipo debemos acudir al \textit{estimador de variables instrumentales}. Una variable instrumental, que hace de instrumento para una variable explicativa del modelo, es una variable que satisface tres condiciones:
\begin{enumerate}
\item No est\'a correlacionada con el t\'ermino de error.
\item No est\'a incluida en el modelo como variable explicativa.
\item Est\'a correlacionada con la variable para la cual hace de instrumento.
\end{enumerate}

En nuestro modelo en particular, el primer retardo de la variable ex\'ogena satisface estas tres condiciones, ya que por las hip\'otesis del modelo est\'a incorrelacionada con $u_{t-1}$ y por tanto con $u_t$, no est\'a incluida en el modelo, y est\'a correlacioneda con $y_{t-1}$ a trav\'es del propio modelo.

Por tanto, tenemos nuestro vector de variables originales, $\boldsymbol{x}_t=(1, y_{t-1}, x_t)$, y el vector $\boldsymbol{z}_t=(1, x_{t-1}, x_t)$, en el que sustituimos las variables correlacionadas con el error por sus instrumentos.

De esta forma, definimos el estimador de variables instrumentales como:
\[\hat{\boldsymbol{\beta}}_{VI}=\left(\boldsymbol{Z}^{\prime}\boldsymbol{X}\right)^{-1}\boldsymbol{Z}^{\prime}\boldsymbol{y}\]

Donde por $\boldsymbol{Z}$ denotamos la matriz de observaciones correspondientes al vector $\boldsymbol{z}_t$, y suponemos que la matriz $\left(\boldsymbol{Z}^{\prime}\boldsymbol{X}\right)$ es invertible.

As\'i, 
\[\hat{\boldsymbol{\beta}}_{VI}=\left(\boldsymbol{Z}^{\prime}\boldsymbol{X}\right)^{-1}\boldsymbol{Z}^{\prime}\boldsymbol{y}=\left(\boldsymbol{Z}^{\prime}\boldsymbol{X}\right)^{-1}\boldsymbol{Z}^{\prime}\left[\boldsymbol{X}\boldsymbol{\beta}+\boldsymbol{u}\right]=\boldsymbol{\beta}+\left(\boldsymbol{Z}^{\prime}\boldsymbol{X}\right)^{-1}\boldsymbol{Z}^{\prime}\boldsymbol{u}\]

La correlaci\'on que debe existir entre el instrumento elegido y la variable explicativa que sustituye debe ser lo m\'as elevada posible pero no tan grade que provoque una correlaci\'on entre el instrumento y el t\'ermino de error.

Para que este estimador sea insesgado, se deber\'ia dar que $E\left[\left(\boldsymbol{Z}^{\prime}\boldsymbol{X}\right)^{-1}\boldsymbol{Z}^{\prime}\boldsymbol{u}\right]=\boldsymbol{0}_k$. aunque en general desconocemos la esperanza de esta expresi\'on. Sin embargo, podremos afirmar que el estimador es consistente bas\'andonos en la siguiente proposici\'on:
\begin{proposicion}
Sea $\boldsymbol{Z}$ una matriz $N\times k$ de observaciones, sea $\boldsymbol{z}_t$ la fila de esta matriz correspondiente al instante $t$, y supongamos que se cumple:
\begin{enumerate}
\item  $E(z_tu_t)=0$ para todo $t$.
\item $\plim\left(\dfrac{\boldsymbol{Z}^{\prime}\boldsymbol{X}}{N}\right)=\boldsymbol{\Sigma}_{zx}$, $\plim\left(\dfrac{\boldsymbol{Z}^{\prime}\boldsymbol{Z}}{N}\right)=\boldsymbol{\Sigma}_{zz}$, ambas no singulares y finitas.
\end{enumerate}
entonces el estimador de variables instrumentales es consistente.
\end{proposicion}

Veamos la demostraci\'on:
\begin{align*}
\plim\hat{\boldsymbol{\beta}}_{VI}=\boldsymbol{\beta}+\plim\left[\left(\dfrac{\boldsymbol{Z}^{\prime}\boldsymbol{X}}{N}\right)^{-1}\right]\plim\left(\dfrac{\boldsymbol{Z}^{\prime}\boldsymbol{u}}{N}\right)=\boldsymbol{\beta}+\boldsymbol{\Sigma}_{zx}^{-1}\boldsymbol{0}_k=\boldsymbol{\beta}
\end{align*}
ya que los instrumentos satisfacen $\plim\left(\dfrac{\boldsymbol{Z}^{\prime}\boldsymbol{u}}{N}\right)=0$

Para ver la distribuci\'on asint\'otica de nuestro estimador, necesitamos demostrar otro resultado:

\begin{proposicion}
Sea $\boldsymbol{Z}$ una matriz $N\times k$ de observaciones, sea $\boldsymbol{z}_t$ la fila de esta matriz correspondiente al instante $t$, y sean $\boldsymbol{X}$ y $\boldsymbol{x}_t$ la matriz y las filas correspondientes a las variables explicativas del modelo. Supongamos que se cumple:
\begin{enumerate}
\item  $E(z_tu_t)=0$ para todo $t$.
\item $\plim\left(\dfrac{\boldsymbol{Z}^{\prime}\boldsymbol{X}}{N}\right)=\boldsymbol{\Sigma}_{zx}$, no singular y finita 
\item $\plim\left(\dfrac{\boldsymbol{Z}^{\prime}\boldsymbol{Z}}{N}\right)=\boldsymbol{\Sigma}_{zz}$, sim\'etrica, definida positiva
\end{enumerate}
entonces para el estimador de variables instrumentales se tiene:
\[\sqrt{N}\left(\hat{\boldsymbol{\beta}}_{VI}-\boldsymbol{\beta}\right)\overset{d}{\to}N\left(\boldsymbol{0}_k;\sigma_u^2\left(\boldsymbol{\Sigma}_{zx}^{-1}\right)\boldsymbol{\Sigma}_{zz}\left(\boldsymbol{\Sigma}_{zx}^{-1}\right)^{\prime}\right)\]
\end{proposicion}

Como el teorema de Mann-Wals asegura que $\plim\left(\dfrac{\boldsymbol{Z}^{\prime}\boldsymbol{u}}{N}\right)=\boldsymbol{0}_k$, y que $\left(\dfrac{\boldsymbol{Z}^{\prime}\boldsymbol{u}}{\sqrt{N}}\right)\overset{d}{\to}N\left(\boldsymbol{0}_k;\sigma_u^2\boldsymbol{\Sigma}_{zz}\right)$, y como $\sqrt{N}\left(\hat{\boldsymbol{\beta}}_{VI}-\boldsymbol{\beta}\right)=\left(\dfrac{\boldsymbol{Z}^{\prime}\boldsymbol{X}}{N}\right)^{-1}\left(\dfrac{\boldsymbol{Z}^{\prime}\boldsymbol{u}}{\sqrt{N}}\right)$, queda demostrada la convergencia en distribuci\'on. Por tanto, para muestras grandes podemos utilizar como matriz de covarianzas del estinador VI:
\[Var\left(\hat{\boldsymbol{\beta}}_{VI}\right)=\dfrac{\sigma_u^2}{N}\left(\boldsymbol{\Sigma}_{zx}^{-1}\right)\boldsymbol{\Sigma}_{zz}\left(\boldsymbol{\Sigma}_{zx}^{-1}\right)^{\prime}\], y si utilizamos las matrices de momentos muestrales para aproximar sus l\'imites, tenemos que para tama\~nos de muestra grandes:
\[Var\left(\hat{\boldsymbol{\beta}}_{VI}\right)=\sigma_u^2\left(\boldsymbol{Z}^{\prime}\boldsymbol{X}\right)^{-1}\left(\boldsymbol{Z}^{\prime}\boldsymbol{Z}\right)\left[\left(\boldsymbol{Z}^{\prime}\boldsymbol{X}\right)^{-1}\right]^{\prime}\]

Un estimador consistente para $\sigma_u^2$ ser\'a:
\[\hat{\sigma}_u^2=\dfrac{\left(\boldsymbol{y}-\boldsymbol{X}\hat{\boldsymbol{\beta}}_{VI}\right)^{\prime}\left(\boldsymbol{y}-\boldsymbol{X}\hat{\boldsymbol{\beta}}_{VI}\right)}{N-k}\]

Siempre que el error no tenga autocorrelaci\'on. En nuestro caso, s\'i que la tiene, aunque debido a la dificultad de extender este resultado al caso con autocorrelaci\'on se suele utilizar la matriz de autocovarianzas a\'un sabiendo que es solo una aproximaci\'on. Un modelo de este tipo se deber\'ia estimar por m\'axima verosimilitud.

Es importante destacar que en la matriz de covaroanzas eel t\'ermino $\boldsymbol{Z}^{\prime}\boldsymbol{X}$, que es un estimador de la correlaci\'on entr elas variables ex\'ogenas y los instrumentos, est\'a dividiendo. Por tanto, a mayor correlaci\'on menor ser\'a la varianza del estimador de variables instrumentales.

\subsectioncol{El estimador de variable sinstrumentales en dos etapas.}

A la hora de obtener el estimador de variables instrumentales hemos supuesto que dispon\'iamos de un instrumento para cada variable que quer\'iamos sustituir. Sin embargo, lo habitual es disponer de m\'as instrumentos que variables explicativas, situaci\'on que se conoce como \textit{sobreidentificaci\'on}. En este caso, y dado que la matriz con la svariables instrumentales debe tener la misma dimensi\'on que la matriz original, podremos construir tantas variables instrumentaes como combinaciones lineales de instrumentos.

Dado que cualquier combinaci\'on lineal de los instrumentos es a su vez un instrumento v\'alido, y nos interesa minimizar la varianza de nuestro estimador, y hemos visto que a mayor correlaci\'on entre variable e instrumento menor ser\'a la varianza, un enfoque v\'alido ser\'ia calcular una regresi\'on de la variable a sustituir sobre los instrumentos, y utilizar como instrumento los resultados de la regresi\'on. De esta forma maximizamos la correlaci\'on entre la variable original y su instrumento.

Si hacemos esto, y utilizamos los nuevos instrumentos para obtener un estimador de variables isntrumentales, al estimador as\'i obtenido se le conoce como estimador de m\'inimos cuadrados en dos etapas, $\hat{\boldsymbol{\beta}}_{MC2E}$.

En general, la regresi\'on de todas las variables explicativas sobre todos los instrumentos, recogidos en la matriz $\boldsymbol{W}$, genera la matriz de coeficientes $\left(\boldsymbol{W}^{\prime}\boldsymbol{W}\right)^{-1}\boldsymbol{W}^{\prime}\boldsymbol{X}$, y genera la nueva matriz de variables ex\'ogenas $\hat{\boldsymbol{X}}=\boldsymbol{W}\left(\boldsymbol{W}^{\prime}\boldsymbol{W}\right)^{-1}\boldsymbol{W}^{\prime}\boldsymbol{X}$, que utilizadas como variables instrumentales conducen a la expresi\'on definitiva del estimador de m\'inimos cuadrados en dos etapas:
\[\hat{\boldsymbol{\beta}}_{MC2E}=\left[\boldsymbol{X}^{\prime}\boldsymbol{W}\left(\boldsymbol{W}^{\prime}\boldsymbol{W}\right)^{-1}\boldsymbol{W}^{\prime}\boldsymbol{X}\right]^{-1}\boldsymbol{X}^{\prime}\boldsymbol{W}\left(\boldsymbol{W}^{\prime}\boldsymbol{W}\right)^{-1}\boldsymbol{W}^{\prime}\boldsymbol{y}\]

Dado que no es m\'as que un estimador de variables instrumentales que utiliza como instrumentos $\hat{\boldsymbol{X}}$, su varianza ser\'a:
\[Var\left(\hat{\boldsymbol{\beta}}_{MC2E}\right)=\sigma_u^2\left(\hat{\boldsymbol{X}}^{\prime}\boldsymbol{X}\right)^{-1}\left(\hat{\boldsymbol{X}}^{\prime}\hat{\boldsymbol{X}}\right)\left[\left(\hat{\boldsymbol{X}}^{\prime}\boldsymbol{X}\right)^{-1}\right]^{\prime}\]

\sectioncol{Contraste de exogeneidad de Asuman.}

Dado que la no exogeneidad de las variables puede darse no solo por la aparici\'on de retardos de la variable end\'ogena, sino incluso con variables ex\'ogenas, es conveniente disponer de un contraste para comprobar la enxogeneidad de las variables explicativas.

Reordenamos nuestro modelo como:
\[\boldsymbol{y}=\boldsymbol{Y}_1\alpha+\boldsymbol{Z}_1\delta+\boldsymbol{u}\]

Donde $\boldsymbol{Y}_1$ es una matriz $N\times r$ con las observaciones de las $r$ variables para las que sospechamos que puede haber endogeneidad, y $\boldsymbol{Z}_1$ es una matriz $N\times (k-r)$ con las observaciones de las $k-r$ variables que sabemos seguro que son ex\'ogenas. SUpongamos que disponemos de una lista de instrumentos para las variables end\'ogenas, en caso de que se necesitaran.

La hip\'otesis nula del contraste es que todas las variables son ex\'ogenas, y la hip\'otesis alternativa que alguna de las $r$ primeras variables es end\'ogena. El contraste consiste en calcular el estimador MCO y obtener su suma residual, $SR_0$, y el estimador de variables instrumentales en dos etapas y obtener su suma residual, $SR_1$. A partir de estas sumas, obtenemso el estad\'istico:
\[\left(\hat{\boldsymbol{\beta}}_{MCO}-\hat{\boldsymbol{\beta}}_{MC2E}\right)^{\prime}\left[Var\left(\hat{\boldsymbol{\beta}}_{MCO}\right)-Var\left(\hat{\boldsymbol{\beta}}_{MC2E}\right)\right]^{-1}\left(\hat{\boldsymbol{\beta}}_{MCO}-\hat{\boldsymbol{\beta}}_{MC2E}\right)=\dfrac{SR_0-SR_1}{\hat{\sigma}_u^2}\sim\chi^2_{r}\]

bajo la hip\'otesis nula. As\'i, la regi\'on cr\'itica del contraste ser\'a aquella para la que el estad\'istico sea mayor que el valor $\chi^2_{r;\alpha}$.

Este tipo de contraste se puede aplicar siempre que tengamos dos estimadores, uno que bajo la hip\'otesis nula es consistente y eficiente, pero que es inconsistente bajo la hip\'otesis alternativa, $\hat{\boldsymbol{\beta}}_{1}$, y otro que es consistente para ambas hip\'otesis, pero es posiblemente menos eficiente, $\hat{\boldsymbol{\beta}}_{2}$. Bajo la hip\'otesis nula $\hat{\boldsymbol{q}}=\hat{\boldsymbol{\beta}}_{1}-\hat{\boldsymbol{\beta}}_{2}$ converge en probabilidad a cero, pero no bajo la hip\'otesis alternativa. Hausman prob\'o que si la distribuci\'on asint\'otica de $\sqrt{N}\left(\hat{\boldsymbol{\beta}}_{2}-\boldsymbol{\beta}\right)$ est\'a bien definida, entonces la covarianza asint\'otica de $\sqrt{N}\hat{\boldsymbol{q}}$ y $\sqrt{N}\left(\hat{\boldsymbol{\beta}}_{1}-\boldsymbol{\beta}\right)$ es igual a cero, y por tanto la matriz de varianzas y covarianzas de $\hat{\boldsymbol{q}}$ es $Var\left(\hat{\boldsymbol{q}}\right)=Var\left(\hat{\boldsymbol{\beta}}_{1}\right)-Var\left(\hat{\boldsymbol{\beta}}_{2}\right)$.

Por tanto el estad\'istico $\sqrt{N}\hat{\boldsymbol{q}}\sim N\left(\boldsymbol{0};Var\left(\hat{\boldsymbol{q}}\right)\right)$ y por tanto,
\[N\hat{\boldsymbol{q}}Var\left(\hat{\boldsymbol{q}}\right)\hat{\boldsymbol{q}}\sim\chi^2_p\]

Siendo $p$ el n\'umero de restricciones que se contrastan.

\sectioncol{Eficiencia relativa de los estimadores de variables instrumentales.}
\sectioncol{Estimaci\'on de modelos con expectativas racionales.}

